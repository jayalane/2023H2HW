\documentclass[11pt,oneside]{article}
\usepackage[hmargin=1in,vmargin=1in]{geometry}               % See geometry.pdf to learn the layout options. There are lots.
\geometry{letterpaper}                   % ... or a4paper or a5paper or ...
%\geometry{landscape}                % Activate for for rotated page geometry
%\usepackage[parfill]{parskip}    % Activate to begin paragraphs with an empty line rather than an indent
\usepackage{graphicx}
\usepackage{datetime}
\usepackage{amssymb}
\usepackage{epstopdf}
\usepackage{url}
%\usepackage{verbatim}
\usepackage{comment}
\specialcomment{solution}{\textbf{Solution. }}{}
%\excludecomment{solution}    %uncomment to remove solutions.

%\usepackage{enumerate}

%Use the enumitem package instead of enumerate
\usepackage[shortlabels]{enumitem}
%\usepackage{enumitem}
%then it will support the same suntax as the enumerate package.
%The enumerate package does not provide any extra configurations other than the label.

%\setlist[enumerate]{topsep=0pt,itemsep=-1ex,partopsep=1ex,parsep=1ex}
\setlist[enumerate]{topsep=0pt,partopsep=0pt}

\DeclareGraphicsRule{.tif}{png}{.png}{`convert #1 `dirname #1`/`basename #1 .tif`.png}
\usepackage{amsmath,amsthm,amscd,amssymb}
\usepackage{latexsym}
\usepackage[colorlinks,citecolor=red,pagebackref,hypertexnames=false]{hyperref}
\numberwithin{equation}{section}

\theoremstyle{definition}
\newtheorem{exercise}{Exercise}
%\newtheorem{solution}{Solution}
\newtheorem*{defn}{Definition}
\newtheorem*{claim}{Claim}

\def\pmod{+'}
\def\boldr{\boldsymbol{r}}
\def\boldc{\boldsymbol{c}}

\def\calA{\mathcal{A}}
\def\calB{\mathcal{B}}
\def\calC{\mathcal{C}}
\def\calT{\mathcal{T}}
\def\OR{\overline{\mathbb{R}}}
\def\RR{\mathbb{R}}
\def\CC{\mathbb{C}}
\def\FF{\mathbb{F}}
\def\QQ{\mathbb{Q}}
\def\ZZ{\mathbb{Z}}
\def\NN{\mathbb{N}}
%\def\NN{\mathbb{Z}_{> 0}}
\def\Nzero{\mathbb{Z}_{\geq 0}}
\def\EE{\mathbb{E}}
\def\PP{\mathbb{P}}
\def\supp{\mathrm{supp}}
\def\diam{\mathrm{diam}}
\def\sp{\mathrm{span}}
\def\ker{\mathrm{ker}}
\def\Aut{\operatorname{Aut}}
\def\Inn{\operatorname{Inn}}
\def\Ker{\operatorname{Ker}}
%\def\sp{\mathrm{span}} %messes up align enviroment
\newcommand{\Mod}{\ (\mathrm{mod}\ )}
\newcommand{\rbr}[1]{\left( {#1} \right)}
\newcommand{\sbr}[1]{\left[ {#1} \right]}
\newcommand{\cbr}[1]{\left\{ {#1} \right\}}
\newcommand{\abr}[1]{\left\langle {#1} \right\rangle}
\newcommand{\abs}[1]{\left| {#1} \right|}
\newcommand{\norm}[1]{\left\|#1\right\|}
\def\one{\mathbf{1}}
\DeclareMathOperator*{\esssup}{ess\,sup}
\newcommand*\wc{{}\cdot{}}
%\newcommand*\wc{ \, \cdot \,}
%wc for wildcard
\renewcommand{\Re}{\operatorname{Re}}
\renewcommand{\Im}{\operatorname{Im}}
\newcommand{\sgn}{\textup{sgn\,}}


\setlength{\parindent}{0pt}
\setlength{\parskip}{11pt}

%\title{\parbox{14cm}{\centering{  Interior points of circle and sphere packings}}}
\begin{document}

\textbf{Missed problem on homework and mid-term}

Date: \hhmmsstime{} \ \today \ \ Git hash: 
\input{/home/jayalane/hw/.git/refs/heads/main}

When I reviewed the mid-term, I was chagrined that I'd missed the same
thing I'd missed on a homework, moving from  $L(P, f) + L(P, g) \leq L(P, f+g)$ to
$ \underline \int _ a ^ b f + \underline \int _a ^b g \leq \underline \int _ a ^ b (f + g) $.

I had studied your admirably succinct solution to this, I just got
flustered on the final, but I want to at least show you I have it in
my mind, and hopefully inscribe it there so I'll remember it long
term. And so I can use it when I'm unsure about putting in an epsilon
vs. appealing to the definition of a sup.

\begin{exercise}
  Show
  
  \[ \underline \int _ a ^ b f + \underline \int _a ^b g \leq \underline \int _ a ^ b (f + g)
  \]
  given
  \[
  L(P, f) + L(P, g) \leq L(P, f+g)
  \]
\end{exercise}

\begin{solution}
  Let $\varepsilon > 0$ be given. Choose a partition $P$ of $[a,b]$ such that the following two inequalities hold:

  \[
  L(P, f) \geq \underline \int _ {a} ^ {b} f + \frac{\varepsilon }{2}
  \]
  and
  \[
  L(P, g) \geq \underline \int _ {a} ^ {b} g + \frac{\varepsilon }{2}
  \]

  Reversing and adding the first two together, we get

  \[
   \underline \int _ {a} ^ {b} f +  \underline \int _ {a} ^ {b} g + \varepsilon \leq L(P, f) + L(Q, f)
  \]

  Adding in the given inequality,

  \[
    L(P, f) + L(Q, f) \leq L(P, f+g)
  \]

  We get

  \[ 
     \underline \int _ {a} ^ {b} f +  \underline \int _ {a} ^ {b} g + \varepsilon \leq L(P, f) + L(Q, f) \leq L(P, f+g)
  \]

  Since $\underline \int _ {a} ^ {b} ( f + g)$ is a $\sup {} $ of $ \{ L(P, f+g) \} $ over all $P$, for a particular $P$,
    
    \[
    L(P, f+g) \leq \underline \int _ {a} ^ {b} ( f + g)
    \]
    
    Combining the last two, we get:

    \[
    \underline \int _ {a} ^ {b} f +  \underline \int _ {a} ^ {b} g + \varepsilon \leq L(P, f) + L(Q, f) \leq L(P, f+g) \leq  \underline \int _ {a} ^ {b} ( f + g)
    \]

    Since $\varepsilon$ was arbitrary,

    \[
    \underline \int _ {a} ^ {b} f +  \underline \int _ {a} ^ {b} g + \leq  \underline \int _ {a} ^ {b} ( f + g)
    \]
    
    \qed
  
\end{solution}

\begin{comment}
  \begin{exercise}
    problem
  \end{exercise}
  \begin{solution}
    \begin{enumerate}[(a)]
    \item
      first answer
    \end{enumerate}
  \end{solution}
\end{comment}

\end{document}
