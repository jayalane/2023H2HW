\documentclass[11pt,oneside]{article}
\usepackage[hmargin=1in,vmargin=1in]{geometry}               % See geometry.pdf to learn the layout options. There are lots.
\geometry{letterpaper}                   % ... or a4paper or a5paper or ...
%\geometry{landscape}                % Activate for for rotated page geometry
\usepackage[parfill]{parskip}    % Activate to begin paragraphs with an empty line rather than an indent
\usepackage{graphicx}
\usepackage{amssymb}
\usepackage{mathrsfs}
\usepackage{epstopdf}
\usepackage{datetime}
\usepackage{url}
%\usepackage{verbatim}
\usepackage{comment}
\specialcomment{solution}{\textbf{Solution. }}{}
%\excludecomment{solution}    %uncomment to remove solutions.

%\usepackage{enumerate}

%Use the enumitem package instead of enumerate
\usepackage[shortlabels]{enumitem}
%\usepackage{enumitem}
%then it will support the same suntax as the enumerate package.
%The enumerate package does not provide any extra configurations other than the label.

%\setlist[enumerate]{topsep=0pt,itemsep=-1ex,partopsep=1ex,parsep=1ex}
\setlist[enumerate]{topsep=0pt,partopsep=0pt}

\DeclareGraphicsRule{.tif}{png}{.png}{`convert #1 `dirname #1`/`basename #1 .tif`.png}
\usepackage{amsmath,amsthm,amscd,amssymb}
\usepackage{latexsym}
\usepackage[colorlinks,citecolor=red,pagebackref,hypertexnames=false]{hyperref}
\numberwithin{equation}{section}

\theoremstyle{definition}
\newtheorem{exercise}{Exercise}
%\newtheorem{solution}{Solution}
\newtheorem*{defn}{Definition}


\def\calA{\mathcal{A}}
\def\calB{\mathcal{B}}
\def\calC{\mathcal{C}}
\def\calT{\mathcal{T}}
\def\OR{\overline{\mathbb{R}}}
\def\RR{\mathbb{R}}
\def\CC{\mathbb{C}}
\def\FF{\mathbb{F}}
\def\QQ{\mathbb{Q}}
\def\ZZ{\mathbb{Z}}
\def\NN{\mathbb{N}}
\def\Nzero{\mathbb{Z}_{\geq 0}}
\def\EE{\mathbb{E}}
\def\PP{\mathbb{P}}
\def\supp{\mathrm{supp}}
\def\diam{\mathrm{diam}}
\def\sp{\mathrm{span}}
\def\ker{\mathrm{ker}}
\def\fancyA{\mathscr{A}}
\def\fancyU{\mathscr{U}}
\def\fancyU{\mathscr{U}}
\def\fancyL{\mathcal{L}}
\def\fancyV{\mathscr{V}}
\def\fancyP{\mathscr{P}}
\def\fancyB{\mathscr{B}}
\def\limn{\lim \limits _n}

\newcommand*\diff{\mathop{}\!\mathrm{d}}

\newcommand{\rbr}[1]{\left( {#1} \right)}
\newcommand{\sbr}[1]{\left[ {#1} \right]}
\newcommand{\cbr}[1]{\left\{ {#1} \right\}}
\newcommand{\abr}[1]{\left\langle {#1} \right\rangle}
\newcommand{\abs}[1]{\left| {#1} \right|}
\newcommand{\norm}[1]{\left\|#1\right\|}
\def\one{\mathbf{1}}
\DeclareMathOperator*{\esssup}{ess\,sup}
\newcommand*\wc{{}\cdot{}}
%\newcommand*\wc{ \, \cdot \,}
%wc for wildcard
\renewcommand{\Re}{\operatorname{Re}}
\renewcommand{\Im}{\operatorname{Im}}
\newcommand{\sgn}{\textup{sgn\,}}

\setlength{\parindent}{0pt}
\setlength{\parskip}{11pt
}
\newtheorem{lemma}{Lemma}

\begin{document}

\textbf{HW 11 - MATH 231A - Fall 2023 - Chris Lane}

Date: \hhmmsstime{} \ \today \ \ Git hash: 
\input{/Users/chlane/src/jayalane/2023H2HW/.git/refs/heads/main} 

\begin{exercise}
  (Embedding of $L^p$ Spaces).  Let $(X_i, \fancyA, \mu)$ be a measure space.
  Let $1 \leq p < q < \infty$.  Note $p, q$ are not necessarily
  conjugate exponents here.
  \begin{enumerate}[(a)]
  \item
    If $(X, \fancyA, \mu) = (\RR , \fancyB ( \RR), \lambda)$ then
    $L^q \nsubseteq L^p$ and     $L^p \nsubseteq L^q$.
  \item
    If $\mu(X) < \infty$, then $||f||_p \leq \mu(X)^{-p^{-1}-q^{-1}} || f|| _q$ and $L^q \subseteq L^p$. 
  \end{enumerate}

  
\end{exercise}
\begin{solution}
  \begin{enumerate}
  \item
    Using the hint, we need a $a$ such that $pa < 1$ and $qa > 1$.
    \[
    a = \frac{2}{q + p}
    \]
    To see this $a$ works, start with $p < q$ so $p + p < p + q$
    (adding $p$ to both sides) so $ \frac{p + p}{p + q} < 1$ (dividing
    by $p + q$).

    Also, $p < q$ so $ q + p < q + q$ (adding $q$ to both sides) so
    $1 < \frac{q + q}{q + p}$ (dividing by $p + q$).

    To show a function in $L^p$ but not in $L^q$, we'll set $f_1 = \one _ {(0, 1)} x^{-a}$.  

    For $f_1$, $|| f_1 || ^p _p = \int _\RR |f_1(x)| ^ p \diff x = \int _ 0^1 \frac{1}{x^{ap}} \diff x $. 
    Since $ap < 1$, this integral converges, so $f_1 \in L^p$.  However,
    $|| f_1 || ^q _q = \int _\RR |f_1(x)| ^ q \diff x = \int _ 0^1 \frac{1}{x^{aq}} \diff x $.
    Since $aq >1$, this integral diverges, so $f_2 \notin L^q$.

    We'll set $f_2 = \one _ {(1, \infty)}x^{-a}$ to show a function in $L_q$ but not in $L_p$.
    
    For $f_2$, $|| f_2 || ^p _p = \int _\RR |f_2(x)| ^ p \diff x = \int _ 1^\infty \frac{1}{x^{ap}} \diff x $. 
    Since $ap < 1$, and this is from $ 1 $ to $\infty$, this integral diverges, so $f_2 \notin L^p$.  Continuing, 
    $|| f_2 || ^q _q = \int _\RR |f_2(x)| ^ q \diff x = \int _ 1^\infty \frac{1}{x^{aq}} \diff x $.
    Since $aq >1$, and the integrationis from $1$ to $\infty$, this integral converges, so $f_2 \in L^q$.

    \qed
  \item
    TODO 

  \end{enumerate}
\end{solution}


\begin{exercise}
  Let $(X, \fancyA, \mu) $ be a measure space.  If $1 \leq p < q < r \leq \infty$, then
  \[
  L^q \subseteq L^p + L^r,
  \]

  in other words, each $f \in L^q$ is the sum of a function in $L^p$ and a function in $L^r$. 
\end{exercise}
\begin{solution}
  Let $E = \{ x \in X : | f(x) | > 1 \}$.  Let $f_1 = \one_E f$ and $f_2 = \one_{E^c} f$.

  Then we claim $f_1 \in L^p$ and $f_2 \in L^r$.  We have $|f_1(x)| > 1$, for all $x \in E$.  Also, 
  \[
  p < q
  \]
  Since $|f(x)| > 1$, for $x \in E$, $\ln (|f(x)|) > 0$, so multiplying preserves the sense of the inequality:
  \[
  p \ln (|f(x)|) < q \ln (|f(x)|)
  \]
  $g(x)$ where $g(x) = e^x$ is monotonic, so it may be applied to both side:
  \[
  |(f(x)|^p < |f(x)|^q
  \]
  Integrating both sides:
  \[
  \int_E |f(x)|^p < \int_E |f(x)|^q < \int_\RR |f(x)|^q < \infty
  \]
  So $f_1 \in L^p$.

  Similarly for $f_2$, $f_2(x) \leq 1$, so the inequality will change sense (and pick up
  an equals case for when $f_2(x) = 1$).  
  \[
  q < r
  \]
  Since $|f_2(x)| \leq 1$, $\ln (|f(x)|) \leq 0$, so multiplying reverses the sense of the inequality:
  \[
  q \ln (|f(x)|) >= r \ln (|f(x)|)
  \]
  $g(x)$ where $g(x) = e^x$ is monotonic, so it may be applied to both side:
  \[
  |(f(x)|^q >= |f(x)|^r
  \]
  Integrating both sides:
  \[
  \int_{E^c} |f(x)|^r < \int_{E^c} |f(x)|^q < \int_\RR |f(x)|^q < \infty
  \]
  So $f_2 \in L^r$.

  So for an arbitrary $f \in L^q$ we found $f_1 \in L^p$ and $f_2 \in L^r$ such that $f = f_1 + f_2$; $L^q \subseteq L^p + L^r$.

  TODO $r = \infty$.  
  
  
\end{solution}

\begin{exercise}
  Let $(X, \fancyA, \mu )$ be a measure space.  Let $S$ denote the set of measurable simple functions $s : X \to \RR$.  Let
  $S_0$ denote the set of measurable simple functions $s : X \to \RR$ that vanish outside a set of
  finite measure (i.e. that satisfy $\mu (\{ x: s(x) \neq 0\}) < \infty$).  
\begin{enumerate}[(a)]
\item
  If $1 \leq p \leq \infty$, $S \cap L^p = S_0$.
\item
  If $f : X \to \RR$ is measurable and $(X, \fancyA, \mu)$ is
  $\sigma-$finite, then there is a sequence $(s_n)$ in $S_0$ such that
  $s_n \to f$ on $X$ and $|s_n|\uparrow |f|$ on $X$.  Hint: Simple Approximation Theorem.
\item
  If $f \in L^p$, and $1 \leq p < \infty$, then there is a sequence $(s_n)$ in $S_0$ such that
  $s_n \to f$ on $X$ and $ |s_n| \uparrow |f|$ on $X$ and $|| s_n - f||_p \to 0$. Hint:
  first show $E = \{ x \in X: | f(x)| ^p \neq 0 \}$ is $\sigma-$finite.  
\end{enumerate}
  
\end{exercise}

\begin{solution}
\begin{enumerate}[(a)]
\item
  Let $s$ be a measurable simple function.  That is $s = \sum _{i=1}^{n} a_i \one _{A_i}$ for  $a_i \in \RR$ and $A_i \in \fancyA$.

  Suppose it is in $L^p$.  That is $|| s||_p < \infty$.  That is $\int |s| ^p < \infty$.
  First off, for $s$, $|s| =  \sum _{i=1}^{n} |a_i| \one _{A_i}$ as the $A_i$ are disjoint.
  And, again by disjointness, $|s|^p =  \sum _{i=1}^{n} |a_i|^p \one _{A_i}$.  
  So $\int |s|^p = \int \sum \left( |a_i|^p \one_{A_i} \right) = \sum |a_i|^p \int \one_{A_i} = \sum |a_i|^p \mu(A_i)$.
  $\sum _ {i=1}^{n} |a_i|^p \mu(A) < \infty$ if and only if, for all $i<n$, $\mu(A_i) < \infty$, i.e. $s \in S_0$.
  But then $\mu( \{ x : s(x) \neq 0 \}) =  \sum _{i=1}^{n} \mu( A_i ) < \infty$, so $s \in S_0$  
  
\item
  $f$ is measureable, so there is a sequence of simple functions, $(s_n)$ such that $s_n \uparrow f$.  Also, $X$ is $\sigma-$finite,
  so there is a sequence of set of finite measure, $M_n$ with $\mu( M_n) < \infty$, that
  cover $X$, $ X = \bigcup \limits _ {i=1}^{\infty} M_n$.  Define
  $t_n = \one_{M_n} s_n$.  $t_n$ is still simple, and still $t_n \uparrow f$, but because
   $\mu( M_n ) < \infty, n \in \NN$, $t_n \in S_0$.  
\item
  Let $E = \{ x \in X : | f(x) | ^p \neq 0 \}$.  To show it is $\sigma-$finite, we need a cover of
  sets of finite measure.  Let $E_n = \{ x \in X : |f(x)| ^ p \geq \frac{1}{n} \}$.  We can
  see that $E = \cup _ {n=1}^{\infty} E_n$.  Now, to understand $\mu(E_n)$, we notice that $n |f(x)| ^ p \geq 1$,
  so $\one_E(x) \leq n |f(x)|^p$, for every $x$.  So $\mu(E_n) = \int \one_{E_n} \leq \int n |f(x)| ^ p = n \int |f(x)| ^p \infty$ as
  $f \in L^p$.

  Also, analagous to (b), we have a simple sequence $(t_n)$ with $t_n \uparrow f$.  We can limit them to the
  $E_n$, let $s_n = t_n \one_{E_n}$, and still they pointwise converge to $f$.  Moreover, 
\end{enumerate}
\end{solution}

\begin{exercise}
  Let $X$ be a nonempty set.  Let $\mu$ be the counting measure on $(X, \fancyP(X))$. 
  \begin{enumerate}[(a)]
  \item
    Every $f : X \to \overline \RR$ is measurable. 
  \item
    If $f: X \to [0, \infty]$, then
\[
\int f \diff{\mu} = \sup \left\{ \sum _ {e \in F} f(x) : F \subseteq X, F \textrm{ is finite} \right\}
\]
  \item
    If $X = \{1, \hdots , n \}$, then $||f||_p = (\sum _{k=1} ^{n} | f(k)|^p)^{1/p}$
  \item
    If $X = \NN$, then $||f||_p = (\sum _{k=1} ^{\infty} | f(k)|^p)^{1/p}$
  \end{enumerate}
\end{exercise}
\begin{solution}
\begin{enumerate}[(a)]
\item
  The sets $\{ x \in X: f(x) \leq a \} $ are just the subsets of $X$ with fewer than $a$ members.
  If $a<=0$, it is $\varnothing$.  Since the $sigma-$algebra is the power set of $X$, the set of all subsets
  of given finite orders are a member of the $sigma-$algebra.  
\item
  TODO - I'm just not sure how to go from arbitrary simple functions to sums over finite sets; maybe some splitting it
  into finite and infinite measure cases. 
\item
  $||f||_p = ( \int |f|^p) ^{1/p} = (\int |f|^p  )^{1/p} = ( \sum_{k \in X} |f(k)|^p) ^{1/p}$ where the last step is by (b).  
\item
  By (b), the integral is the supremum of the sum for finite $F$. Since the quantities involved are all
  positive, and $\NN = \cup_{n=1}^{\infty} \{ 1, 2, \hdots , n \}$, the supremum is just the limit of the
  sum for (c),  $ ( \sum_{k \in X} |f(k)|^p ) ^{1/p}$.  
\end{enumerate}
\end{solution}

\begin{exercise}
  Let $(\Omega, \fancyA, P)$ be a probability space.  Let $X$ be an r.v. on $(\Omega, \fancyA, P)$.  The
  \textbf{distribution} of $X$ is the set function
  \[
  P_X = P(X^{-1}(B)) = P(X \in B)
  \]
  for each $B \in \fancyB(\RR)$.  Prove: the distribution of $X$, $P_X$,  is a probability measure on
  $(\RR, \fancyB(\RR)$. 
\end{exercise}
\begin{solution}
  There are three things to verify for a probability measure.  First, $P_X(\varnothing) = 0$.  This is to say,
  $ P(X^{-1}(\varnothing) = P(X \in \varnothing) = 0$ as $P$ is a measure.

  Secondly, $P_X(\RR) = P(X ^{-1}(\RR)) = P( \Omega ) = 1$ (as $P$ is from $\Omega$'s $\fancyA$ to $\RR$, the inverse
  is everythin).

  Thirdly, countable additivity of disjoint subsets.  First, the inverse image of disjoint subsets is disjoint; then
  the result follows from the countable additivity of $P$.  
\end{solution}

\begin{exercise}
  Let $X$ be an r.v. on a probability space $(\Omega, \fancyA, P)$.  Let $F$ be the cdf of $X$.  We use the notation
  $F(c-) = \lim _ {t \to c^-}F(t)$. Let $a \leq b$ be real numbers.  
  \begin{enumerate}[(a)]
  \item
    $P_X((a, b]) = F(b) - F(a)$
  \item
    $P_X((-\infty, a)) = F(a^-)$
  \item
    $P_X({a}) = F(a) - F(a^-)$
  \end{enumerate}
\end{exercise}

\begin{solution}
\begin{enumerate}[(a)]
\item
  As $P_X$ is a measure, $P_X((a, b]) = P_X((-\infty, b] \setminus (-\infty, a])$ Becuase P is a measure,
  that is $P_X((-\infty, b]) -  P_X((-\infty, a]) = F(b) - F(a)$.
          
\item
  $P_X (( - \infty, a)) = P_X \left( \bigcup \limits _{n = 1} ^{\infty} P_X ((- \infty, a - \frac{1}{n}]) \right)$.
  Again because $P$ is a measure, this is $\lim P_X((-\infty, a - \frac{1}{n}]) = F(a^-)$. 
\item
  $P_X ({a}) = P_X \left( (-\infty, a] \setminus (-\infty, a) \right) = F(a) - F(a^-1)$ by definition and (b) respectively. 
\end{enumerate}
\end{solution}

\begin{exercise}
  Uniqueness of the constructed probability measure with a given distribution.  
\end{exercise}
\begin{solution}
TODO
\end{solution}


\begin{comment}
\begin{exercise}
  problem
\end{exercise}
\begin{solution}
\begin{enumerate}[(a)]
\item
  first answer
\end{enumerate}
\end{solution}
\end{comment}


\end{document}
