\documentclass[11pt,oneside]{article}
\usepackage[hmargin=1in,vmargin=1in]{geometry}               % See geometry.pdf to learn the layout options. There are lots.
\geometry{letterpaper}                   % ... or a4paper or a5paper or ...
%\geometry{landscape}                % Activate for for rotated page geometry
\usepackage[parfill]{parskip}    % Activate to begin paragraphs with an empty line rather than an indent
\usepackage{graphicx}
\usepackage{amssymb}
\usepackage{mathrsfs}
\usepackage{epstopdf}
\usepackage{datetime}
\usepackage{url}
%\usepackage{verbatim}
\usepackage{comment}
\specialcomment{solution}{\textbf{Solution. }}{}
%\excludecomment{solution}    %uncomment to remove solutions.

%\usepackage{enumerate}

%Use the enumitem package instead of enumerate
\usepackage[shortlabels]{enumitem}
%\usepackage{enumitem}
%then it will support the same suntax as the enumerate package.
%The enumerate package does not provide any extra configurations other than the label.

%\setlist[enumerate]{topsep=0pt,itemsep=-1ex,partopsep=1ex,parsep=1ex}
\setlist[enumerate]{topsep=0pt,partopsep=0pt}

\DeclareGraphicsRule{.tif}{png}{.png}{`convert #1 `dirname #1`/`basename #1 .tif`.png}
\usepackage{amsmath,amsthm,amscd,amssymb}
\usepackage{latexsym}
\usepackage[colorlinks,citecolor=red,pagebackref,hypertexnames=false]{hyperref}
\numberwithin{equation}{section}

\theoremstyle{definition}
\newtheorem{exercise}{Exercise}
%\newtheorem{solution}{Solution}
\newtheorem*{defn}{Definition}


\def\calA{\mathcal{A}}
\def\calB{\mathcal{B}}
\def\calC{\mathcal{C}}
\def\calT{\mathcal{T}}
\def\OR{\overline{\mathbb{R}}}
\def\RR{\mathbb{R}}
\def\CC{\mathbb{C}}
\def\FF{\mathbb{F}}
\def\QQ{\mathbb{Q}}
\def\ZZ{\mathbb{Z}}
\def\NN{\mathbb{N}}
\def\Nzero{\mathbb{Z}_{\geq 0}}
\def\EE{\mathbb{E}}
\def\PP{\mathbb{P}}
\def\supp{\mathrm{supp}}
\def\diam{\mathrm{diam}}
\def\sp{\mathrm{span}}
\def\ker{\mathrm{ker}}
\def\fancyA{\mathscr{A}}
\def\fancyU{\mathscr{U}}
\def\fancyU{\mathscr{U}}
\def\fancyL{\mathcal{L}}
\def\fancyV{\mathscr{V}}
\def\fancyP{\mathscr{P}}
\def\fancyB{\mathscr{B}}
\def\limn{\lim \limits _n}
\newcommand{\rbr}[1]{\left( {#1} \right)}
\newcommand{\sbr}[1]{\left[ {#1} \right]}
\newcommand{\cbr}[1]{\left\{ {#1} \right\}}
\newcommand{\abr}[1]{\left\langle {#1} \right\rangle}
\newcommand{\abs}[1]{\left| {#1} \right|}
\newcommand{\norm}[1]{\left\|#1\right\|}
\def\one{\mathbf{1}}
\DeclareMathOperator*{\esssup}{ess\,sup}
\newcommand*\wc{{}\cdot{}}
%\newcommand*\wc{ \, \cdot \,}
%wc for wildcard
\renewcommand{\Re}{\operatorname{Re}}
\renewcommand{\Im}{\operatorname{Im}}
\newcommand{\sgn}{\textup{sgn\,}}

\setlength{\parindent}{0pt}
\setlength{\parskip}{11pt
}
\newtheorem{lemma}{Lemma}

\begin{document}

\textbf{HW 9 - MATH 231A - Fall 2023 - Chris Lane}

Date: \hhmmsstime{} \ \today \ \ Git hash: 
\input{/Users/chlane/src/jayalane/2023H2HW/.git/refs/heads/main} 

\begin{exercise}
  Consider the Lebesgue measure $\lambda$ on $\fancyL(\RR)$.  Show that
  \[
  \int \left| \frac{\sin(x)}{x} \right| \mathop{d\lambda(x)} = \infty \quad \textrm{and} \quad   \int \frac{\sin(x)}{x} \mathop{d\lambda(x)} \ \ \textrm{ is not defined}
  \]
  
\end{exercise}
\begin{solution}

  If $f(x) = \frac{\sin(x)}{x}$, then the two statements we desire to
  show are $\int f^+ = \infty$, and $\int f^- = \infty$.  We will show
  that they each are the limit of functions whose integrals exceed the
  harmonic series, and hence diverge.

  $f$ is symmetric with respect to sign, so we'll just look at the integral from
  $[0, \infty)$, as it has the same finiteness properties of its double.

    Let the region where $f+$ is non-zero be called $P$.

    \[
    P_n = \{ x \in \RR : x \in [ 2 \pi n, 2 \pi n + \pi) \} 
   \]

   and $ P = \cup _ { n \in \NN } P_n$.  
   
      Let the region where $f+$ is non-zero be called $N_n$.

  \[
    N_n = \{ x \in \RR : x \in [ 2 \pi n + \pi , 2 \pi (n + 1) \}
  \]

  and $N = \cup _ { n \in \NN} N_n$.
  
  So $n \in \NN$, $f(x) \geq 0$ for $ x \in P$ and $f(x) \leq 0$ for $ x \in N$.  

  So

      \[
      f^+ = \begin{cases}
        f(x), \ \ \textrm{ if } \ x \in P, \\
          0, \ \ \textrm{ otherwise }
      \end{cases}
      \]
      
      and

      \[
      f^- = \begin{cases}
        f(x), \ \ \textrm{ if } \ x \in N, \\
          0, \ \ \textrm{ otherwise }
      \end{cases}
      \]
      
      
 Now, $\int _ {P_n} |\sin x| = 2$ (tho as far
 as I know we can't prove that yet; at anyrate its some finite
 number).  Similarly, $\int_ {N_n} | \sin(x) | = 2 $.

 Now for our $f+$, for all $x$ in one of these segments $P_n$, $f(x) \leq \frac{\sin(x)}{2 \pi n + \pi}$.

 But $\int f^+ = \sum \limits _ { n \in \NN} \int _ {P_n} f^+$.  

 Now $ \one _ {P_n} f^+ = \frac{\sin(x)}{x} \geq \frac{\sin(x)}{2 \pi n + \pi}$ on $P_n$ (as $x \leq 2\pi n + \pi$ and we
 are dividing by the members of the inequality, and nothing is negative or zero).  
 
 So $ \int _ {P_n} f^+ \leq \int _ {P_n} \frac{\sin(x)}{2 \pi n + \pi} $.

 Then $ \int _{P_n} f^+ \leq \frac{2}/{2\pi n + \pi} $.

 Now, letting $f_n = \sum \limits _ {i = 0} ^ {n} \one _ { P_i} f^+$, we get

 $ \int f_n \geq \sum \limits _ {i = 0} ^ {n} \frac{2}{2 \pi i + \pi}$.

 But this sum $\sum \limits _ {i = 0} ^ {n} \frac{2}{2 \pi i + \pi}$ diverges as it is basically
 a harmonic series.

 Any limit $L$ say, has to be approached to within an $\varepsilon$ by the partial sum.

 \[
 L + \varepsilon > \sum \limits _ {i = 1} ^ {n} \frac{2}{2 \pi i + \pi}
 \]

 But then the sum from $n$ to $2n$ will be greater than the sum from $n$ to $2n$ of $\frac{2}{2 \pi (2n) + \pi}$.

 \begin{align*}
   \sum _{i=n+1}^{2n} \frac{2}{2 \pi (2n) + \pi} & \geq n \frac{2}{2\pi 2n + \pi} \\
   & \geq \frac{n}{4 \pi n + \pi} \\
   & \geq \frac{1}{4 \pi +  \frac{\pi}{n}} \\
   & \geq \frac{1}{4 \pi +  \frac{\pi}{n}} \\
 \end{align*}
 So if $\varepsilon < \frac{1}{4\pi + 1}$, the sum will exceed $L + \varepsilon$ and so $L$ cannot be a limit, so the sums diverge.

 So $\int f^+ = \infty$.

 A similar continuation of the $f^-$ logic shows that $\int f^- = \infty$.

 So, $\int \frac{\sin(x)}{x}$ is not defined nor finite.  The $\int \left| \frac{sin(x)}{x} \right| = \int f^+ + \int f^- = \infty + \infty = \infty$.

   \qed
  
\end{solution}

\begin{exercise}
  Given a measure space $(\RR, \fancyL(\RR), \lambda)$.  Let $a \in \RR$.  Let $f : [a, \infty) \to \RR$.
\begin{enumerate}[(a)]
\item
  Assume $f \one_{[a, b]}$ is measurable and $f$ is Lebesgue
  integrable on $[a, b]$ (meaning $\int_{[a, b]} f$ is defined and
  finite) for every $b \geq a$.  Assume that there exists
  an $M \in [0, \infty)$ such that $ \int |f| \mathop{d \lambda} \leq M$ for
  every $b \geq a$.  Then $f$ is Lebesgue integrable on $[a, \infty)$
  (meaning $\int _{[a, \infty)} f \mathop{d \lambda}$ is defined and finite and

      \[
      \int _ { [a, \infty)} f \mathop{d \lambda} = \lim \limits _ {b \to \infty} \int _{[a, b]} f \mathop{d \lambda}
      \]

\item
  Assume $f$ is Riemann integrable on $[a, b]$ for every $b \geq a$.
  Assume that there exists an $M \in [0, \infty)$ such
  that $\int _a^b|f| \mathop{d \lambda} \leq M$ for every
  $ b \geq a$.

  Then $f$ is Lebesgue integrable on $[a, \infty)$ and

  \[
  \int _ {[a,\infty)} f \mathop{d \lambda} = \lim \limits_{b \to \infty} \int _ {[a,b]} f \mathop{d \lambda} =
    \lim \limits _ {b \to \infty} \int  _ a ^ b f.
  \]
\end{enumerate}

\end{exercise}
\begin{solution}
\begin{enumerate}[(a)]
\item
  Let $f_n = \one _ { [a, a+n]} f^+$ and $g_n = \one_ { [a, a+n]} f^-$.

  Then $f^+ = \limn f_n$ and $0 \leq f_n \leq f_{n+1}$; as well $f^- = \limn g_n$ and $0 \leq g_n \leq g_{n+1}$. 

  Then $f = f^+ - f^- = \limn f_n - \limn g_n$

  The conditions for the Monotone Convergence Theorem apply to $f^+$
  and $f_n$ and $f^-$ and $g_n$ and we get that $f^+$ and $f^-$ are
  measurable and integrable.

  \begin{align*}
    \limn \int f_n &= \int f^+ \\
    \limn \int g_n  &= \int f^-
  \end{align*}

  As well, we have $\int _ { [a, a +n]} f^+ \leq \int _ { [a, a+n] } |f| \leq M$,
  so $\int f_n \leq M$, for all $n$.  Similarly, $\int_ {[a, b+n]} f^- \leq \int_ { [a, a+n]} |f| \leq M$, for all $n$.

  Since those inequalities are true for all $n$ they are true in the limit,

  $\limn \int f_n \leq M$ and $\limn \int g_n \leq M$.  
  
  So $\int f^+ \leq M$ and $\int f^- \leq M$ so $\int f$ is defined and
  is finite.

  \begin{align*}
  \int _ { [a, \infty) } f \mathop{d \lambda} &= \int _ { [a, \infty) } f^+ - \int_{[a, \infty]} f^- \\
      &= \limn \int _ { [a, a + n ] } f^+ - ( \limn \int_{[a, a + n ]} f^- ) \\
      &= \lim _ { b \to \infty }  \int _ {[a, b]} f^+ - ( \lim  _ { b \to \infty } \int_{[a, b]} f^- )\\
      &= \lim _ { b \to \infty }  \left( \int _ {[a, b]} f^+ -  \int_{[a, b]} f^-  \right) \\
      &= \lim _ { b \to \infty }  \int_{[a, b]} \left( f^+ -   f^-  \right) \\
      &= \lim _ { b \to \infty }  \int_{[a, b]} f \mathop{d \lambda}
  \end{align*}
  \qed

\item
  On $[a, b]$ if $f$ is Riemann integrable, then it is Lebesgue
  integrable as well, by the Riemann Lebesgue equivalence theorem.  

  Setting, as above, $f_n = \one _ { [a, a+n]} f$ and $g_n = \one _ {[a, a+n]} f$, we again
  get $ \int _{[a, \infty)} f \mathop{d \lambda} = \lim \limits _ { b \to \infty } \int f \mathop{d \lambda}$.

    For the second equality, since the Riemann integral and the Lebesgue integral are the same for all
    finite intervals $[a, b]$,

    \[
    \int _ [a, a+n] f \mathop{d \lambda} = R \int _ {[a, a+n]} f
    \]

    These integrals are equal for all $n \in \ZZ$ so they are equal in
    the limit; the limit is guaranteed to exist on the left hand side
    due to the existance of the integral in $[a, \infty)$ provided for
    by the first part.  Although the values inside the limit are
    Riemann integrals, taking the limit here isn't pushing anything under the
    Riemann integral, it's just a limit of a sequence (at each point).  

    I may be missing something as I don't see the need to show that $\int |f|$ is
    Riemann integral, as it's sort of given as a pre-condition on the existance of $M$.

\end{enumerate}
\end{solution}

\begin{exercise}
  Questions on the video lectures.  
\end{exercise}
\begin{solution}
\begin{enumerate}[(a)]
\item
  Is there really any use to the Riemann integral?  It seems much trickier to work with and offer no advantages over the Lebesgue integral?  
\item
  Is the value of $\sigma-$finiteness just that it enables the sorts of proofs where you are going from integrals and so on over a set of finite
  measure to some result over a set of infinite measure and you can just take the limit of the finite covering of finite measure sets? Or is it
  something else about the measures involved? 
\end{enumerate}
\end{solution}


\begin{comment}
\begin{exercise}
  problem
\end{exercise}
\begin{solution}
\begin{enumerate}[(a)]
\item
  first answer
\end{enumerate}
\end{solution}
\end{comment}


\end{document}
