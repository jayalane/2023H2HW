\documentclass[11pt,oneside]{article}
\usepackage[hmargin=1in,vmargin=1in]{geometry}               % See geometry.pdf to learn the layout options. There are lots.
\geometry{letterpaper}                   % ... or a4paper or a5paper or ...
%\geometry{landscape}                % Activate for for rotated page geometry
\usepackage[parfill]{parskip}    % Activate to begin paragraphs with an empty line rather than an indent
\usepackage{graphicx}
\usepackage{amssymb}
\usepackage{mathrsfs}
\usepackage{epstopdf}
\usepackage{datetime}
\usepackage{url}
%\usepackage{verbatim}
\usepackage{comment}
\specialcomment{solution}{\textbf{Solution. }}{}
%\excludecomment{solution}    %uncomment to remove solutions.

%\usepackage{enumerate}

%Use the enumitem package instead of enumerate
\usepackage[shortlabels]{enumitem}
%\usepackage{enumitem}
%then it will support the same suntax as the enumerate package.
%The enumerate package does not provide any extra configurations other than the label.

%\setlist[enumerate]{topsep=0pt,itemsep=-1ex,partopsep=1ex,parsep=1ex}
\setlist[enumerate]{topsep=0pt,partopsep=0pt}

\DeclareGraphicsRule{.tif}{png}{.png}{`convert #1 `dirname #1`/`basename #1 .tif`.png}
\usepackage{amsmath,amsthm,amscd,amssymb}
\usepackage{latexsym}
\usepackage[colorlinks,citecolor=red,pagebackref,hypertexnames=false]{hyperref}
\numberwithin{equation}{section}

\theoremstyle{definition}
\newtheorem{exercise}{Exercise}
%\newtheorem{solution}{Solution}
\newtheorem*{defn}{Definition}


\def\calA{\mathcal{A}}
\def\calB{\mathcal{B}}
\def\calC{\mathcal{C}}
\def\calT{\mathcal{T}}
\def\OR{\overline{\mathbb{R}}}
\def\RR{\mathbb{R}}
\def\CC{\mathbb{C}}
\def\FF{\mathbb{F}}
\def\QQ{\mathbb{Q}}
\def\ZZ{\mathbb{Z}}
\def\NN{\mathbb{N}}
\def\Nzero{\mathbb{Z}_{\geq 0}}
\def\EE{\mathbb{E}}
\def\PP{\mathbb{P}}
\def\supp{\mathrm{supp}}
\def\diam{\mathrm{diam}}
\def\sp{\mathrm{span}}
\def\ker{\mathrm{ker}}
\def\fancyA{\mathscr{A}}
\def\fancyU{\mathscr{U}}
\def\fancyU{\mathscr{U}}
\def\fancyL{\mathcal{L}}
\def\fancyV{\mathscr{V}}
\def\fancyP{\mathscr{P}}
\def\fancyB{\mathscr{B}}
\def\limn{\lim \limits _n}
\newcommand{\rbr}[1]{\left( {#1} \right)}
\newcommand{\sbr}[1]{\left[ {#1} \right]}
\newcommand{\cbr}[1]{\left\{ {#1} \right\}}
\newcommand{\abr}[1]{\left\langle {#1} \right\rangle}
\newcommand{\abs}[1]{\left| {#1} \right|}
\newcommand{\norm}[1]{\left\|#1\right\|}
\def\one{\mathbf{1}}
\DeclareMathOperator*{\esssup}{ess\,sup}
\newcommand*\wc{{}\cdot{}}
%\newcommand*\wc{ \, \cdot \,}
%wc for wildcard
\renewcommand{\Re}{\operatorname{Re}}
\renewcommand{\Im}{\operatorname{Im}}
\newcommand{\sgn}{\textup{sgn\,}}

\setlength{\parindent}{0pt}
\setlength{\parskip}{11pt
}
\newtheorem{lemma}{Lemma}

\begin{document}

\textbf{HW 7 - MATH 231A - Fall 2023 - Chris Lane}

Date: \hhmmsstime{} \ \today \ \ Git hash: 
\input{/Users/chlane/src/jayalane/2023H2HW/.git/refs/heads/main} 

\begin{exercise}
  Let  $(X, \fancyA)$ be a measurable space.  Let $E$ be measurable. 
  \begin{enumerate}[(a)]
  \item
    Let $f, g : E \to \overline \RR$ be measurable functions.  Prove $fg$ is measurable.
  \item
    Let  $f_n : E \to \overline \RR$ be a measurable function for each $n \in \NN$.  Prove
    $ \sup _n f_n$ is measurable.  
  \end{enumerate}
\end{exercise}
\begin{solution}
\begin{enumerate}[(a)]
\item
  \begin{lemma}
    If $f : E \to \overline \RR$ is measurable, then $f^2$, (defined as
    $f^2(x) = f(x) f(x)$ not $f(f(x))$), is measurable. 
  \end{lemma}
  \begin{proof}
    Given $\alpha \in \RR$ we want to show

    \[
    G = \{ x \in X : \overline f^2(x) \leq \alpha \} \in \fancyA
    \]

    There are three cases.  If $\alpha < 0$, then $G = \varnothing \in \fancyA$

    If $\alpha = 0$, then

    \[
    G = \{ x \in X: \overline f^2 (x) \leq 0 \} = \{ x \in E : f(x) = 0 \} \cup E^c
    \]

    Here $G \in \fancyA$ as the union of two measurable sets.

    For the final case, $\alpha > 0$, and $\sqrt{\alpha}$ exist and are real. We proceed like this:
    \begin{align*}
      \{ x \in X : f^2(x) \leq \alpha \} & = \{ x \in E : -\sqrt{\alpha} \leq f(x) \leq \sqrt{\alpha} \} \cup 
      E^c \\
      & = ( \{ x \in E : f(x) \geq -\sqrt{\alpha} \} \cup E^c ) \cap ( \{ x \in X : f(x) \leq \sqrt{\alpha} \} \cup E^c ) 
    \end{align*}

    As $f$ is measurable, each set there is in $\fancyA$ and so their finite unions, complements and intersections are in
    $\fancyA$.  Therefore $G \in \fancyA$, and $\overline f^2: X \to \overline \RR $ is measurable 
  \end{proof}

  Now we show a sequence of combining mesaureable functions to get the result that $\overline f \overline g: X \to \overline \RR$ is measureable:

  $f, g$ are measurable, given.

  So $(f + g)$ is measurable, $\overline f + \overline g = \overline {( f + g)} $

  From the above lemma, $f^2$ and $g^2$ are measurable, as are $-f^2$ and $-g^2$.

  $ (f+g)^2$ is measurable.

  $(f^2 + 2 fg + g^2)$ is measurable (just pointwise expanding prior line).

  $(f^2 + 2 fg + g^2 - f^2 - g^2)$ is measurable (adding in $-f^2$ and $-g^2$)

  $2 f g $ is measurable.

  $ 1/2 \times 2 f g$ is measurable.

  $ f g$ is measurable.  
  \qed
  
\item
  $\sup \limits _{n} f_n$

  First of all, note that if $x \in E^c$,

  $f_n(x) = 0 = \sup \limits _ {n} f_n$.

  I think perhaps that is sufficient to use the result for functions
  on $X$ to conclude the the $\sup$ is measurable, however, being more explicit:
  
  So given some $\alpha \in \RR$,

  \[
  G = \{ x \in X : \sup \limits _{n} f_n(x) >  \alpha \}
  \]

  we want to show $G \in \fancyA$.

  \[
  G = \bigcup \limits _ {n=1} ^ \infty \{ x \in X : \overline f_n(x) > \alpha \}
  \]

  Since the $\overline f_n$ are measurable,  $\{ x \in X : \overline f_x(x) > \alpha \} \in \fancyA$,
  and $G$, being the countable union of measurable sets, is measurable.
  \qed
\end{enumerate}
\end{solution}

\begin{exercise}
  Let $(X, \fancyA)$ be a measurable space.  Let $f, g : X \to \overline \RR$ be measurable
  functions.  
\begin{enumerate}[(a)]
\item
  Let $f, g : X \to \overline \RR$ be measurable functions.
  Prove $E = \{ x \in X : g(x) \neq 0 \}$ is measurable and that
  the function $f/g$ with domain $E$ is measurable. 
\item
  Let $f, g : X \to \overline \RR$ be measurable functions.
  Prove $E = X \\ \{ x \in X : f(x) = -g(x) = \infty\} \cup \{ x \in X : f(x) = -g(x) = - \infty \}$ is measurable
  and that the function with domain $E$ is measurable.  
\item
  Let $f_n, g : X \to \overline \RR$ be measurable functions for
  each $n \in \NN$.  In HW6, we showed $E = \{ x \in X : \limn f_n(x) \textrm{ converges go a number in } \RR \}$
  is measurable.  Prove that the function $\lim_n f_n$ with the domain $E$ is measurable.
\end{enumerate}
\end{exercise}
\begin{solution}
  \begin{enumerate}[(a)]
  \item
    $g$ is a measurable function, and the constant function $0$ is
    measurable, so by the lemma in class of 26 Sep 23, the
    set $E^c = \{ x \in X : g(x) = 0\} $ is measurable; complements of
    measurable sets being measurable, $E$ is measurable.

    Now, $f / g : E \to \overline \RR$ has domain $E$, as does $1 / g : E \to \overline E$.

    I'll first show $1/g$ is measurable on $E$, then use problem 1
    above to conclude $f/g$ is measurable.
    
    \begin{lemma}
    $1/g$ is measurable on $E$. 
    \end{lemma}
    \begin{proof}
    Given some $\alpha \in \RR$, we'll test for 

    \[
    G = \{ x \in \RR : \overline{\frac{1}{g(x)}}  < \alpha \} \in \fancyA
    \]

    There are three cases, depending on the sign of $\alpha$.  First, if $\alpha = 0$,

    \[
    \{ x \in \RR : \overline{\frac{1}{g(x)}} < 0 \}
    \]

    is the same as

    \[
    \{ x \in \RR: g(x) = 0 \} 
    \]

    ($\overline{\frac{1}{g}}$ is only zero in $X$ where $g(x) = 0$) which is clearly in $\fancyA$ as $g$ is measurable; this set is $E$.  

    Secondly, we have $\alpha < 0$.  (In this case $G \subseteq E$).  

    \begin{align*}
    G & =  \{ x \in \RR : \overline{\frac{1}{g(x)}} < \alpha \} \\
    & = \left( \{ x \in \RR : g(x) < 0 \} \cap \{ \frac{1}{g(x)} < \alpha \} \right) \bigcup
    \left( \{ x \in \RR : g(x) > 0 \} \cap \{ \frac{1}{g(x)} < \alpha \} \right)\\
    & = \left( \{ x \in \RR : g(x) < 0 \} \cap \{ \frac{1}{\alpha} < g(x) \} \right) \bigcup
    \left( \{ x \in \RR : g(x) > 0 \} \cap \{ \frac{1}{\alpha} > g(x) \} \right)
    \end{align*}

    (Because

    \[
    \alpha < 0, g < 0, \frac{1}{g} < \alpha, \textrm{ imply } 1 > \alpha g \textrm{ implies } \frac{1}{\alpha} < g
    \]

    and
    
    \[
    \alpha < 0, g > 0, \frac{1}{g} < \alpha, \textrm{ imply } 1 < \alpha g \textrm{ implies } \frac{1}{\alpha} > g
    \]
    
    But all these are measurable by the basic theorem from 26 Sep on
    the various types of tests for measurable functions. It's
    important to note that when we are given that $g$ is measurable,
    we can conclude all the tests are true; to prove $1/g$ is
    measurable we have to stick to one test.


    The third case is very similar.  We have $\alpha > 0$.  This means $E^c \subseteq G$.  

    \begin{align*}
    G & =  \{ x \in \RR : \overline{\frac{1}{g(x)}} < \alpha \} \\
    & = \left( \{ x \in \RR : g(x) < 0 \} \cap \{ \frac{1}{g(x)} < \alpha \} \right) \bigcup
    \left( \{ x \in \RR : g(x) > 0 \} \cap \{ \frac{1}{g(x)} < \alpha \} \right)\\
    & = \left( \{ x \in \RR : g(x) < 0 \} \cap \{ \frac{1}{\alpha} > g(x) \} \right) \bigcup
    \left( \{ x \in \RR : g(x) > 0 \} \cap \{ \frac{1}{\alpha} < g(x) \} \right)
    \end{align*}

    (Because

    \[
    \alpha > 0, g < 0, \frac{1/g} < \alpha, \textrm{ imply } 1 > \alpha g \textrm{ implies } \frac{1}{\alpha} > g
    \]

    and
    
    \[
    \alpha > 0, g > 0, \frac{1/g} < \alpha, \textrm{ imply } 1 < \alpha g \textrm{ implies } \frac{1}{\alpha} < g
    \]
    
    But all these are measurable by the basic theorem from 26 Sep on
    the various types of tests for measurable functions. It's
    important to note that when we are given that $g$ is measurable,
    we can conclude all the tests are true; to prove $1/g$ is
    measurable we have to stick to one test.

    In all three cases, $G \in \fancyA$

    \end{proof}
    So $f$ and $1/g$ are measurable on $E$. So by Exercise 1(a), their product $f/g$ is measurable on $E$.   

  \item
    \begin{align*}
      E & = X \setminus \left( \{ x \in X : f(x) = -g(x) = \infty \} \cup \{ x\in X : f(x) = -g(x) = -\infty \} \right) \\\
      & = X \setminus \left( ( \{ x \in X : f(x) = \infty \} \cap \{ x \in X : g(x) = -\infty \} ) \cup ( \{ x \in X : f(x) = -\infty \} \cap
      \{ x \in X : g(x) = \infty \} ) \right) \\
    \end{align*}
    Each of these are measurable sets of the form $ \{ x \in X : f(x) = a, a \in \overline \RR \} $ or intersections, unions, or set
    differences, hense measurable, hence $E$ is measurable.

    For $f + g$, it is clearly defined on $X \setminus E$.  Let $\alpha \in \RR$ be given.

    We need to show $ \{ x \in X : f(x) + g(x) < \alpha \} $.  We use the enumerate the rationals trick.

    Our set is equivalent to $ \{ x \in X : \textrm{ there exists } q_i \in \QQ \textrm{ such that } f(x) < q_i < \alpha - g(x) \}$.

    This works out to be

    $\bigcup \limits _{q_i \in \QQ} ( \{ x \in X : f(x) < q_i \} \cap \{ x \in X : g(x) < \alpha - q_i \} ) $
      
    Again, this is a countable union of intersections of measurable sets, and hence measurable.  \qed
    
  \item
    Given an $\alpha$, we will show that

    \[
    A = \{ x \in X : \limn f_n(x) < \alpha \}
    \]

    is measurable.  Let $q_k = \sup \limits _ { n geq k } f_n$.  This
    is measurable as a $\sup$ of measurable functions. 

    Then the intersection: 

    \[
    A = \bigcap \limits _ { n = 1} ^ \infty \{ x \in X : q_n(x) < \alpha \}
    \]

    shows that $A$ is measurable.  \qed
        
  \end{enumerate}
\end{solution}

\begin{exercise}
  Let $(X, \fancyA, \mu)$ be a measure space.
  \begin{enumerate}[(a)]
  \item
    Prove: If $f : X \to [0, \infty]$ and $c \in [0, \infty]$,
    then $\int c f \mathop{d\mu} = c \int f \mathop{d \mu}$.
  \item
    Prove: If $f : X \to \overline \RR$ is extended-integrable and $c \in \RR$, then
    $cf$ is extended integrable and $cf \mathop { d\mu} = c \int f \mathop{d \mu}$
  \item
    Give an example where $f : X \to \RR$ is integrable and $ c \in \RR$ but
    $\int c f$ is not defined.  
    
  \end{enumerate}
\end{exercise}

\begin{solution}
\begin{enumerate}[(a)]
\item
  The image of $f$ is in $[0, \infty]$, so $f^-(x) = 0$, for all $x \in X$.

    So $\int f = \int f^+$.  $\int f^+$ always exists as $f^-$ is non-negative and it gets
    defintion two for $\int f^+$, the $\sup$ which always exists in $[0, \infty]$

    Let $s$ an arbitrary member in the set

    \[
    \{  \int s : 0 \leq s \leq f, s \ \  \textrm{ simple } \}
    \]

    $ s = \sum \limits_{i = 1}^\infty a_i \one_{A_i} $ for some $a_i \in \RR$ and $A_i \in \fancyA$.

    So $cs = \sum \limits_{i = 1} ^ \infty c \one_{A_i}$.  $c$ is constant to the limit variables,
    so $cs = c \sum \limits_{i=1} ^ \infty \one_{A_i}$.

    So $\int cs = c \int s$ (where $\int$ is definition 1, for simple functions).

    Since $s$ was arbitrary and $c$ is just multiplying all the members of the set by $c$,
    the $\sup$ is just multiplied by $c$.  So $\int cf \mathop{d \mu} = c \int f \mathop{d \mu}$ as required.  
    

  
\item
  $\int f$ being extended integrable means at least one of $\int f^+$, $\int f^-$ is finite.

  Section (a) above applies to both $f^+$ and $f^-$ as they are both $ \geq 0$.

  So $\int cf = \int (cf)^+ - \int(cf)^-$.  Three cases based on $c$:

  If $c = 0$, then $(cf)(x) = 0$, for all $x \in \RR$, and all the integrals are zero and equal.

  If $c > 0$, then $(cf)^+ = cf^+$, $(cf)^- = cf^-$, and by the above exercise,
  $\int (cf+) = c \int f+$, and $\int (cf)^- = c \int f^-$, and $\int cf = c \int f$; if one
  of the integrals was $\infty$ or $- \infty$, $c$ times that integral is still the same.

  If $c < 0$, then $(cf)^+ = cf^-$, and $(cf)^- = cf^+$, and by the
  above exercise, $\int (cf+) = (-c) \int f^-$,
  and $\int (cf)^- = (-c) \int f^+$, and:

  \begin{align*}
    \int (cf) & = \int(cf) ^ + - \int(cf)^- \\
    & = (-c) \int^- - (-c) \int f^+ \\
    & = c \int f^+ - c \int f^- \\
    & = c \int f
  \end{align*}

  As required.  \qed  

  
\item
  Let $f : \RR \to \RR$ be defined:

  \[
  f(x) =  \one_{[0, 1]} - \one_{[-2, -1]}
  \]

  So $f^+ =  \one_{[0, 1]}$ and $\int f^+ = \mu([0,1]) = 1$.

  Likewise, $f^- =  \one_{[-2, -1]}$ and $\int f^- = \mu([0,1]) = 1$.

  So $\int f \mathop{d \mu} = 1 - 1 = 0$.

  However, if we set $c$ to $\infty$, then $ \int c f^+ = c \int f^+ = \infty \times 1 = \infty$ and
  also  $ \int c f^- = c \int f^- = \infty \times 1 = \infty$ and neither $\int f^-$ nor $\int f^-$ is finite, so
  $\int c f$ is not defined.  
  \qed
  
\end{enumerate}
\end{solution}

\begin{exercise}
  \begin{enumerate}[(a)]
  \item
    $(f_n) $ is increasing and $f_1$ is integrable.
    
    $\int f_1$ is integrable meaning $ \int f_1^+  < \infty $ and $\int f_1 ^- <  \infty$.
    
    So define $g_n = f_n + f_1^-$.  Now, $g_1 = f_1^+ - f_1^- + f_1^- = f_1+ \geq 0$.
    
    Now, $f_n \leq f_{n+1}$ so $f_n + f_1^- \leq f_{n+1} + f_1^-$, so
    
    \[
    0 \leq g_1 \leq g_2 \leq g_3 ...
    \]
    
    So the sequence $(g_n)$ satisfies the conditions for the Monotone convergence theorem, and
    
    \[
    \int g = \lim_{n \to \infty} \int g_n
    \]
    
    That is
    
    \begin{align*}
      \int \limn (f _n + f_1 ^ -) &= \limn \int (f_n + f_1^+) \quad \textrm { definition of } g_n \\ 
      \int \left[ \limn ( f_n) + \limn f_1^- \right] &= \limn \left[ \int f_n + \int f_1^- \right] \quad ,  \ \textrm{ as } \int a + b = \int a + \int b \textrm{ for measurable functions } \\
      \int \limn ( f_n) + \int \limn {f_1^-} &= \limn \int f_n + \limn \int f_1^- \quad \textrm{ pointwise, } f_1^- \textrm{ is constant w.r.t. } n \\
      \int \limn ( f_n) & = \limn \int f_n  \\
    \end{align*}
    
    \qed
    
  \item
    
    If $(f_n)$ is decreasing, and $f_1$ is integrable, and $f = \limn fn$, this case reduces to (a) by letting $g_n = - f_n$ and $g = \limn g_n$.  The integrability
    condition is the same, and the $f_1 \geq f_2 \geq f_3 ...$ transforms into $g_1 \leq g_2 \leq g_3 ...$.
    
    So (a) applies and $ \limn \int g_n = \int g$.  Then, multiplying by $-1$ again, we get
    
    \[
    \limn \int f_n = \int f
    \]
    
    as desired. \qed
    
  \item
    If $0 \leq f_n \leq f$ is given, the we let $g_n = \sup \limits _ {n \leq k} \{f_k\}$.  Then $0 \leq g_n$ and $g_1 \leq g_2 \leq ...$

    In this case, $g_n$ satisfy the requirements of the monotone convergence theorem, and

    \[
    \limn \int g_n = \int \limn g_n
    \]

    $ \limn g_n = \limn f_n$, as we are given $(f_n)$ converges,
    $\sup$ of the tail of the sequence will converge to the same
    point.


\end{enumerate}
  
\end{exercise}
\begin{solution}
\begin{enumerate}[(a)]
\item
  first answer
\end{enumerate}
\end{solution}
\begin{exercise}
  Let $(X, \fancyA, \mu)$ be a measure space.  Prove:
  If $f: X \to [0, \infty]$ is measurable and $ f < \infty$, then
  for every $\varepsilon > 0$ there is a $\delta > 0$ such that
  for every set $A \in \fancyA$ with $\mu(A) < \delta$ we
  have

  \[
  \int f \one_A < \varepsilon
  \]
  
\end{exercise}
\begin{solution}

  Given $ \varepsilon > 0, \varepsilon \in \RR$.  
  
  Set $f_n = \min(f, n)$.  
  
  Then we have $0 \leq f_1 \leq f_2 \leq ... \leq f$

  $f$ is measurable, and $f_1$ is bounded below, so the monotone convergence theorem applies and

  \[
  \limn \int f_n = \int \limn f_n = \int f
  \]

  So, by the definition of limits, there is a $j \in \NN$ such that

  \[
  \int f - \int f_j < \varepsilon
  \]

  Now, for this $j$, $X$ is partitioned into $ B = \{ x \in X : f(x) \leq j \}$
  and $ C = \{ x \in X : f(x) > j \} $.

  \[
  \int f_j = \int _ C n + \int _B f
  \]

  $ \int f < \infty$, say $\int f = N$.  So

  \[
  \int f_j = \int _C n + \int _B f
  \]

  Combined this we get

  \[
  \int _X f - \int _B f - \int _C n < \varepsilon
  \]
  
  However, $\int_x f - \int_B f  = \int_C f$. 

  So this becomes
  
  \[
  \int_C f - \int_C n < \varepsilon
  \]

  \[
  \int_C (f - n) < \varepsilon
  \]

  Now, for some general $A \in \fancyA$ with $\mu(A) < \mu(B)$,

  $\int f \one _ A $

  TODO
  
\end{solution}
\begin{exercise}
  \begin{enumerate}[(a)]
  \item
    Find a measurable function $f: \RR \to \RR$ that is continuous
    a.e. but 
  \item
    Find a measurable function $f: \RR \to \RR$ that is not equal to a
    continuous function almost everywhere but is not is continuous a.e.
  \end{enumerate}
\end{exercise}
\begin{solution}
  \begin{enumerate}[(a)]
  \item
    Let

    \[
    f = \begin{cases}
      0, x < 0 \\
      1, x \geq \\
    \end{cases}
    \]

    Then $f$ is continuous everywhere except at 0.  However, there is
    no measure zero set such that $f = g$ for a continuous $g$.  To
    show this, look at a neighborhood of $0$.  If a function $g = f$
    a.e. then any interval $I = [-\eta, 0]$, $\eta > 0$, of non-zero measure,
    will have to contain points $a \in [ -\eta, 0]$ where $g(a) = 0$; other
    wise $f \neq g$ on the whole of $I$, which has measure $\eta$.

    Likewise, any interval $J = [0, \varepsilon]$, of non-zero measure, will
    have to contain points $b \in [0, \varepsilon]$ where $g(b) = 1$.

    However, this means that for any $ \varepsilon < 1$, there does not exist a
    $\delta$ to establish continuity of $g$ at $0$:

    \[
    | x - 0 | < \delta \textrm{ implying } g(x) - g(0) < \varepsilon
    \]

    Hence $g$ cannot be continous at 0.  
    
    
  \item
    The classic counter-example $ f = \one_{\QQ}$ is useful here.

    $f(x) = 0$ for all $x \notin \QQ$.  Since $\lambda(\QQ) = 0$, this
    says $f$ and $0$ are equal almost everywhere, and $0$ is of course
    continuous.

    However, $f$ is discontinuous everywhere, as every open interval has
    points that map $1$ away from each other.  
    
  \end{enumerate}
\end{solution}
\begin{exercise}
  Let $(X, \fancyA, \mu)$ be a measure space.
  Let $f, g: X \to \overline \RR$ be measurable, integrable
  functions. Prove: $f=g$ a.e iff $\int _E f = \int _E g$
  for every $E \in \fancyA$.  
\end{exercise}
\begin{solution}
  First the forward ``if'' direction.

  Assume $f = g$ almost everywhere, given some $E \in \fancyA$.

  Let

  \[
  A = \{ x \in X : f(x) = g(x) \}
  \]

  and

  \[
  B = \{ x \in X : f(x) \neq g(x) \}
  \]

  The assumption means $\mu(B) = 0$.

  The $X = A \cup B$ so the given $E = ( E \cap A ) \cup (E \cap B) $

  So

  \begin{align*}
    \int _E f &= \int _ X f \one _ E \\
    &= \int _X f ( \one _ {E \cap A} + \one {E \cap B}) \\
    &= \int _X f \one _ { E \cap A} + \int f \one _ {E \cap B} \\
    &= \int _{E \cap A} f + \int _ {E \cap B} f  \\
  \end{align*}

  \begin{lemma}
    If $\mu(F) = 0$, and $f$ integrable, then
    $ \int _ F f = 0$.
  \end{lemma}

  \begin{proof}
    $f \one_ {E \cap B} = 0$, a.e., so by a theorem from class Tuesday,

    $ \int | f  \one_ {E \cap B} | \leq 0$.  Since $ | \int g | \leq \int | g|$, for $g$
    measurable, this means $ \int _F f = 0$.   
  \end{proof}

  So $\int _{E \cap B} f = 0$.  But on $E \cap A$, $f = g$, so $\int _ {E \cap A} f = \int _ {E \cap A} g$.

  Putting it together, $ \int _E f = \int _E g $.

  For the backward ``only if'':

  Hint is to do prove by contradiction. Assume there is an $A \subseteq X$ such that
  $\mu ( A) > 0$ and $ \int _ A ( f - g) = 0$.

  Let $A_n = (A \cap \{ x \in X : f(x) - g(x) > \frac{1}{n} \} $.

  Then $A = \bigcup \limits _n A_n$ and $A_1 \subseteq A_2 \subseteq A_3 ...$

  By , $\mu(A) =  \limn \mu(A_n) > 0$.  So, there exists $N \in \NN$ such that
  $\mu(A_j) > \frac{1}{N}$.  That is

  \[
  \mu(A \cap \{ x \in X : f(x) - g(x) > \frac{1}{N} \})
  \]

  But this gives rise to a simple function

  $ 0 \leq s \leq f$ such that

  $s = \frac{1}{N} \one _ {A_j}$ and $\int s \geq \frac{1}{N}$.

  Then, $\int _ {A} f - g > \frac{1}{N}$ by the definition of $\sup$.

  But this is a contraction, so there is no such $A$.  \qed
  
  
\end{solution}

\begin{comment}
\begin{exercise}
  problem
\end{exercise}
\begin{solution}
\begin{enumerate}[(a)]
\item
  first answer
\end{enumerate}
\end{solution}
\end{comment}


\end{document}

