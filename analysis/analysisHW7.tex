\documentclass[11pt,oneside]{article}
\usepackage[hmargin=1in,vmargin=1in]{geometry}               % See geometry.pdf to learn the layout options. There are lots.
\geometry{letterpaper}                   % ... or a4paper or a5paper or ...
%\geometry{landscape}                % Activate for for rotated page geometry
\usepackage[parfill]{parskip}    % Activate to begin paragraphs with an empty line rather than an indent
\usepackage{graphicx}
\usepackage{amssymb}
\usepackage{mathrsfs}
\usepackage{epstopdf}
\usepackage{datetime}
\usepackage{url}
%\usepackage{verbatim}
\usepackage{comment}
\specialcomment{solution}{\textbf{Solution. }}{}
%\excludecomment{solution}    %uncomment to remove solutions.

%\usepackage{enumerate}

%Use the enumitem package instead of enumerate
\usepackage[shortlabels]{enumitem}
%\usepackage{enumitem}
%then it will support the same suntax as the enumerate package.
%The enumerate package does not provide any extra configurations other than the label.

%\setlist[enumerate]{topsep=0pt,itemsep=-1ex,partopsep=1ex,parsep=1ex}
\setlist[enumerate]{topsep=0pt,partopsep=0pt}

\DeclareGraphicsRule{.tif}{png}{.png}{`convert #1 `dirname #1`/`basename #1 .tif`.png}
\usepackage{amsmath,amsthm,amscd,amssymb}
\usepackage{latexsym}
\usepackage[colorlinks,citecolor=red,pagebackref,hypertexnames=false]{hyperref}
\numberwithin{equation}{section}

\theoremstyle{definition}
\newtheorem{exercise}{Exercise}
%\newtheorem{solution}{Solution}
\newtheorem*{defn}{Definition}


\def\calA{\mathcal{A}}
\def\calB{\mathcal{B}}
\def\calC{\mathcal{C}}
\def\calT{\mathcal{T}}
\def\OR{\overline{\mathbb{R}}}
\def\RR{\mathbb{R}}
\def\CC{\mathbb{C}}
\def\FF{\mathbb{F}}
\def\QQ{\mathbb{Q}}
\def\ZZ{\mathbb{Z}}
\def\NN{\mathbb{N}}
\def\Nzero{\mathbb{Z}_{\geq 0}}
\def\EE{\mathbb{E}}
\def\PP{\mathbb{P}}
\def\supp{\mathrm{supp}}
\def\diam{\mathrm{diam}}
\def\sp{\mathrm{span}}
\def\ker{\mathrm{ker}}
\def\fancyA{\mathscr{A}}
\def\fancyU{\mathscr{U}}
\def\fancyL{\mathcal{L}}
\def\fancyV{\mathscr{V}}
\def\fancyP{\mathscr{P}}
\def\fancyB{\mathscr{B}}
\newcommand{\rbr}[1]{\left( {#1} \right)}
\newcommand{\sbr}[1]{\left[ {#1} \right]}
\newcommand{\cbr}[1]{\left\{ {#1} \right\}}
\newcommand{\abr}[1]{\left\langle {#1} \right\rangle}
\newcommand{\abs}[1]{\left| {#1} \right|}
\newcommand{\norm}[1]{\left\|#1\right\|}
\def\one{\mathbf{1}}
\DeclareMathOperator*{\esssup}{ess\,sup}
\newcommand*\wc{{}\cdot{}}
%\newcommand*\wc{ \, \cdot \,}
%wc for wildcard
\renewcommand{\Re}{\operatorname{Re}}
\renewcommand{\Im}{\operatorname{Im}}
\newcommand{\sgn}{\textup{sgn\,}}

\setlength{\parindent}{0pt}
\setlength{\parskip}{11pt
}
\newtheorem{lemma}{Lemma}

\begin{document}

\textbf{HW 7 - MATH 231A - Fall 2023 - Chris Lane}

Date: \hhmmsstime{} \ \today \ \ Git hash: 
\input{/Users/chlane/src/jayalane/2023H2HW/.git/refs/heads/main} 

\begin{exercise}
  Let  $(X, \fancyA)$ be a measurable space.  Let $E$ be measurable. 
  \begin{enumerate}[(a)]
  \item
    Let $f, g : E \to \overline \RR$ be measurable functions.  Prove $fg$ is measurable.
  \item
    Let  $f_n : E \to \overline \RR$ be a measurable function for each $n \in \NN$.  Prove
    $ \sup _n f_n$ is easurable.  
  \end{enumerate}
\end{exercise}
\begin{solution}
\begin{enumerate}[(a)]
\item
  \begin{lemma}
    If $f : E \to \overline \RR$ is measurable, then $f^2$, (defined as
    $f^2(x) = f(x) f(x)$ not $f(f(x))$), is measurable. 
  \end{lemma}
  \begin{proof}
    Given $\alpha \in \RR$ we want to show

    \[
    G = \{ x \in X : \overline f^2(x) \leq \alpha \} \in \fancyA
    \]

    There are three cases.  If $\alpha < 0$, then $G = \varnothing \in \fancyA$

    If $\alpha = 0$, then

    \[
    G = \{ x \in X: \overline f^2 (x) \leq 0 \} = \{ x \in E : f(x) = 0 \} \cup E^c
    \]

    Here $G \in \fancyA$ as the union of two measurable sets.

    For thhe final case, $\alpha > 0$, and $\sqrt{\alpha}$ exist and are real. We proceed like this:
    \begin{align*}
      \{ x \in X : f^2(x) \leq \alpha \} & = \{ x \in E : -\sqrt{\alpha} \leq f(x) \leq \sqrt{\alpha} \} \cup 
      E^c \\
      & = ( \{ x \in E : f(x) \geq -\sqrt{\alpha} \} \cup E^c ) \cap { f(x) \leq \sqrt{\alpha} } \cup E^c
    \end{align*}

    Each set there is in $\fancyA$ and so their finite unions, complements and intersections are in
    $\fancyA$.  Therefore $G \in \fancyA$, and $\overline f^2: X \to \overline \RR $ is measurable 
  \end{proof}

  Now we show a sequence of combining mesaureable functions to get the result that $\overline f \overline g: X \to \overline \RR$ is measureable:

  $f, g$ are measurable, given.

  So $(f + g)$ is measurable, $\overline f + \overline g = \overline {( f + g)} $

  From the above lemma, $f^2$ and $g^2$ are measurable, as are $-f^2$ and $-g^2$.

  $ (f+g)^2$ is measurable.

  $(f^2 + 2 fg + g^2)$ is measurable (just pointwise expanding prior line).

  $(f^2 + 2 fg + g^2 - f^2 - g^2)$ is measurable (adding in $-f^2$ and $-g^2$)

  $2 f g $ is measurable.

  $ 1/2 \times 2 f g$ is measurable.

  $ f g$ is measurable.  
  \qed
  
\item
  $\sup \limits _{n} f_n$

  First of all, note that if $x \in E^c$,

  $f_n(x) = 0 = \sup \limits _ {n} f_n$.

  I think perhaps that is sufficient to use the result for functions
  on $X$ to conclude the the $\sup$ is measurable, however, being more explicit:
  
  So given some $\alpha \in \RR$,

  \[
  G = \{ x \in X : \sup \limits _{n} f_n(x) >  \alpha \}
  \]

  we want to show $G \in \fancyA$.

  \[
  G = \bigcup \limits _ {n=1} ^ \infty \{ x \in X : \overline f_x(x) > \alpha \}
  \]

  Since the $\overline f_n$ are measurable,  $\{ x \in X : \overline f_x(x) > \alpha \} \in \fancyA$,
  and $G$, being the countable union of measurable sets, is measurable.
  \qed
\end{enumerate}
\end{solution}

\begin{exercise}
  Let $(X, \fancyA)$ be a measurable space.  Let $f, g : X \to \overline \RR$ be measurable
  functions.  
\begin{enumerate}[(a)]
\item
  Let $f, g : X \to \overline \RR$ be measurable functions.
  Prove $E = \{ x \in X : g(x) \neq 0 \}$ is measurable and that
  the function $f/g$ with domain $E$ is measurable. 
\item
  Let $f, g : X \to \overline \RR$ be measurable functions.
  Prove $E = X \\ \{ x \in X : f(x) = -g(x) = \infty\} \cup \{ x \in X : f(x) = -g(x) = - \infty \}$ is measurable
  and that the function with domain $E$ is measurable.  
\item
  Let $f_n, g : X \to \overline \RR$ be measurable functions for
  each $n \in \NN$.  In HW6, we showed $E = \{ x \in X : lim_n f_n(x) \textrm{ converges go a number in } \RR \}$
  is measurable.  Prove that the function $\lim_n f_n$ with the domain $E$ is measurable.

  
\end{enumerate}
\end{exercise}
\begin{solution}
  \begin{enumerate}[(a)]
  \item
    $g$ is a measurable function, and the constant function $0$ is
    measurable, so by the lemma in class of 26 Sep 23, the
    set $E^c = \{ x \in X : g(x) = 0\} $ is measurable; complements of
    measurable sets being measurable, $E$ is measurable.

    Now, $f / g : E \to \overline \RR$ has domain $E$, as does $1 / g : E \to \overline E$.

    I'll first show $1/g$ is measurable on $E$, then use problem 1
    above to conclude $f/g$ is measurable.
    
    \begin{lemma}
    $1/g$ is measurable on $E$.  
    \end{lemma}
    \begin{proof}
    Given some $\alpha \in \RR$, we'll test for 

    \[
    G = \{ x \in \RR : \overline{\frac{1}{g(x)}}  < \alpha \} \in \fancyA
    \]

    There are three cases, depending on the sign of $\alpha$.  First, if $\alpha = 0$,

    \[
    \{ x \in \RR : \overline{\frac{1}{g(x)}} < 0 \}
    \]

    is the same as

    \[
    \{ x \in \RR: g(x) < 0 \} 
    \]

    which is clearly in $\fancyA$ as $g$ is measurable.

    Secondly, we have $\alpha < 0$.  This means $G \subseteq E$.

    \begin{align*}
    G & =  \{ x \in \RR : \overline{\frac{1}{g(x)}} < \alpha \} \\
    & = \left( \{ x \in \RR : g(x) < 0 \} \cap \{ \frac{1}{g(x)} < \alpha \} \right) \bigcup
    \left( \{ x \in \RR : g(x) > 0 \} \cap \{ \frac{1}{g(x)} < \alpha \} \right)\\
    & = \left( \{ x \in \RR : g(x) < 0 \} \cap \{ \frac{1}{\alpha} < g(x) \} \right) \bigcup
    \left( \{ x \in \RR : g(x) > 0 \} \cap \{ \frac{1}{\alpha} > g(x) \} \right)
    \end{align*}

    (Because

    \[
    \alpha < 0, g < 0, \frac{1/g} < \alpha, \textrm{ imply } 1 > \alpha g \textrm{ implies } \frac{1}{\alpha} < g
    \]

    and
    
    \[
    \alpha < 0, g > 0, \frac{1/g} < \alpha, \textrm{ imply } 1 < \alpha g \textrm{ implies } \frac{1}{\alpha} > g
    \]
    
    But all these are measurable by the basic theorem from 26 Sep on
    the various types of tests for measurable functions. It's
    important to note that when we are given that $g$ is measurable,
    we can conclude all the tests are true; to prove $1/g$ is
    measurable we have to stick to one test.


    The third case is very similar.  We have $\alpha > 0$.  This means $E^c \subseteq G$.  

    \begin{align*}
    G & =  \{ x \in \RR : \overline{\frac{1}{g(x)}} < \alpha \} \\
    & = \left( \{ x \in \RR : g(x) < 0 \} \cap \{ \frac{1}{g(x)} < \alpha \} \right) \bigcup
    \left( \{ x \in \RR : g(x) > 0 \} \cap \{ \frac{1}{g(x)} < \alpha \} \right)\\
    & = \left( \{ x \in \RR : g(x) < 0 \} \cap \{ \frac{1}{\alpha} > g(x) \} \right) \bigcup
    \left( \{ x \in \RR : g(x) > 0 \} \cap \{ \frac{1}{\alpha} < g(x) \} \right)
    \end{align*}

    (Because

    \[
    \alpha > 0, g < 0, \frac{1/g} < \alpha, \textrm{ imply } 1 > \alpha g \textrm{ implies } \frac{1}{\alpha} > g
    \]

    and
    
    \[
    \alpha > 0, g > 0, \frac{1/g} < \alpha, \textrm{ imply } 1 < \alpha g \textrm{ implies } \frac{1}{\alpha} < g
    \]
    
    But all these are measurable by the basic theorem from 26 Sep on
    the various types of tests for measurable functions. It's
    important to note that when we are given that $g$ is measurable,
    we can conclude all the tests are true; to prove $1/g$ is
    measurable we have to stick to one test.

    In all three cases, $G \in \fancyA$

    \end{proof}
    So $f$ and $1\g$ are measurable on $E$. So by Exercise 1(a), their product $f/g$ is measurable on $E$.   

  \item
    TODO
    \item TODO
  \end{enumerate}
\end{solution}
 

\begin{comment}
\begin{exercise}
  problem
\end{exercise}
\begin{solution}
\begin{enumerate}[(a)]
\item
  first answer
\end{enumerate}
\end{solution}
\end{comment}


\end{document}

