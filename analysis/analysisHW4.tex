\documentclass[11pt,oneside]{article}
\usepackage[hmargin=1in,vmargin=1in]{geometry}               % See geometry.pdf to learn the layout options. There are lots.
\geometry{letterpaper}                   % ... or a4paper or a5paper or ...
%\geometry{landscape}                % Activate for for rotated page geometry
\usepackage[parfill]{parskip}    % Activate to begin paragraphs with an empty line rather than an indent
\usepackage{graphicx}
\usepackage{amssymb}
\usepackage{mathrsfs}
\usepackage{epstopdf}
\usepackage{url}
%\usepackage{verbatim}
\usepackage{comment}
\specialcomment{solution}{\textbf{Solution. }}{}
%\excludecomment{solution}    %uncomment to remove solutions.

%\usepackage{enumerate}

%Use the enumitem package instead of enumerate
\usepackage[shortlabels]{enumitem}
%\usepackage{enumitem}
%then it will support the same suntax as the enumerate package.
%The enumerate package does not provide any extra configurations other than the label.

%\setlist[enumerate]{topsep=0pt,itemsep=-1ex,partopsep=1ex,parsep=1ex}
\setlist[enumerate]{topsep=0pt,partopsep=0pt}

\DeclareGraphicsRule{.tif}{png}{.png}{`convert #1 `dirname #1`/`basename #1 .tif`.png}
\usepackage{amsmath,amsthm,amscd,amssymb}
\usepackage{latexsym}
\usepackage[colorlinks,citecolor=red,pagebackref,hypertexnames=false]{hyperref}
\numberwithin{equation}{section}

\theoremstyle{definition}
\newtheorem{exercise}{Exercise}
%\newtheorem{solution}{Solution}
\newtheorem*{defn}{Definition}


\def\calA{\mathcal{A}}
\def\calB{\mathcal{B}}
\def\calC{\mathcal{C}}
\def\calT{\mathcal{T}}
\def\OR{\overline{\mathbb{R}}}
\def\RR{\mathbb{R}}
\def\CC{\mathbb{C}}
\def\FF{\mathbb{F}}
\def\QQ{\mathbb{Q}}
\def\ZZ{\mathbb{Z}}
\def\NN{\mathbb{N}}
%\def\NN{\mathbb{Z}_{> 0}}
\def\Nzero{\mathbb{Z}_{\geq 0}}
\def\EE{\mathbb{E}}
\def\PP{\mathbb{P}}
\def\supp{\mathrm{supp}}
\def\diam{\mathrm{diam}}
\def\sp{\mathrm{span}}
\def\ker{\mathrm{ker}}
%\def\sp{\mathrm{span}} %messes up align enviroment
\newcommand{\rbr}[1]{\left( {#1} \right)}
\newcommand{\sbr}[1]{\left[ {#1} \right]}
\newcommand{\cbr}[1]{\left\{ {#1} \right\}}
\newcommand{\abr}[1]{\left\langle {#1} \right\rangle}
\newcommand{\abs}[1]{\left| {#1} \right|}
\newcommand{\norm}[1]{\left\|#1\right\|}
\def\one{\mathbf{1}}
\DeclareMathOperator*{\esssup}{ess\,sup}
\newcommand*\wc{{}\cdot{}}
%\newcommand*\wc{ \, \cdot \,}
%wc for wildcard
\renewcommand{\Re}{\operatorname{Re}}
\renewcommand{\Im}{\operatorname{Im}}
\newcommand{\sgn}{\textup{sgn\,}}

\setlength{\parindent}{0pt}
\setlength{\parskip}{11pt
}
\newtheorem{lemma}{Lemma}

\begin{document}

\textbf{HW 4 - MATH 231A - Fall 2023 - Chris Lane}
\begin{exercise}
  Let $X$ be a set.  Let $\mathcal{A}$ be a $\sigma$-algebra on $X$.  Let $\phi : \mathcal{A} \to [0, \infty]$.
  Suppose $ \phi$ satisfies
  \begin{enumerate}[(i)]
  \item
    $$
    \phi (\varnothing) = 0
    $$
  \item
    (Additivity) If $A$ and $B$ are disjoint sets in $\mathcal{A}$, then $\phi( A \cup B) = \phi(A) + \phi(B).$
  \item
    (Continuity From Below) If $A_1, A_2, ... \in \mathcal(A)$ and $A_1 \subseteq A_2 \subseteq ...$, then
    $\phi ( \bigcup \limits _ {i=1} ^ {\infty} A_i)  = \lim _{n \to \infty} \phi (A_n)$.
  \end{enumerate}
  Prove $\phi$ is a measure on $\mathcal{A}$. 
\end{exercise}
\begin{solution}
  We need to show two things.

  \begin{enumerate}[(i)]
  \item
    $$
    \phi(\varnothing) = 0
    $$

    This is given.
  \item
    Given a disjoint sequence of sets in $\mathcal{A}$, $A_i$, we need to show:
    $$
    \phi(\bigcup \limits_{i=1}^\infty A_i) = \sum \limits _ {i=1} ^ \infty \phi( A_i)
    $$

    We'll turn this into an example of continuity from below, starting with setting
    $ A'_i = \bigcup \limits _{i=1} ^ \infty A_i$.  This makes a chain $ A'_1 \subseteq A'_2 \subseteq A'_3, ... $, so that (iii) above applies.

    $$
    \phi(\bigcup \limits_{i=1}^\infty A'_i) = \lim \limits _{n \to \infty} \phi(A'_n) 
    $$

    We know that $\phi(A'_n) = \phi( \bigcup \limits _ {i = 1} ^ n A_i) = \sum \limits _{i=1}^n \phi(A_i) $ exactly, so the limits are also
    equal in the extended reals.

    That is,

    $$
    \phi(\bigcup \limits_{i=1}^\infty A_i) = \lim \limits _ {n \to \infty} \phi(A'_n) = \lim \limits_{n \to \infty} \sum \limits _ {i=1} ^n \phi( A_i) = \sum \limits_{i=1} ^\infty \phi(A_i)
    $$

    \qed
    
  \end{enumerate}

\end{solution}


\begin{exercise}
  Let $X$ be an infinite set.  For each $A \subseteq X$,
  define $\mu^* (A) = 0$ if $A$ is empty, $\mu^*(A) = 1$ if
  $A$ is a non-empty and finite set, and $\mu^*(A) = \infty$
  if $A$ is infinite.  Prove that $\mu^*$ is an
  outer measure and that $\mu^*$ is not continuous from below.
\end{exercise}
\begin{solution}
  There are two things to prove for an outer measure.
  \begin{enumerate}[(a)]
    \item
      $$
      \mu ^ *(\varnothing) = 0
      $$

    True by definition.
    \item
      $      \mu ^ * $ is countably monotone.  That is, if $A_1, A_2, ... \subseteq \mathscr{P}$ and
      $A \subseteq \bigcup \limits _ {n=1} ^ {\infty} A_n$, then
      $\mu ^ * (A) \leq \sum \limits _ {i=1} ^ {\infty} \mu ^* ( A_i)$.

      So given such a sequence of $A_i$, we have three cases of their finitude.

      Case one, at least one of the $A_i$, say $A_k$, is infinite.  In this case,
      the union is infinite and the $\mu ^ * (A_k) = \infty$ and
      \begin{align*}
        \mu ^ * (A) &= \infty  & \\
        \sum \limits _ {i=1} ^ {\infty} \mu ^* ( A_i) &= \mu ^* (A_k) + \sum _{i\neq k} ^ \infty \mu ^*  (A_i) & \\ 
        & = \infty  & 
      \end{align*}
      And so,
      $$
      \mu ^ * (A) \leq   \sum \limits _ {i=1} ^ {\infty} \mu ^* ( A_i)  
      $$

      as needed.

      Case two, all of the $A_i$ are finite, but there are an infinite
      number of them that have at least one (new/distinct) element.

      Then, $\mu ^*(A_i) = 1$ for an infinite number of $i$.  So
      $   \sum \limits _ {i=1} ^ {\infty} \mu ^* ( A_i) = \infty $. 

      In this case, the union could be finite or infinite, but either way 
      $$
      \mu^* (A) \leq \infty =  \sum \limits _ {i=1} ^ {\infty} \mu ^* ( A_i) 
      $$

      In the third case, all but finitely many of the $A_i$ are zero,
      and the others (at least one or it would be case 1) are finite.
      In this case, the union is finite, and $\mu^* (A) = 1 \leq \sum
      \limits _ {i=1} ^ {\infty} \mu ^* ( A_i)$.

      In all cases, the definition of an outer measure are fulfilled.
  \end{enumerate}

  However, $\mu ^*$ is not necessarily continuous from below.  To see
  this, let $X = \ZZ$, and let $A_i = \{ n \in \ZZ : |n| < i \}$.
  Then $A = \bigcup \limits _ { i=i } ^ \infty A_i$ has infinite
  cardinality, so $\mu ^* ( A) = \infty$ but each $A_1$ has finite
  cardinality, so $\mu ^*(A_i) = 1$, for all i, hence

  $$ \lim _ { i \to \infty } \mu ^* (A_i) = \lim _ { i \to \infty} 1 = 1
  $$
  And $ \mu^* ( A) = \infty > \lim \limits_{ i \to \infty } \mu ^* (A_i)$ violating
  continuity from below.  
  
\end{solution}

\begin{exercise}
  Prove:
  \begin{enumerate}[(a)]
  \item
    $\lambda ^* ( \QQ ) = 0 $
  \item
    For every $ \varepsilon > 0$, there exists a subset $F$ of $[0,1]$ such that every element of $F$ is an irrational number,
    and $ \lambda ^* ( F) > 1 - \varepsilon $.
    
  \end{enumerate}
\end{exercise}
\begin{solution}
  \begin{enumerate}[(a)]
  \item
    $$
    \lambda ^* ( \QQ) = \inf \{ \sum _{i=1} ^\infty \ell ( I _ i ) : I_1, I_2, I_3, ... \ \ \text{are intervals such that} \ \ \QQ \subseteq \bigcup \limits _ {i=1} ^ \infty I_i \} 
    $$
    
    Given an $\varepsilon > 0$, order $\QQ $ into a sequence $ \{ q_1, q_2, q_3, q_4, ... \}$. 
    
    Let $I_i = [q_i - \frac{\varepsilon}{2^{i+1}},  q_i + \frac{\varepsilon}{2^{i+1}} ] $ 
    
    \begin{align*}
      \ell [ I_i ] & = q_i + \frac{\varepsilon}{2^{i+1}} - q_i +  \frac{\varepsilon}{2^{i+1}} & \\  
      & = \frac{\varepsilon}{2^{i+1}} & \\  
    \end{align*}
    
    So

    $$
    \sum \limits _ { i=1} ^ \infty \ell [I_i] =  \sum \limits _ { i=1} ^ \infty \frac{\varepsilon}{2^i} = \varepsilon \sum \limits _ {i=1} ^ \infty \frac{1}{2^i}= \varepsilon 
    $$
    So the $ \inf \{ \text{all such} \} \leq \varepsilon$.  Since $\varepsilon$ was arbitrary, the $ \inf = 0$.
    \item
      The $F_\varepsilon$ is constructed as above, but starting with the full $[0,1]$ and subtracting out the intervals around the members of $\QQ$.

      So enumerate the rationals in $ \QQ \cap [0,1]$ in a sequence $ \{ q_1, q_2, q_3, ... \}$.

      Let $J_i =  [ q_i - \frac{\varepsilon}{2^{i+1}},  q_i + \frac{\varepsilon}{2^{i+1}}]$. 
      
      $$
      F_\varepsilon = [0, 1] - \bigcup \limits _ {i = 1} ^ \infty J_i
      $$

      Now, to make a statement about $\lambda ^* (F _ \varepsilon)$, we are looking at

      $$
      \lambda ^ * (F_\varepsilon) = \{ \inf \{ \sum \limits _ { i = 1 } ^ \infty \ell ( I_i ) : \text { intervals I with } \ F_\varepsilon \subseteq \bigcup \limits _ { i = 0 } ^ \infty \}
      $$

      To prove the $\inf$ is greater than a certain number,
      $1-\varepsilon$, we can produce for any covering of
      $F_\varepsilon$ a new covering such that the sum of the lengths
      is smaller than the given covering and greater than $1 -
      \varepsilon$.
          
      Let $ \{ I_1, I_2, I_,3 ... \}$ be intervals such that
      $ F _ \varepsilon  \subseteq \bigcup \limits _ {i=1} ^ \infty I_i$.

      We will create a new covering by removing all the $J_i$ from the
      $I_j$ and making new intervals $K_i,k$

      \begin{tabular}{ |c|c|c|c|c|}
        \hline
        \   & $I_1$   & $I_2$ & ... & $I_i$ \\
        \hline
        $J_1$ & $I_1 - J_1 (K_1, K_2)$ & $I_2 - J_1 (K_5, K_6)$ & ... & $I_i - J_1 (K_{m_1}, K_{m'_1})$ \\
        $J_2$ & $I_1 - J_2 (K_3, K_4)$ & $I_2 - J_1 (K_9, K_{10})$ & ... & $I_i - J_1 (K_{m_2}, K_{m'_2})$ \\
        ... &       &   &  &  \\
        $J_j$ & $I_1 - J_j (K_{i_1}, K_i'')$ & $I_2 - J_j (K_{i_1}, K_{i_2})$ & ... & $ I_i - J_j (K_{m_n}, K_{m'_n})$ \\
        ... &       &   &  &  \\
        \hline
      \end{tabular}
      Each $I_i - J_j$ yields 1 or 2 intervals $K_a$ and maybe $K_b$.
      This is seen by examining the cases of the beginning and
      endpoints of the two intervals $I_i$ and $J_j$.  If the are
      non-overlapping, $K_a$ is just $I_i$, and $ \ell(K_a) =
      \ell(I_i)$.  If $J_j \subset I_i$, then there's two new
      intervals, $K_a$ and $K_b$ and $\ell (I_i) < \ell (K_a) + \ell
      (K_b)$.  If $J_j$ overlaps $I_i$, then there's just one $K_a$ and
      $ \ell (I_i) < \ell (K_a)$.

      Now, in all cases the contribution of $I_i$ is greater than or
      equal to the contribution of its derived $K_a$s.  

      $$ \ell(I_i) >= \ell(K_a) + \ell(K_a')
      $$
      And the sums (in the extended reals) are also smaller:
      $$
      \sum \limits _ {i=1} ^ \infty \ell(I_i) >= \sum \limits _ { i = 1} ^ \infty \ell(K_i )
      $$
      
      
  \end{enumerate}
\end{solution}

\begin{exercise}
  Let $ \phi : \mathscr{P}(\RR) \to [0, \infty$ and suppose that
  $\phi$ is translation invarient (i.e. $ \phi ( A + t) = \phi (A) $
  for all $ A \subseteq \RR, t \in \RR$).  Let $V$ be a subset of $\RR$ such that
  for each $r \in \RR$ there is exactly one $v \in V$ such that $v - r \in \QQ$.
  Prove:
  \begin{enumerate}[(a)]
  \item
    Let ${r_1, r_2, ... }$ be an enumeration of $\QQ$.
    
  \item
    If $\phi$ is monotone, finitely additive, and $\phi(V + V - V) < \infty$, then
    $\phi(V) = 0$.  ($V + V - V = \{ v_1 + v_2 - v_3 : v_1, v_2, v_3 \in V\}$)
  \end{enumerate}
\end{exercise}
\begin{solution}
  These are basically the two contradictory halves of the Worse News
  Theorem shown in class (but with variations that make a new proof
  necessary).  
  \begin{enumerate}[(a)]
  \item

    First, lets enumerate the rationals $\{ q_1, q_2, q_3, ... \}$ 
    
    Let $A_i = q_i + V$ be translated copies of the set $V$.

    $ \RR = \bigcap \limits _ {i=1} ^ \infty A_i$.  To see this, let $x \in \RR$.  If $x \in V$, then
    $ x \in A_k$ where $q_k = 0$.  Otherwise, let $v_0$ be such that $v_0 - x \in \QQ$.
    That is $v_0 - x = q_m$.  Let $n$ be such that $q_n = -q_m$.  Then $x \in q_m + V$.  So, for all
    $x \in \RR$, there is an index $i$ so $x \in A_i$.  So $\RR \subseteq \bigcap \limits _ {i=1} ^ \infty A_i$.

    The other direction is obvious, all the $A_i \subseteq \RR$.  

    Let's examine $A_i \cap A_j$.  Let $x \in A_i$.  $x \in A_i$ means
    $x = v_0 + q_i$ for some unique $v_0 \in V$. By assumption, there
    is just one such $v_0$ that $v_0 - x \in \QQ$.  But then $A_j = V + q_j$, so if $x \in A_j$, then
    there would be $q_j$ also so that $v_1 - x \in \QQ$.  So $A_i \cup A_j = \varnothing$.

    Since $\phi$ is countably monotone and $\phi (\RR) > 0$, and $\RR = \bigcup \limits _ {n = 1} ^ \infty A_n$,
    $$
    \phi ( \RR ) \leq \sum \limits _ {n=1} ^ \infty \phi (A_i)
    $$

    But $\phi(A_i) = \phi(A_j), i, j \in \NN$ due to translation invariance, which is $\mu(V)$.

    Also, since $\phi(\RR) > 0$, so the above inequality works out to be:

    $$
    0 < \phi(\RR) \leq \sum \limits _ {n=1} ^ \infty \mu(v)
    $$

    Now, if $\mu(V) = 0$, that produces a contradiction:

    $ 0 < \phi( \RR) \leq 0$

    So $\mu(V) > 0$.

    \qed
    
  \item
    We are told $\phi(V + V - V) < \infty$.  Now, $V \subseteq V + V -
    V$.  To see this, let $v \in V$.  Pick some $w \in V$.  Then $v
    \in \{ v + w - w : v, w, w \in V \}$.  We are also told that
    $\phi(V + V - V) < \infty$.  Since $\phi$ is monotone, this
    implies $\phi(V) < \infty$.

    Now, $\phi$ is monotone and finitely additive, so we can look at
    TODO

\end{enumerate}
\end{solution}


\begin{comment}
\begin{exercise}
  problem
\end{exercise}
\begin{solution}
\begin{enumerate}[(a)]
\item
  first answer
\end{enumerate}
\end{solution}
\end{comment}


\end{document}

