\documentclass[11pt,oneside]{article}
\usepackage[hmargin=1in,vmargin=1in]{geometry}               % See geometry.pdf to learn the layout options. There are lots.
\geometry{letterpaper}                   % ... or a4paper or a5paper or ...
%\geometry{landscape}                % Activate for for rotated page geometry
\usepackage[parfill]{parskip}    % Activate to begin paragraphs with an empty line rather than an indent
\usepackage{graphicx}
\usepackage{amssymb}
\usepackage{mathrsfs}
\usepackage{epstopdf}
\usepackage{datetime}
\usepackage{url}
%\usepackage{verbatim}
\usepackage{comment}
\specialcomment{solution}{\textbf{Solution. }}{}
%\excludecomment{solution}    %uncomment to remove solutions.

%\usepackage{enumerate}

%Use the enumitem package instead of enumerate
\usepackage[shortlabels]{enumitem}
%\usepackage{enumitem}
%then it will support the same suntax as the enumerate package.
%The enumerate package does not provide any extra configurations other than the label.

%\setlist[enumerate]{topsep=0pt,itemsep=-1ex,partopsep=1ex,parsep=1ex}
\setlist[enumerate]{topsep=0pt,partopsep=0pt}

\DeclareGraphicsRule{.tif}{png}{.png}{`convert #1 `dirname #1`/`basename #1 .tif`.png}
\usepackage{amsmath,amsthm,amscd,amssymb}
\usepackage{latexsym}
\usepackage[colorlinks,citecolor=red,pagebackref,hypertexnames=false]{hyperref}
\numberwithin{equation}{section}

\theoremstyle{definition}
\newtheorem{exercise}{Exercise}
%\newtheorem{solution}{Solution}
\newtheorem*{defn}{Definition}


\def\calA{\mathcal{A}}
\def\calB{\mathcal{B}}
\def\calC{\mathcal{C}}
\def\calT{\mathcal{T}}
\def\OR{\overline{\mathbb{R}}}
\def\RR{\mathbb{R}}
\def\CC{\mathbb{C}}
\def\FF{\mathbb{F}}
\def\QQ{\mathbb{Q}}
\def\ZZ{\mathbb{Z}}
\def\NN{\mathbb{N}}
\def\Nzero{\mathbb{Z}_{\geq 0}}
\def\EE{\mathbb{E}}
\def\PP{\mathbb{P}}
\def\supp{\mathrm{supp}}
\def\diam{\mathrm{diam}}
\def\sp{\mathrm{span}}
\def\ker{\mathrm{ker}}
\def\fancyA{\mathscr{A}}
\def\fancyU{\mathscr{U}}
\def\fancyU{\mathscr{U}}
\def\fancyL{\mathcal{L}}
\def\fancyV{\mathscr{V}}
\def\fancyP{\mathscr{P}}
\def\fancyB{\mathscr{B}}
\def\limn{\lim \limits _n}

\newcommand*\diff{\mathop{}\!\mathrm{d}}

\newcommand{\rbr}[1]{\left( {#1} \right)}
\newcommand{\sbr}[1]{\left[ {#1} \right]}
\newcommand{\cbr}[1]{\left\{ {#1} \right\}}
\newcommand{\abr}[1]{\left\langle {#1} \right\rangle}
\newcommand{\abs}[1]{\left| {#1} \right|}
\newcommand{\norm}[1]{\left\|#1\right\|}
\def\one{\mathbf{1}}
\DeclareMathOperator*{\esssup}{ess\,sup}
\newcommand*\wc{{}\cdot{}}
%\newcommand*\wc{ \, \cdot \,}
%wc for wildcard
\renewcommand{\Re}{\operatorname{Re}}
\renewcommand{\Im}{\operatorname{Im}}
\newcommand{\sgn}{\textup{sgn\,}}

\setlength{\parindent}{0pt}
\setlength{\parskip}{11pt
}
\newtheorem{lemma}{Lemma}

\begin{document}

\textbf{HW 11 - MATH 231A - Fall 2023 - Chris Lane}

Date: \hhmmsstime{} \ \today \ \ Git hash: 
\input{/Users/chlane/src/jayalane/2023H2HW/.git/refs/heads/main} 

\begin{exercise}
  (Embedding of $L^p$ Spaces).  Let $(X_i, \fancyA, \mu)$ be a measure space.
  Let $1 \leq p < q < \infty$.  Note $p, q$ are not necessarily
  conjugate exponents here.
  \begin{enumerate}[(a)]
  \item
    If $\mu(X) < \infty$, then $||f||_p \leq \mu(X)^{-p^{-1}-q^{-1}} || f|| _q$ and $L^q \subseteq L^p$. 
  \end{enumerate}
\end{exercise}

\begin{solution}
  \begin{enumerate}
  \item
    The hint in the grading was to use H\"{o}lder's inequality, which needs two
    numbers $p'$, $q'$, that obey
    \[
      1 = \frac{1}{p'} + \frac{1}{q'}
  \]

  \begin{align*}
    p' & = \frac{q}{p} \\
    \frac{1}{p'} & = \frac{p}{q} \\
    \frac{1}{p'} + \frac{1}{q'} & = 1 \\
    \frac{p}{q}  + \frac{1}{q'} & = 1 \\
    \frac{1}{q'}  & = 1 - \frac{p}{q}  = \frac{q - p}{q} \\
    {q'}  & = \frac{q}{q - p} 
  \end{align*}

  Using these $p'$, $q'$, and $\| |f|^p \| \cdot 1 $ in H\"{o} inequality:
  \[
    \| |f|^p \cdot 1 \| \leq \| |f|^p || _ {p'} \| 1 \| _ {q'} 
  \]
  So subsituting in $p'$, $q'$:
  \[
    \int |f|^p \leq \| |f|^p \| _{\frac{q}{p}} \| 1 \| _{\frac{q}{q-p}} 
  \]
  Continuing: 
  \[
    \int |f|^p \leq \left( \int |f|^{p\frac{q}{p}} \right)^{\frac{p}{q}} \left( \int 1^{\frac{q}{q-p}} \right)^{\frac{q-p}{q}}
  \]
  And:
  \[
    \int |f|^p \leq \left( \int |f|^{q} \right)^{\frac{p}{q}} \left( \int 1 \right)^{\frac{q-p}{q}}
  \]
  Taking $p$th roots:
  \[
    \left( \int |f|^p \right) ^{\frac{1}{p}} \leq \left( \int |f|^{q} \right)^{\frac{1}{q}} \left( \int 1 \right)^{\frac{q-p}{qp}}
  \]
  Breaking it back out to norms:
  \[
    \| f^p \| \leq \| f \|^{q} \left( \mu(X) \right)^{\frac{q}{qp} - \frac{p}{qp}} 
  \]  
\end{enumerate}
\end{solution}

\begin{exercise}
  Let $(X, \fancyA, \mu) $ be a measure space.  If $1 \leq p < q < r = \infty$, then
  \[
  L^q \subseteq L^p + L^\infty,
  \]

  in other words, each $f \in L^q$ is the sum of a function in $L^p$ and a function in $L^\infty$. 
\end{exercise}
\begin{solution}
  Let $E = \{ x \in X : | f(x) | > 1 \}$.  Let $f_1 = \one_E f$ and $f_2 = \one_{E^c} f$.

  Then we claim $f_1 \in L^p$ and $f_2 \in L^\infty$.  We have $|f_1(x)| > 1$, for all $x \in E$.  Also, 
  \[
  p < q
  \]

  TODO $r = \infty$.  
\end{solution}

\begin{exercise}
  Let $(X, \fancyA, \mu )$ be a measure space.  Let $S$ denote the set of measurable simple functions $s : X \to \RR$.  Let
  $S_0$ denote the set of measurable simple functions $s : X \to \RR$ that vanish outside a set of
  finite measure (i.e. that satisfy $\mu (\{ x: s(x) \neq 0\}) < \infty$).  
\begin{enumerate}[(a)]
\item
  If $1 \leq p \leq \infty$, $S \cap L^p = S_0$.
\item
  If $f : X \to \RR$ is measurable and $(X, \fancyA, \mu)$ is
  $\sigma-$finite, then there is a sequence $(s_n)$ in $S_0$ such that
  $s_n \to f$ on $X$ and $|s_n|\uparrow |f|$ on $X$.  Hint: Simple Approximation Theorem.
\item
  If $f \in L^p$, and $1 \leq p < \infty$, then there is a sequence $(s_n)$ in $S_0$ such that
  $s_n \to f$ on $X$ and $ |s_n| \uparrow |f|$ on $X$ and $|| s_n - f||_p \to 0$. Hint:
  first show $E = \{ x \in X: | f(x)| ^p \neq 0 \}$ is $\sigma-$finite.  
\end{enumerate}
  
\end{exercise}

\begin{exercise}
  Let $X$ be a nonempty set.  Let $\mu$ be the counting measure on $(X, \fancyP(X))$. 
  \begin{enumerate}[(a)]
  \item
    If $f: X \to [0, \infty]$, then
\[
\int f \diff{\mu} = \sup \left\{ \sum _ {e \in F} f(x) : F \subseteq X, F \textrm{ is finite} \right\}
\]
  \end{enumerate}
\end{exercise}
\begin{solution}
\begin{enumerate}[(a)]
\item
  TODO - I'm just not sure how to go from arbitrary simple functions to sums over finite sets; maybe some splitting it  into finite and infinite measure cases. 

\end{enumerate}
\end{solution}

\begin{exercise}
  Uniqueness of the constructed probability measure with a given distribution.  
\end{exercise}
\begin{solution}
TODO
\end{solution}


\begin{comment}
\begin{exercise}
  problem
\end{exercise}
\begin{solution}
\begin{enumerate}[(a)]
\item
  first answer
\end{enumerate}
\end{solution}
\end{comment}


\end{document}
