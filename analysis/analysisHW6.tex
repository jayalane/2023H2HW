\documentclass[11pt,oneside]{article}
\usepackage[hmargin=1in,vmargin=1in]{geometry}               % See geometry.pdf to learn the layout options. There are lots.
\geometry{letterpaper}                   % ... or a4paper or a5paper or ...
%\geometry{landscape}                % Activate for for rotated page geometry
\usepackage[parfill]{parskip}    % Activate to begin paragraphs with an empty line rather than an indent
\usepackage{graphicx}
\usepackage{amssymb}
\usepackage{mathrsfs}
\usepackage{epstopdf}
\usepackage{datetime}
\usepackage{url}
%\usepackage{verbatim}
\usepackage{comment}
\specialcomment{solution}{\textbf{Solution. }}{}
%\excludecomment{solution}    %uncomment to remove solutions.

%\usepackage{enumerate}

%Use the enumitem package instead of enumerate
\usepackage[shortlabels]{enumitem}
%\usepackage{enumitem}
%then it will support the same suntax as the enumerate package.
%The enumerate package does not provide any extra configurations other than the label.

%\setlist[enumerate]{topsep=0pt,itemsep=-1ex,partopsep=1ex,parsep=1ex}
\setlist[enumerate]{topsep=0pt,partopsep=0pt}

\DeclareGraphicsRule{.tif}{png}{.png}{`convert #1 `dirname #1`/`basename #1 .tif`.png}
\usepackage{amsmath,amsthm,amscd,amssymb}
\usepackage{latexsym}
\usepackage[colorlinks,citecolor=red,pagebackref,hypertexnames=false]{hyperref}
\numberwithin{equation}{section}

\theoremstyle{definition}
\newtheorem{exercise}{Exercise}
%\newtheorem{solution}{Solution}
\newtheorem*{defn}{Definition}


\def\calA{\mathcal{A}}
\def\calB{\mathcal{B}}
\def\calC{\mathcal{C}}
\def\calT{\mathcal{T}}
\def\OR{\overline{\mathbb{R}}}
\def\RR{\mathbb{R}}
\def\CC{\mathbb{C}}
\def\FF{\mathbb{F}}
\def\QQ{\mathbb{Q}}
\def\ZZ{\mathbb{Z}}
\def\NN{\mathbb{N}}
%\def\NN{\mathbb{Z}_{> 0}}
\def\Nzero{\mathbb{Z}_{\geq 0}}
\def\EE{\mathbb{E}}
\def\PP{\mathbb{P}}
\def\supp{\mathrm{supp}}
\def\diam{\mathrm{diam}}
\def\sp{\mathrm{span}}
\def\ker{\mathrm{ker}}
\def\fancyU{\mathscr{U}}
\def\fancyL{\mathscr{L}}
\def\fancyV{\mathscr{V}}
\def\fancyP{\mathscr{P}}
%\def\sp{\mathrm{span}} %messes up align enviroment
\newcommand{\rbr}[1]{\left( {#1} \right)}
\newcommand{\sbr}[1]{\left[ {#1} \right]}
\newcommand{\cbr}[1]{\left\{ {#1} \right\}}
\newcommand{\abr}[1]{\left\langle {#1} \right\rangle}
\newcommand{\abs}[1]{\left| {#1} \right|}
\newcommand{\norm}[1]{\left\|#1\right\|}
\def\one{\mathbf{1}}
\DeclareMathOperator*{\esssup}{ess\,sup}
\newcommand*\wc{{}\cdot{}}
%\newcommand*\wc{ \, \cdot \,}
%wc for wildcard
\renewcommand{\Re}{\operatorname{Re}}
\renewcommand{\Im}{\operatorname{Im}}
\newcommand{\sgn}{\textup{sgn\,}}

\setlength{\parindent}{0pt}
\setlength{\parskip}{11pt
}
\newtheorem{lemma}{Lemma}

\begin{document}

\textbf{HW 5 - MATH 231A - Fall 2023 - Chris Lane}
Date: \hhmmsstime{} \ \today \ \ Git hash: 
% $ \input{/home/jayalane/hw/.git/refs/heads/main} $

\begin{exercise}
  Prove or disprove: If $ |f| $ is measurable, then $f$ is measurable.
\end{exercise}
\begin{solution}
  Counterexample:

  Let $V$ be a Vitali set over $[0,1]$.  Then let $f$ be defined:

  $$
  f(x) = \begin{cases}
    1, \textrm{ if } x \in V,
    -1 \textrm{ if } x \notin V
  \end{cases}
  $$

  Then $|f(x)| = 1$, identically.  But $f(x)$ is not measurable, as $
  { x : f(x) = 1 } = V$ the pre-image of ${1}$ is the Vitali set,
  which is not measurable (Borel or Lebesgue).
\end{solution}

\begin{exercise}
  If $ f: \RR \to \RR$ is increasing or decreasing, then $f$ is Borel
  measurable (and hence Lebesgue measurable.

  Assume, without loss of generality (as $ \{ f \geq a \} $ is just as good as $ \{ f \leq a \}$,
  that $f$ is increasing.  That is:

  $$
  x_0 < x_1 \textmf { implies } f(x_0) < f(x_1)
  $$

  Pick an $a \in \RR$.

  $$
  J = \{ x \in \RR : f(x) \geq a \}
  $$

  is the set we need to show is in $\fancyB(\RR)$.  There are two cases, one
  where $ J = \varnothing \in fancyB(\RR)$.

  Second case, there is at least one $x_0 \in J$.
  Let $x_1 = \inf \limits _ { x \in \RR } \{ f(x) \geq a \}$.
  
  Then $[x_1, \infty\] \subseteq J$.  To show this, let $x \in J$, $x$
  
  
\end{exercise}
\begin{solution}
\end{solution}


\begin{comment}
\begin{exercise}
  problem
\end{exercise}
\begin{solution}
\begin{enumerate}[(a)]
\item
  first answer
\end{enumerate}
\end{solution}
\end{comment}


\end{document}

