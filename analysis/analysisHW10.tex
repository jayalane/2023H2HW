\documentclass[11pt,oneside]{article}
\usepackage[hmargin=1in,vmargin=1in]{geometry}               % See geometry.pdf to learn the layout options. There are lots.
\geometry{letterpaper}                   % ... or a4paper or a5paper or ...
%\geometry{landscape}                % Activate for for rotated page geometry
\usepackage[parfill]{parskip}    % Activate to begin paragraphs with an empty line rather than an indent
\usepackage{graphicx}
\usepackage{amssymb}
\usepackage{mathrsfs}
\usepackage{epstopdf}
\usepackage{datetime}
\usepackage{url}
%\usepackage{verbatim}
\usepackage{comment}
\specialcomment{solution}{\textbf{Solution. }}{}
%\excludecomment{solution}    %uncomment to remove solutions.

%\usepackage{enumerate}

%Use the enumitem package instead of enumerate
\usepackage[shortlabels]{enumitem}
%\usepackage{enumitem}
%then it will support the same suntax as the enumerate package.
%The enumerate package does not provide any extra configurations other than the label.

%\setlist[enumerate]{topsep=0pt,itemsep=-1ex,partopsep=1ex,parsep=1ex}
\setlist[enumerate]{topsep=0pt,partopsep=0pt}

\DeclareGraphicsRule{.tif}{png}{.png}{`convert #1 `dirname #1`/`basename #1 .tif`.png}
\usepackage{amsmath,amsthm,amscd,amssymb}
\usepackage{latexsym}
\usepackage[colorlinks,citecolor=red,pagebackref,hypertexnames=false]{hyperref}
\numberwithin{equation}{section}

\theoremstyle{definition}
\newtheorem{exercise}{Exercise}
%\newtheorem{solution}{Solution}
\newtheorem*{defn}{Definition}


\def\calA{\mathcal{A}}
\def\calB{\mathcal{B}}
\def\calC{\mathcal{C}}
\def\calT{\mathcal{T}}
\def\OR{\overline{\mathbb{R}}}
\def\RR{\mathbb{R}}
\def\CC{\mathbb{C}}
\def\FF{\mathbb{F}}
\def\QQ{\mathbb{Q}}
\def\ZZ{\mathbb{Z}}
\def\NN{\mathbb{N}}
\def\Nzero{\mathbb{Z}_{\geq 0}}
\def\EE{\mathbb{E}}
\def\PP{\mathbb{P}}
\def\supp{\mathrm{supp}}
\def\diam{\mathrm{diam}}
\def\sp{\mathrm{span}}
\def\ker{\mathrm{ker}}
\def\fancyA{\mathscr{A}}
\def\fancyU{\mathscr{U}}
\def\fancyU{\mathscr{U}}
\def\fancyL{\mathcal{L}}
\def\fancyV{\mathscr{V}}
\def\fancyP{\mathscr{P}}
\def\fancyB{\mathscr{B}}
\def\limn{\lim \limits _n}

\newcommand*\diff{\mathop{}\!\mathrm{d}}

\newcommand{\rbr}[1]{\left( {#1} \right)}
\newcommand{\sbr}[1]{\left[ {#1} \right]}
\newcommand{\cbr}[1]{\left\{ {#1} \right\}}
\newcommand{\abr}[1]{\left\langle {#1} \right\rangle}
\newcommand{\abs}[1]{\left| {#1} \right|}
\newcommand{\norm}[1]{\left\|#1\right\|}
\def\one{\mathbf{1}}
\DeclareMathOperator*{\esssup}{ess\,sup}
\newcommand*\wc{{}\cdot{}}
%\newcommand*\wc{ \, \cdot \,}
%wc for wildcard
\renewcommand{\Re}{\operatorname{Re}}
\renewcommand{\Im}{\operatorname{Im}}
\newcommand{\sgn}{\textup{sgn\,}}

\setlength{\parindent}{0pt}
\setlength{\parskip}{11pt
}
\newtheorem{lemma}{Lemma}

\begin{document}

\textbf{HW 10 - MATH 231A - Fall 2023 - Chris Lane}

Date: \hhmmsstime{} \ \today \ \ Git hash: 
\input{/Users/chlane/src/jayalane/2023H2HW/.git/refs/heads/main} 

\begin{exercise}
  Let $(X, \fancyA, \mu)$ and $(Y, \fancyB, \nu)$ be $\sigma-$finite
  measure spaces.  Prove that
  \[
  ( X \times Y, M((\mu \times \nu)^*), \mu \times \nu)
  \]
  is $\sigma-$finite.
\end{exercise} 
\begin{solution}
  X is $\sigma-$finite, that is:
  \[
  X = \bigcup \limits _ {i=0}^{\infty} X_i, \quad \textrm{ where } \mu(X_i) < \infty
  \]

  And Y is also:
  \[
  Y = \bigcup \limits _{j=0}^{\infty} Y_i, \quad \textrm{ where } \mu(Y_j) < \infty
  \]

  Then we claim $(X \times Y) = \bigcup _ {i=0} ^ \infty \bigcup _{j=0}^\infty X_i \times Y_jj$.

  To see this, let $(x, y) \in X \times Y$.

  Then $x \in X_{i_0}$ for some $i_0 $.

  And  $y \in Y_{j_0}$ for some $j_0 $.

  Then $(x, y) \in X_{i_0} \times Y_{j_0}$.

  Since $X_i \times Y_j$ are measurable rectangles, $(\mu \times
  \nu)(X_i \times Y_j) = \mu(X_i) \cdot \nu(Y_j) < \infty$ so the product space
  is $\sigma-$finite.
  
\end{solution}

\begin{exercise}
  We use $\diff x$ and $\diff y$ to abbreviate $\diff \lambda(x)$
  and $\diff \lambda(y)$, where $\lambda$ is Lebesgue measure on $\fancyL(\RR)$.
  Lebesgue measure on $\RR^2$ is $\lambda^2 = \lambda \times \lambda$.  Let $E = [0,1 ] \times [0,1]$.
  Determine whether the following integrals exist and whether any of them equals another.
  \begin{enumerate}[(a)]
\item
  \[
  f(x,y) = \frac{x^2 - y^2}{(x^2+y^2)^2}
  \]
\item
  \[
  f(x,y) = \frac{1}{(1-xy)^a}, \ \textrm{ where } (a > 0)
  \]
 \item
   \[
   f(x,y) = \begin{cases}
     (x - \frac{1}{2})^{-3}, \  \textrm{ if } 0 < y < \left| x - \frac{1}{2} \right| \\
     0, \  \textrm{ otherwise }
   \end{cases}
   \]
      
\end{enumerate}
  
\end{exercise}
\begin{solution}
\begin{enumerate}[(a)]
\item
  This one is in the Wikipedia page on Fubini's theorem and seems a
  rather common example of what happens when the conditions of
  Fubnini's theorem are not met.  As I didn't consider the possibility
  that the exercises would be examples where Fubini does not apply
  until I read about this example, I feel bad just typing it in.
  Also, I'm tired, and just spent pages of paper trying to integral
  $\frac{x^2 - y^2}{x^2+y^2}$, no final square in the denominator,
  using trig substitutions. I may ask to revise it tomorrow.
  
\item
  First we will establish a condition of absolute integrability.  Everything is in $E$ so
  $x$ and $y$ are in $[0,1]$.  

  \begin{align*}
    y & \leq 1 \\
    yx & \leq x \\
    1- yx & \geq 1 - x \\
    \frac{1}{1 - yx} & \leq \frac{1}{1-x}\\
    0 \leq  \frac{1}{(1 - yx)^a} & \leq \frac{1}{(1-x)^a} \textrm{ raising to a power } >0 \textrm{ does not reorder } [0,1] \\
    0 \leq \int  \frac{1}{(1 - yx)^a} \diff \lambda^2 & \leq \int \frac{1}{(1-x)^a} \diff \lambda^2 \\
  \end{align*}

  I'm assuming some sort of projective property here where if $E \in \RR^2 = A \times B$ where
  $\lambda(B) = 1$, then

  \[
  \int _ {E} f(x) \diff \lambda^2 = \int _A f(x) \diff \lambda
  \]

  I believe it would be shown by reducing to measurable rectangles where
  one dimension was from the simple functions under $f$ and the other would just
  be $B$, where multiplying by $\lambda(B)$ wouldn't change anything.  

  If that holds, then the $\diff \lambda$ integral converges for $0 < a < 1$ and
  diverges otherwise.

  So at the very least the conditions for Fubini are met for $a<1$, and the three
  integrals exist and are equal.  
\item
  I find this a better example of how weird it can do when the
  conditions for Fubini are not met.

  I won't do the exact calculations on $\int \left| f \right|$ but it's
  basically $\int \frac{1}{x^3} \cdot x $ (the $(x - \frac{1}{2})$ is
  taken to the negative three power, and then as one approaches 1/2,
  its down weighted by the interval where it is non zero due to $y$
  shrinking proportionately with $y$; $\int \frac{1}{x^2}$ however diverges
  around $0$.

  But we don't need to prove that; by showing $\int _ E f \diff x \diff y \neq \int _ E f \diff y \diff x$
  Fubini assures us the $\int _E \left| f \right| $ cannot be finite.  


  \begin{align*}
  \int \limits _0^1  \int \limits _0 ^ 1 f \diff y \diff x & =   \int \limits _0^1  \int \limits _0 ^ {\abs{x - \frac{1}{2}}} \frac{1}{(x - \frac{1}{2})^3} \diff y \diff x \\
  & =   \int \limits _0^1  \frac{1}{(x - \frac{1}{2})^3} \int \limits _0 ^ {\abs{x - \frac{1}{2}}} 1 \diff y \diff x \\
  & =   \int \limits _0^1  \frac{1}{(x - \frac{1}{2})^3} y \rvert _ 0 ^ {\abs{x - \frac{1}{2}}} \diff x \\
  & =   \int \limits _0^{\frac{1}{2}} \frac{1}{(x - \frac{1}{2})^3} (\frac{1}{2} - x) \diff x  + \\
    & \quad \quad  \int \limits _{\frac{1}{2}}^{1} \frac{1}{(x -\frac{1}{2})^3} (x - \frac{1}{2} ) \diff x  \\
    & = - \infty + \infty 
  \end{align*}

  So in that order, the integral fails to exist.  

  In the other order, take some $\varepsilon > 0$ (to skip the line where $y = 0$; a line has measure zero in $\RR$ so the integrals won't change).  

  \begin{align*}
    \int \limits _\varepsilon ^1  \int \limits _0 ^ 1 f \diff x \diff y & = \int \limits _\varepsilon^{\frac{1}{2}}  \left( \int \limits _0 ^ {\frac{1}{2}-y} \frac{1}{(x - \frac{1}{2})^3} \diff x  + \int \limits _{y + \frac{1}{2}}^1 \frac{1}{(x - \frac{1}{2})^3} \diff x \right) \diff y \\ 
    & = \int \limits _\varepsilon^{\frac{1}{2}} 0  \diff y,  \ \ \textrm{ as the two } \diff x \textrm{ integrals are equal except the sign} \\
    & = 0 \\
  \end{align*}

  This would be clearer if I knew how to draw diagrams in LaTeX - the
  odd definite integrals are the expression of the inequality defining
  where $f$ is non-zero.
  
  That's true for all $\varepsilon$, therefor it is true for the
  entire integral.  So the order changes the integral and the $f$ is
  not Lebesgue integrable.
  
\end{enumerate}
\end{solution}


\begin{comment}
\begin{exercise}
  problem
\end{exercise}
\begin{solution}
\begin{enumerate}[(a)]
\item
  first answer
\end{enumerate}
\end{solution}
\end{comment}


\end{document}
