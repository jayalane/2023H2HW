\documentclass[11pt,oneside]{article}
\usepackage[hmargin=1in,vmargin=1in]{geometry}               % See geometry.pdf to learn the layout options. There are lots.
\geometry{letterpaper}                   % ... or a4paper or a5paper or ...
%\geometry{landscape}                % Activate for for rotated page geometry
%\usepackage[parfill]{parskip}    % Activate to begin paragraphs with an empty line rather than an indent
\usepackage{graphicx}
\usepackage{amssymb}
\usepackage{epstopdf}
\usepackage{url}
%\usepackage{verbatim}
\usepackage{comment}
\specialcomment{solution}{\textbf{Solution. }}{}
%\excludecomment{solution}    %uncomment to remove solutions.

%\usepackage{enumerate}

%Use the enumitem package instead of enumerate
\usepackage[shortlabels]{enumitem}
%\usepackage{enumitem}
%then it will support the same suntax as the enumerate package.
%The enumerate package does not provide any extra configurations other than the label.

%\setlist[enumerate]{topsep=0pt,itemsep=-1ex,partopsep=1ex,parsep=1ex}
\setlist[enumerate]{topsep=0pt,partopsep=0pt}

\DeclareGraphicsRule{.tif}{png}{.png}{`convert #1 `dirname #1`/`basename #1 .tif`.png}
\usepackage{amsmath,amsthm,amscd,amssymb}
\usepackage{latexsym}
\usepackage[colorlinks,citecolor=red,pagebackref,hypertexnames=false]{hyperref}
\numberwithin{equation}{section}

\theoremstyle{definition}
\newtheorem{exercise}{Exercise}
%\newtheorem{solution}{Solution}
\newtheorem*{defn}{Definition}


\def\calA{\mathcal{A}}
\def\calB{\mathcal{B}}
\def\calC{\mathcal{C}}
\def\calT{\mathcal{T}}
\def\OR{\overline{\mathbb{R}}}
\def\RR{\mathbb{R}}
\def\CC{\mathbb{C}}
\def\FF{\mathbb{F}}
\def\QQ{\mathbb{Q}}
\def\ZZ{\mathbb{Z}}
\def\NN{\mathbb{N}}
%\def\NN{\mathbb{Z}_{> 0}}
\def\Nzero{\mathbb{Z}_{\geq 0}}
\def\EE{\mathbb{E}}
\def\PP{\mathbb{P}}
\def\supp{\mathrm{supp}}
\def\diam{\mathrm{diam}}
\def\sp{\mathrm{span}}
\def\ker{\mathrm{ker}}
%\def\sp{\mathrm{span}} %messes up align enviroment
\newcommand{\rbr}[1]{\left( {#1} \right)}
\newcommand{\sbr}[1]{\left[ {#1} \right]}
\newcommand{\cbr}[1]{\left\{ {#1} \right\}}
\newcommand{\abr}[1]{\left\langle {#1} \right\rangle}
\newcommand{\abs}[1]{\left| {#1} \right|}
\newcommand{\norm}[1]{\left\|#1\right\|}
\def\one{\mathbf{1}}
\DeclareMathOperator*{\esssup}{ess\,sup}
\newcommand*\wc{{}\cdot{}}
%\newcommand*\wc{ \, \cdot \,}
%wc for wildcard
\renewcommand{\Re}{\operatorname{Re}}
\renewcommand{\Im}{\operatorname{Im}}
\newcommand{\sgn}{\textup{sgn\,}}


\setlength{\parindent}{0pt}
\setlength{\parskip}{11pt
}
\newtheorem{lemma}{Lemma}

\begin{document}

\textbf{HW 1 - MATH 231A - Fall 2023 - Chris Lane}

\begin{exercise}
Let $ f : [a, b] \to \RR$. Prove: If $f$ is Riemann integrable on $[a, b]$, then $f$ is bounded.

\end{exercise}
\begin{solution}
  Proof by contraction.  Suppose $f$ is unbounded by in the positive direction;
  the unbounded below case is similar, replacing $\inf$ for $\sup$, $>$ for $<$,
  and $ - \infty $ for $ \infty$.  
  \begin{lemma}
    Given a (finite) partition $ P $ of $ [ a, b] $,  if
    $ \sup \limits_{[p_{i-1}, p_i]}f < \infty $ on all segments of a partion
    $P$ of $[a, b]$, then $\sup \limits_{[a,b]} f < \infty $
\end{lemma}
  \begin{proof}
    Otherwise, $ \max \limits_{i} \{ \sup \limits_{[x_{i-1}, x_ i]} f \} $
    would be the supremum.

    To be more explicit, for the partition $ P = \{ a= p_0, p_1, ... p_n = b \}$,
    $ \sup\limits_{[a, b]} f = \max\limits_{i} \{ \sup\limits_{[p _ {i-1}, p _ i]} f \}$
    That is, the least upper bound of a finite union is the max of the least upper bounds.

    If $ \max \limits_i \{ \sup \limits_{[p_{i-1}, p_i]} f \} $ is less than $ \sup \limits_{[a,b]} f $
    then the $\sup$ isn't a least upper bound. 
    If $ \max \limits_i \{ \sup \limits_{[p_{i-1}, p_i]} f \} $ is greater
    than $ \sup \limits_{[a,b]}$ then it
    isn't an upper bound. So they must be equal.  
  \end{proof}

  So, if $f$ were unbounded, then every partition $P$ of $[a,b]$ would
  contain one segment where $\sup \limits_{[x _ {i-1}, x_ i]} = \infty$,
  so $ U ( f, P) = \infty $, so
  the $ \overline \int _ { a} ^ {b} $ does not exist, and
  therefore the Riemann integral of $f$ does not exist on $ [a , b]$.   
\end{solution}

\begin{exercise}
  \begin{enumerate}[(a),start=3]
    \item
    If $P$ and $Q$ are any partition of $[a,b]$, then
    $$
    L(f, P) \leq U(f, Q)
    $$
  \item
    $$
    \underline \int _ {a} ^ {b} f \leq  \overline \int _ {a} ^ {b} f
    $$    
  \end{enumerate}
\end{exercise}
\begin{solution}
  \begin{enumerate}[(a),start=3]
  \item
    If $P$ and $Q$ are partitions of $[a,b]$, let $R = P \bigcup Q$.

    By property (a), $L(f, R) \leq U(f, R)$.

    $ P \subseteq R$, so by property (b),
    $$
    L(f, P) \leq L(f, R)
    $$
    and 
    $$
    U(f, R) \leq U(f, P)
    $$

    Likewise, $ Q \subseteq R$, so by property (b),
    $$
    L(f, Q) \leq L(f, R)
    $$
    and
    $$ 
    U(f, R) \leq U(f, Q)
    $$
    Combining these,
    $$
    L(f,P) \leq L(f, R) \leq U(f, R) \leq U(f, Q)
    $$
    \qed
  \item

    We want to show that 
    $$ \underline \int _ a ^ b f \leq \overline \int _ a ^ b  f $$

    For all $P, Q$ partitions,
    $$
    L(f,P) \leq U(f, Q)
    $$
    by (c).  So
    $$
    \sup \limits_ {\text{P in all partitions}} \{ L(f, P) \} \leq \inf \limits_{\text{Q in all partitions }} \{ U(f, Q) \}
    $$
    Suppose, in contradiction, that $ \sup L > \inf U$.
    Then, there is an $ \epsilon $ such that $\sup L > \inf U + \epsilon$.
    So there is a partition $P$ such that
    $$
    L(f, P) \geq \inf \limits_{\text{Q in all partitions }} \{ U(f, Q) \} + \epsilon
    $$
    (otherwise the upper bound is not a least upper bound) or to regain strict inequality, 
    $$
    L(f, P) > \inf \limits_{\text{Q in all partitions }} \{ U(f, Q) \} + \frac{\epsilon}{2} 
    $$
    So there is a partition $Q$ such that
    $$
    L(f, P) \geq U(f, Q) + \frac{\epsilon}{2}
    $$
    (otherwise the lower bound is not a greatest lower bound) Or, restoring strict inequality, 
    $$
    L(f, P) >  U(f, Q) + \frac{\epsilon}{4}
    $$
    But this contradicts (a), so therefore:
    $$
    \underline \int _ a ^ b f \leq \overline \int _ a ^ b  f
    $$
    
\end{enumerate}
\end{solution}
\begin{exercise}
  \begin{enumerate}[(a)]
  \item
    If $P$ is any partition of $[a,b]$, then $L(f, P) + L(g, P) \leq L(f+g) $ and
    $U(f+g, P) \leq U(f,P) + U(g, P)$.
  \item
    $ \underline \int _ a ^ b f + \underline \int _a ^b g \leq \underline \int _ a ^ b (f + g) $ and
    $ \overline \int _ a ^ b (f + g) \leq \overline \int _ a ^ b f + \overline \int _a ^b g $.
  \end{enumerate}
    
\end{exercise}
\begin{solution}
\begin{enumerate}[(a)]
\item

  \begin{lemma}
    Given a $f$, $g$ on closed interval $[x_0, x_1]$,
    $$
    \sup \limits_{[x_0, x_1]} f + \sup \limits_{[x_0, x_1]} g \leq   \sup \limits_{[x_0, x_1]}  ( f + g)
    $$
  \end{lemma}
  \begin{proof}
    For all $ x \in [x_0, x_1]$, $f(x) \leq \sup \limits_{[x_0, x_1]} f$ and
    $g(x) \leq \sup \limits_{[x_0, x_1]} g$.
    So
    $$
    f(x) + g(x) \leq  \sup \limits_{[x_0, x_1]} f + \sup \limits_{[x_0, x_1]} g
    $$
    and therefore the least upper bound is also less:
    $$
    \sup \limits_{[x_0, x_1]} ( f + g ) \leq \sup \limits_{[x_0, x_1]} f + \sup \limits_{[x_0, x_1]} g
    $$
  \end{proof}
  
  So, given a particular partition $P$,
  $$
  \sum _{ i = 1} ^ { n} \sup \limits_{[x_{i-1}, x_i]} (f + g) \leq \sum _ { i = 1} ^ {n}  \sup \limits_{[x_{i-1}, x_i]} f +  \sum _{ i = 1} ^ {n}  \sup \limits_{[x_{i-1}, x_i]}
  $$
  Or
  $$
  L(f, P) + L(g, P) \leq L(f+g, P)
  $$
  
  Likewise for the upper bounds with $U$ substituted for $U$, and $>$ for $<$.
  
\item
  Given a partition $P$,
  $$
  L(f, P) + L(g, P) \leq L( f + g, P)
  $$
  from (a), so
  $$
  \sup \limits_{P} \{ L(f, P) \} + \sup \limits_ { P} \{ L(g, P) \} \leq \sup \limits_{P} \{ L(f + g, P) \}
  $$
  (Otherwise, if it were larer, then there would be an $ \epsilon $ and a partition Q such that
  $\sup \limits_{P} L(F, p) + L(g, Q) > \sup \limits_{P} L(f + g, P)$
  To prove it for each, each $\sup$ needs one more application of the specific partition and an $\epsilon$ to
  go from least upper bound to a contradiction.  
\item
  $R \int _ a ^ b f$ exists and $ R \int _a ^ b g $ exist, then
  $$
 R \int _a ^b  f  \leq \underline \int _a ^b f \leq \overline \int _a ^b f \leq R \int _a ^b  f
  $$
  and 
  $$
  R \int _a ^b  g \leq \underline \int _a ^b g \leq \overline \int _a ^b g \leq R \int _a ^b  g
  $$
  Then the series of inequalities comes from those and from (b) :
  $$
  R \int _a ^b f + R \int _a ^b g \leq \underline \int  _a ^b f + \underline \int _a ^b g \leq
  \underline \int _a ^b ( f + g ) \leq \overline \int _a ^b ( f + g ) \leq \overline \int _a ^b f + \overline \int _a ^b g \leq R \int _a ^b f + R \int _a ^b g
  $$
  so they are all equal.  
\end{enumerate}
\end{solution}

\begin{comment}
\begin{exercise}
  problem
\end{exercise}
\begin{solution}
\begin{enumerate}[(a)]
\item
  first answer
\end{enumerate}
\end{solution}
\end{comment}




\end{document}
