\documentclass[11pt,oneside]{article}
\usepackage[hmargin=1in,vmargin=1in]{geometry}               % See geometry.pdf to learn the layout options. There are lots.
\geometry{letterpaper}                   % ... or a4paper or a5paper or ...
%\geometry{landscape}                % Activate for for rotated page geometry
\usepackage[parfill]{parskip}    % Activate to begin paragraphs with an empty line rather than an indent
\usepackage{graphicx}
\usepackage{amssymb}
\usepackage{mathrsfs}
\usepackage{epstopdf}
\usepackage{url}
%\usepackage{verbatim}
\usepackage{comment}
\specialcomment{solution}{\textbf{Solution. }}{}
%\excludecomment{solution}    %uncomment to remove solutions.

%\usepackage{enumerate}

%Use the enumitem package instead of enumerate
\usepackage[shortlabels]{enumitem}
%\usepackage{enumitem}
%then it will support the same suntax as the enumerate package.
%The enumerate package does not provide any extra configurations other than the label.

%\setlist[enumerate]{topsep=0pt,itemsep=-1ex,partopsep=1ex,parsep=1ex}
\setlist[enumerate]{topsep=0pt,partopsep=0pt}

\DeclareGraphicsRule{.tif}{png}{.png}{`convert #1 `dirname #1`/`basename #1 .tif`.png}
\usepackage{amsmath,amsthm,amscd,amssymb}
\usepackage{latexsym}
\usepackage[colorlinks,citecolor=red,pagebackref,hypertexnames=false]{hyperref}
\numberwithin{equation}{section}

\theoremstyle{definition}
\newtheorem{exercise}{Exercise}
%\newtheorem{solution}{Solution}
\newtheorem*{defn}{Definition}


\def\calA{\mathcal{A}}
\def\calB{\mathcal{B}}
\def\calC{\mathcal{C}}
\def\calT{\mathcal{T}}
\def\OR{\overline{\mathbb{R}}}
\def\RR{\mathbb{R}}
\def\CC{\mathbb{C}}
\def\FF{\mathbb{F}}
\def\QQ{\mathbb{Q}}
\def\ZZ{\mathbb{Z}}
\def\NN{\mathbb{N}}
%\def\NN{\mathbb{Z}_{> 0}}
\def\Nzero{\mathbb{Z}_{\geq 0}}
\def\EE{\mathbb{E}}
\def\PP{\mathbb{P}}
\def\supp{\mathrm{supp}}
\def\diam{\mathrm{diam}}
\def\sp{\mathrm{span}}
\def\ker{\mathrm{ker}}
\def\fancyU{\mathscr{U}}
\def\fancyL{\mathscr{L}}
\def\fancyV{\mathscr{V}}
\def\fancyP{\mathscr{P}}
%\def\sp{\mathrm{span}} %messes up align enviroment
\newcommand{\rbr}[1]{\left( {#1} \right)}
\newcommand{\sbr}[1]{\left[ {#1} \right]}
\newcommand{\cbr}[1]{\left\{ {#1} \right\}}
\newcommand{\abr}[1]{\left\langle {#1} \right\rangle}
\newcommand{\abs}[1]{\left| {#1} \right|}
\newcommand{\norm}[1]{\left\|#1\right\|}
\def\one{\mathbf{1}}
\DeclareMathOperator*{\esssup}{ess\,sup}
\newcommand*\wc{{}\cdot{}}
%\newcommand*\wc{ \, \cdot \,}
%wc for wildcard
\renewcommand{\Re}{\operatorname{Re}}
\renewcommand{\Im}{\operatorname{Im}}
\newcommand{\sgn}{\textup{sgn\,}}

\setlength{\parindent}{0pt}
\setlength{\parskip}{11pt
}
\newtheorem{lemma}{Lemma}

\begin{document}

\textbf{HW 5 - MATH 231A - Fall 2023 - Chris Lane}

\begin{exercise}
  Prove: For every $A \subseteq \RR$,

  $$
  \lim \limits _{n \to \infty} \lambda ^*(A \cap [-n, n]) = \lambda^*(A)
  $$

  If $\lambda^*(A) = \infty$, then it's straightfoward, if the $\lim$ is less than $\infty$, then $\lambda^*(A)$
  must be finite as well.  

  So assuming $\lambda ^*(A) < \infty$, we have

  Let $ \varepsilon < 0 $ be given.  $\lambda ^*(A)$ is a $\inf$ so there is an
  actual set of intervals such that:

  $$
  \sum \limits _ {i = 1} ^{\infty} \ell(I_i) < \lambda^*(A) + frac{\varepsilon}{2}
  $$

  Since the sum converges to something less than infinity, there is also an $N$ such that
  
  $$
  \sum \limits_{i=N}^{\infty} \ell(I_i) < \frac{\varepsilon}{2} 
  $$

  Then we have a finite number of intervals.  Let

  $$
  n = \max{\sup_i  {I_i}, - \inf_i {I_i}, i < N}
  $$

  Then,

  \begin{align*}
    \sum \limits _{i = 1}^{\infty} \ell(I_i) & \leq \lambda^*(A) + \frac{3 \varepsilon}{2} & \\
    \sum \limits _{i = 1}^N \ell(I_i) + \sum \limits _ {i = N}^\infty \ell(I_i) & \leq \lambda^*(A) + \frac{3 \varepsilon}{2} & \\
    \sum \limits _{i=1}^N \ell(I_i \cap [-n, n] ) + \frac{\varepsilon}{3} & \leq \lambda^*(A) + \frac{3 \varepsilon}{2} & \\
    \sum \limits _{i=1}^N \ell(I_i \cap [-n, n] ) & \leq \lambda^*(A) + \varepsilon
  \end{align*}

    Since $I_i \cap [-n, n]$ is $I_i$ for all $n > N$,
    
    $$
    \lim \limits_{n=1}^{\infty} ( \sum \limits _{i=1}^N \ell(I_i \cap [-n, n] ) )  \leq \lambda^*(A) + \varepsilon 
    $$

    Since $\varepsilon$ was arbitrary, they are equal.

    $$
    \lim \limits_{n=1}^{\infty} ( \lambda ^*( A  \cap [-n, n] ) )  = \lambda^*(A)
    $$
  
  
  
\end{exercise}
\begin{solution}
  $ \lambda ^*(A \cap [-n, n]) = \inf _ {\textrm{all covers of $A \cap [-n, n]$}} \{ \sum \limits _ {n=1}^\infty \ell(I_i), A \subseteq \bigcup \limits _{i=1} ^ {\infty} \} $

  So given some $\varepsilon > 0$, we'll show $\lambda ^* (A)$

\end{solution}

\begin{exercise}
  \begin{enumerate}[(a)]
  \item
    Prove: A set $U$ in $\RR ^ d$ is open if and only if $U$ is the union of a countable collection of open balls.
  \item
    Let $\fancyU$ be the collection of all open sets in $\RR^d$. Let
    $\fancyV$ be the collection of all open balls in $\RR ^d$.  Prove:
    $\sigma(\fancyU) = \sigma(\fancyV)$.
  \item
    Prove: If $(I_\alpha)_{\alpha \in A}$ is a collection of intervals
    in $\RR$ such that $I_\alpha \cap I_\beta \neq \varnothing$ for
    any two $\alpha, \beta \in A$,
    then $\bigcup \limits _ { \alpha \in A} I_\alpha$ is an interval.
    
  \item
    Prove: A set $U$ in $\RR$ is open if and only if $U$ is the union
    of a countable collection of disjoint open intervals.  
    
    
  \end{enumerate}
\end{exercise}
\begin{solution}
  \begin{enumerate}[(a)]
  \item
    \label{ex2a}
    Proving ``if'':
    

    If a set $U$ is a collection of countable collection of open balls, then let
    $a \in U$.

    $U = \bigcup \limits _{n=1} ^ {\infty} B(p_n, r_n)$

    $a$ is in $U$ so there is one member of the union it is in, say
    $B(p_n, r_n)$.  That is, $ a \in B(p_n, r_n)$.  $|p_n - a| < r$,
    where $| p_n - a| $ is the normal Euclidean distance.  Then the
    ball $B(a, r - |p_n - a|) \subseteq B(p_n, r_n) \subseteq U$,
    showing openness of $U$.

    Proving ``only if'':

    Enumerate all pairs of rationals, $(q_i, r_i)$.  Let $U$ be an
    open set.  For each $a \in U$, there is a $B(a, r) \subseteq U$.
    By the density of $\QQ$, there is a pair $(q_i, r_i)$ such that
    $|a-q_i| < r_i$. Then $B(q_i, r_i) \subseteq B(a, r) \subseteq U$.  

    The set of all chosen pairs, however, as a subset of a countable
    set, is countable. Enumerate it indexed by $j$ as $(q_j, r_j)$,
    $j \in \NN$.  Then $U = \bigcup \limits _{j=1} ^ \infty B(q_j,
    r_j)$, as it was constructed such that all $a \in U$ are in some
    $B(q_j, r_j)$, all of which are in $U$.

    \qed
  \item
    Let $\fancyU$ be the collection of all open sets in $\RR^d$. Let
    $\fancyV$ be the collection of all open balls in $\RR^d$.  Prove
    $\sigma(\fancyU) = \sigma(\fancyV)$.

    First, $\sigma(\fancyU) \subseteq \sigma(\fancyV)$:

    Let $U$ be an open set in $\RR^d$.  By Exercise \ref{ex2a}, there
    are balls $B(q_j, r_{kj}), k \in \ZZ$, such
    that $U = \bigcup \limits _ {j=1}^\infty B(q_j, r_j)$.
    Since $\sigma(\fancyV)$ is
    closed under finite unions, $U \in \sigma(\fancyV)$.  Since the
    basis (generators?) of $\sigma(\fancyU) \in \sigma(\fancyV)$,
    the inclusion of the $\sigma$-algebras follows.

    For $\sigma(\fancyV) \subseteq \sigma(\fancyU)$, it's rather
    trivial since open balls are open sets, so the basis (generators?)
    of $\sigma(\fancyV)$ are in $\sigma(\fancyU)$, the inclusion of
    the $\sigma$-algebras follows.

  \item
    \label{ex2c}
    Let $(I_\alpha)_{\alpha \in A}$ be a collection of intervals.  Then
    let $J = \bigcup \limits _ { \alpha \in A} I_\alpha$.  Let $a =
    \inf(J)$ and $b = \sup(J)$.  Suppose $c$ is in $(a, b)$, i.e.  $
    a< c < b$.  The claim is that $c \in J$.

    Suppose not.  Then, for every $I_\alpha$, $c \notin I_\alpha$.
    Now, there must be some $\alpha_1$ such that $(c - (c-a) / 2) \in
    J_{\alpha_1}$ (since otherwise, $a$ would not be the $\inf(J)$).
    Likewise, there must be some $\alpha_2$ such that $(c + (b -c) /2
    ) \in J_{\alpha_2}$.  Now, $I_{\alpha_1}$ and $I_{\alpha_2}$ are
    intervals.  Since $c \notin I_{\alpha_1}$ and $c \notin I_{\alpha_2}$,
    it must be that $I_{\alpha_1} \cap I_{\alpha_2} = \varnothing$ (because
    every $x \in I_{\alpha_1}$ is greater than
    $a$ and less than $c$, while every $y \in I_{\alpha_2}$ is
    greater than $c$ and less than $b$.

    But that contradicts none of the intervals having an empty overlap.

    So for all $c$ such that $ a < c < b$, $c \in J$.  Clearly,
    $ x < a$ implies that $x \notin J$, by the definition of $\inf$.
    Likewise, $x > b$ implies that $x \notin J$, by the definition of
    $\sup$.  So the only points to examine are $a$ and $b$.  There are four
    cases; either $\inf J \in J$ or not; either $\sup J \in J$ or not.  These
    correspond to the four types of intervals $(a,b)$, $[a, b)$, $(a,b]$, and
    $[a, b]$.  In all cases the result is true.

  \item
    A countable union of disjoint open intervals is open, so
    that direction is trivial.

    Given an open set, $U$, that is the union of a countable number of
    open intervals, by Exercise \ref{ex2a}.  To show the result, we
    need to coalesce all the overlapping intervals.  Enumerate the
    rationals in $U$.  For each $q_i \in U$, $q_i$ is in some
    intervals, could be finite or infinitely many; Let $I_{i,j}$ be the
    intervals containing $q_i$ and $ k = \max{j}$.
    Then, as they are overlapping,

    $$
    J_i = \bigcup \limits _ {j = 1} ^ k I_{i, j}
    $$

    And the $J_i$ are intervals, by Exercise \ref{ex2c}.

    Make $J'_i$ be open intervals with the same limits as $J_i$.  The points in $U$ will be
    interior points, not at the boundary; by construction of the open balls, every point was in the
    interior. 

    Now let $U' = \bigcup \limits _{i=1} ^ \infty J'_i$.

    Now let $x \in U$. Then the construction of the open intervals in
    Exercise \ref{ex2c}, had an 
    
      

    There's not really space in the real line for an uncountable,
    non-overlapping, set of open intervals. 
    

    
  \end{enumerate}
\end{solution}

\begin{comment}
\begin{exercise}
  problem
\end{exercise}
\begin{solution}
\begin{enumerate}[(a)]
\item
  first answer
\end{enumerate}
\end{solution}
\end{comment}


\end{document}

