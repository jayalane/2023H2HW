\documentclass[11pt,oneside]{article}
\usepackage[hmargin=1in,vmargin=1in]{geometry}               % See geometry.pdf to learn the layout options. There are lots.
\geometry{letterpaper}                   % ... or a4paper or a5paper or ...
%\geometry{landscape}                % Activate for for rotated page geometry
\usepackage[parfill]{parskip}    % Activate to begin paragraphs with an empty line rather than an indent
\usepackage{graphicx}
\usepackage{amssymb}
\usepackage{mathrsfs}
\usepackage{epstopdf}
\usepackage{url}
%\usepackage{verbatim}
\usepackage{comment}
\specialcomment{solution}{\textbf{Solution. }}{}
%\excludecomment{solution}    %uncomment to remove solutions.

%\usepackage{enumerate}

%Use the enumitem package instead of enumerate
\usepackage[shortlabels]{enumitem}
%\usepackage{enumitem}
%then it will support the same suntax as the enumerate package.
%The enumerate package does not provide any extra configurations other than the label.

%\setlist[enumerate]{topsep=0pt,itemsep=-1ex,partopsep=1ex,parsep=1ex}
\setlist[enumerate]{topsep=0pt,partopsep=0pt}

\DeclareGraphicsRule{.tif}{png}{.png}{`convert #1 `dirname #1`/`basename #1 .tif`.png}
\usepackage{amsmath,amsthm,amscd,amssymb}
\usepackage{latexsym}
\usepackage[colorlinks,citecolor=red,pagebackref,hypertexnames=false]{hyperref}
\numberwithin{equation}{section}

\theoremstyle{definition}
\newtheorem{exercise}{Exercise}
%\newtheorem{solution}{Solution}
\newtheorem*{defn}{Definition}


\def\calA{\mathcal{A}}
\def\calB{\mathcal{B}}
\def\calC{\mathcal{C}}
\def\calT{\mathcal{T}}
\def\OR{\overline{\mathbb{R}}}
\def\RR{\mathbb{R}}
\def\CC{\mathbb{C}}
\def\FF{\mathbb{F}}
\def\QQ{\mathbb{Q}}
\def\ZZ{\mathbb{Z}}
\def\NN{\mathbb{N}}
%\def\NN{\mathbb{Z}_{> 0}}
\def\Nzero{\mathbb{Z}_{\geq 0}}
\def\EE{\mathbb{E}}
\def\PP{\mathbb{P}}
\def\supp{\mathrm{supp}}
\def\diam{\mathrm{diam}}
\def\sp{\mathrm{span}}
\def\ker{\mathrm{ker}}
\def\fancyU{\mathscr{U}}
\def\fancyL{\mathscr{L}}
\def\fancyV{\mathscr{V}}
\def\fancyP{\mathscr{P}}
%\def\sp{\mathrm{span}} %messes up align enviroment
\newcommand{\rbr}[1]{\left( {#1} \right)}
\newcommand{\sbr}[1]{\left[ {#1} \right]}
\newcommand{\cbr}[1]{\left\{ {#1} \right\}}
\newcommand{\abr}[1]{\left\langle {#1} \right\rangle}
\newcommand{\abs}[1]{\left| {#1} \right|}
\newcommand{\norm}[1]{\left\|#1\right\|}
\def\one{\mathbf{1}}
\DeclareMathOperator*{\esssup}{ess\,sup}
\newcommand*\wc{{}\cdot{}}
%\newcommand*\wc{ \, \cdot \,}
%wc for wildcard
\renewcommand{\Re}{\operatorname{Re}}
\renewcommand{\Im}{\operatorname{Im}}
\newcommand{\sgn}{\textup{sgn\,}}

\setlength{\parindent}{0pt}
\setlength{\parskip}{11pt
}
\newtheorem{lemma}{Lemma}

\begin{document}

\textbf{HW 5 - MATH 231A - Fall 2023 - Chris Lane}

\begin{exercise}
  Prove: For every $A \subseteq \RR$,

  $$
  \lim \limits _{n \to \infty} \lambda ^*(A \cap [-n, n]) = \lambda^*(A)
  $$

\end{exercise}
\begin{solution}
\end{solution}

\begin{exercise}
  \begin{enumerate}[(a)]
  \item
    Prove: A set $U$ in $\RR ^ d$ is open if and only if $U$ is the union of a countable collection of open balls.  
  \end{enumerate}
\end{exercise}
\begin{solution}
  \begin{enumerate}[(a)]
  \item
    Proving ``if'':

    If a set $U$ is a collection of countable collection of open balls, then let
    $a \in U$.

    $U = \bigcup \limits _{n=1} ^ {\infty} B(p_n, r_n)$

    $a$ is in $U$ so there is one member of the union it is in, say
    $B(p_n, r_n)$.  That is, $ a \in B(p_n, r_n)$.  $|p_n - a| < r$,
    where $| p_n - a| $ is the normal Euclidean distance.  Then the
    ball $B(a, r - |p_n - a|) \subseteq B(p_n, r_n) \subseteq U$,
    showing openness of $U$.

    Proving ``only if'':

    Enumerate all pairs of rationals, $(q_i, r_i)$.  Let $U$ be an
    open set.  For each $a \in U$, there is a $B(a, r) \subseteq U$.
    By the density of $\QQ$, there is a pair $(q_i, r_i)$ such that
    $|a-q_i| < r_i$. Then $B(q_i, r_i) \subseteq B(a, r) \subseteq U$.  

    The set of all chosen pairs, however, as a subset of a countable
    set, is countable. Enumerate it indexed by $j$ as $(q_j, r_j)$,
    $j \in \NN$.  Then $U = \bigcup \limits _{j=1} ^ \infty B(q_j,
    r_j)$, as it was constructed such that all $a \in U$ are in some
    $B(q_j, r_j)$, all of which are in $U$.

    \qed
  \item
    Let $\fancyU$ be the collection of all open sets in $\RR^d$. Let
    $\fancyV$ be the collection of all open balls in $\\R^d$.  Prove
    $\sigma(\fancyU) = \sigma(\fancyV)$.

    First, $\sigma(\fancyU) \subseteq \sigma(\fancyV)$:

    Let $U$ be an open set in $\RR^d$.  

    

    
  \end{enumerate}
\end{solution}

\begin{comment}
\begin{exercise}
  problem
\end{exercise}
\begin{solution}
\begin{enumerate}[(a)]
\item
  first answer
\end{enumerate}
\end{solution}
\end{comment}


\end{document}

