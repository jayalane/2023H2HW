\documentclass[11pt,oneside]{article}
\usepackage[hmargin=1in,vmargin=1in]{geometry}               % See geometry.pdf to learn the layout options. There are lots.
\geometry{letterpaper}                   % ... or a4paper or a5paper or ...
%\geometry{landscape}                % Activate for for rotated page geometry
%\usepackage[parfill]{parskip}    % Activate to begin paragraphs with an empty line rather than an indent
\usepackage{graphicx}
\usepackage{amssymb}
\usepackage{epstopdf}
\usepackage{url}
%\usepackage{verbatim}
\usepackage{comment}
\specialcomment{solution}{\textbf{Solution. }}{}
%\excludecomment{solution}    %uncomment to remove solutions.

%\usepackage{enumerate}

%Use the enumitem package instead of enumerate
\usepackage[shortlabels]{enumitem}
%\usepackage{enumitem}
%then it will support the same suntax as the enumerate package.
%The enumerate package does not provide any extra configurations other than the label.

%\setlist[enumerate]{topsep=0pt,itemsep=-1ex,partopsep=1ex,parsep=1ex}
\setlist[enumerate]{topsep=0pt,partopsep=0pt}

\DeclareGraphicsRule{.tif}{png}{.png}{`convert #1 `dirname #1`/`basename #1 .tif`.png}
\usepackage{amsmath,amsthm,amscd,amssymb}
\usepackage{latexsym}
\usepackage[colorlinks,citecolor=red,pagebackref,hypertexnames=false]{hyperref}
\numberwithin{equation}{section}

\theoremstyle{definition}
\newtheorem{exercise}{Exercise}
%\newtheorem{solution}{Solution}
\newtheorem*{defn}{Definition}


\def\calA{\mathcal{A}}
\def\calB{\mathcal{B}}
\def\calC{\mathcal{C}}
\def\calT{\mathcal{T}}
\def\OR{\overline{\mathbb{R}}}
\def\RR{\mathbb{R}}
\def\CC{\mathbb{C}}
\def\FF{\mathbb{F}}
\def\QQ{\mathbb{Q}}
\def\ZZ{\mathbb{Z}}
\def\NN{\mathbb{N}}
%\def\NN{\mathbb{Z}_{> 0}}
\def\Nzero{\mathbb{Z}_{\geq 0}}
\def\EE{\mathbb{E}}
\def\PP{\mathbb{P}}
\def\supp{\mathrm{supp}}
\def\diam{\mathrm{diam}}
\def\sp{\mathrm{span}}
\def\ker{\mathrm{ker}}
%\def\sp{\mathrm{span}} %messes up align enviroment
\newcommand{\rbr}[1]{\left( {#1} \right)}
\newcommand{\sbr}[1]{\left[ {#1} \right]}
\newcommand{\cbr}[1]{\left\{ {#1} \right\}}
\newcommand{\abr}[1]{\left\langle {#1} \right\rangle}
\newcommand{\abs}[1]{\left| {#1} \right|}
\newcommand{\norm}[1]{\left\|#1\right\|}
\def\one{\mathbf{1}}
\DeclareMathOperator*{\esssup}{ess\,sup}
\newcommand*\wc{{}\cdot{}}
%\newcommand*\wc{ \, \cdot \,}
%wc for wildcard
\renewcommand{\Re}{\operatorname{Re}}
\renewcommand{\Im}{\operatorname{Im}}
\newcommand{\sgn}{\textup{sgn\,}}


\setlength{\parindent}{0pt}
\setlength{\parskip}{11pt
}
\newtheorem{lemma}{Lemma}

\begin{document}

\textbf{HW 3 - MATH 231A - Fall 2023 - Chris Lane}

\begin{exercise}
  Let $(X, \mathcal{A}, \mu)$ be a measure space.  Prove:
  \begin{enumerate}
  \item
    Continuity from Below: If $A_1, A_2, ... \in \mathcal{A}$ and $A_1 \subseteq A_2 \subseteq ...$, then
    $\mu ( \bigcup \limits _{n=1} ^\infty A_n ) = \lim _n  \mu ( A_n)$.
  \item
    Continuity from Above: If $A_1, A_2, ... \in \mathcal{A}$ and $A_1 \subseteq A_2 \subseteq ...$, then
    $\mu ( \bigcup \limits _{n=1} ^\infty A_n ) = \lim _n \mu(A_n)$.
  \end{enumerate}
\end{exercise}
\begin{solution}
  \begin{enumerate}
  \item
    Let $B_i$ be defined as
    \begin{align*}
      B_1 &= A_1 & \\
      B_2 &= A_2 \setminus A_1 & \\
      B_3 &= A_3 \setminus A_2 & \\
      ...\\
      B_i & = A_i \setminus A_{i-i} & \\
      ...\\
    \end{align*}
    Then $\bigcup \limits _{n=1} ^ \infty B_i = \bigcup \limits _{n=1}
    ^ \infty A_i$; if $x \in \bigcup \limits _ {n=1} ^ \infty A_n$ ,
    then there is a first $n$ such that $x \in A_n$.  Then $x \in B_n$
    (and not in $A_{n-1}$ or $n$ wouldn't be the first such $n$;
    therefore due to the set subtraction and that $A_n \subseteq A_m$,
    $x \notin B_m$, $m > n$, because $A_n \subseteq A_m$, so $x$ is subtracted out of $B_M$.

    Also note that $A_n = \bigcup \limits _ {i = 1}^n B_i$.  This is easily seen by induction.  The base case is true by defintion, $A_1 = B_1$.
    For $A_n$, $B_n = A_n \setminus A_{n-1}$.  So, 

    \begin{align*} 
      \bigcup \limits _ {i=1}^n B_i &= ( A_n \setminus ( A_{n-1} ) \cup \bigcup \limits _ {i=1} ^ {n-1} B_i &\\
       &= ( A_n \setminus ( A_{n-1} ) \cup A_{n-1} & \\
       &= A_n \\
    \end{align*}

    Unlike the $A_i$, the $B_i$ are non-overlapping.
    
    Let $x \in B_n$ for some (first) $n$.  Then $x \in A_n$. Then for
    all $m > n$, $x \notin B_m$, as $B_{m} = A_{m} \setminus A_{m-1}$, and
    $x \in A_{m-1}$, when $m > n$. 

    So, by (ii) of the definition of a measure space,
    $$
    \mu ( \bigcup \limits _ {i=1} ^ \infty B_i ) = \sum \limits _ {i=1} ^ \infty \mu (B_i)
    $$
    Since the union of the $A_i$ and the $B_i$ are equal, 
    $$
    \mu ( \bigcup \limits _ {i=1} ^ \infty B_i ) =    \mu ( \bigcup \limits _ {i=1} ^ \infty A_i)
    $$
    Now we have to show that
    $$
    \lim _ {n \to \infty} \mu ( A_n ) = \sum \limits _ {i=1} ^ \infty \mu (B_i)
    $$
    However,
    $$
    \mu (A_n) =  \mu \bigcup \limits _ {i=1} ^ n (B_i)
    $$
    So, 
    $$
    \mu(A_n) = \sum \limits _ {i=i} ^ n \mu (B_i).
    $$
    Taking limits (the sequences are identical at every term and we are in the extended reals so convergance is not an issue):

    $$
    \lim _ {n \to \infty} \mu ( A_n ) = \sum \limits _ {i=1} ^ \infty \mu (B_i)
    $$
    As needed to prove the result \qed
  \end{enumerate}
\end{solution}

\begin{exercise}
  Let $X$ be a set.  The \textbf{counting measure} for $X$ is the
  measure $\mu _c: \mathcal{P}(X) \to [0, \infty]$ defined
  by $\mu_c(A) = \infty$ if $A$ is infinite and $\mu_c (A) = | A | $ if
  $A$ is finite. Here $|A|$ denotes the cardinality of $A$.
  Prove $\mu _c$ is (actually) a measure.
\end{exercise}
\begin{solution}
  There are two things to verify:
  \begin{enumerate}[(i)]
    \item
      $\mu_c ( \varnothing ) = 0$: 

      $  | \varnothing | = 0$, so this requirement is true.
    \item
      $ \mu _c ( \bigcup \limits_{i=1} ^{\infty} A_i )= \sum \limits _{i=1} ^ \infty \mu_c (A _i)$
      where $A_1, A_2, ... \in \mathcal(A)$ are disjoint:

      There are three cases for the $A_i$.  If any of
      the $|A_i| = \infty$, then
      the $| \bigcup \limits_{i=1} ^ \infty A_i | = \infty$, and the condition is met.
      
      If $|A_i| < \infty$, for all the $A_i$, but there are an
      infinite number of them not equal to the empty set, then the sum
      of their cardinality diverges to $\infty$ and as well the union
      is countably infinite (otherwise either they aren't disjoint or are empty).

      So the last case is a finite union of disjoint sets of finite size, which
      is, by elementary set theory, finite and equal to the sum of the sizes. 

      That is, where $ M$ is sufficiently large that $A_i = \varnothing, i > M $:

      $$
      | \bigcup \limits _ {i=1} ^ M A_i | = \sum \limits _ { i=1} ^ M | (A_i) |
      $$
      or 
      $$
      \mu_c ( \bigcup \limits _ {i=1} ^ \infty A_i ) = \sum \limits _ { i=1} ^ \infty \mu_c (A_i) 
      $$

      as required to be a measure.  \qed
      
  \end{enumerate}
\end{solution}

\begin{exercise}
  Show that the Lebesgue outer measure $ \lambda ^ *$ is translation invarient.  In other words,
  show that
  $$
  \lambda ^*(A + t) = \lambda ^*(A)
  $$
  for all $A \subseteq \RR$ and $t \in \RR$.  Here $A + t = \{ a + t: a \in A \}$.  
\end{exercise}
\begin{solution}
  $$
  \lambda ^* (A+t) = \inf \{ \sum \limits _ i ^ \infty \ell (I_i) : I_1, I_2, ... \text{such that} \ I_i \ \text{are intervals} \ 
  A + t \subseteq \bigcup \limits _{i_1} ^ \infty I_i \}
  $$
  So looking at the elements of the set over which the $\inf$ is being taken, we have
  $$
  \inf \{ \sum \limits _ i ^ \infty \ell (I_i) : I_1, I_2, ... \}
  $$

  I claim that for each collection of $I_i$s that cover $A+t$,
  there is a collection of $I'_i$s that cover $A$, and vice versa.  if
  $I_i = [a_i, b_i]$, then $I'_i = [a_i-t, b_i-t]$.  Let $x \in A + t $.
  Then there are some $I_i$ s.t. $ x \in I_i$, i.e. $ a_i  \leq x \leq b_i$.
  But $x = a + t$, for some $a \in A$.  So $ a_i - t \leq a \leq b_i$.  That is,
  $a \in I'_i$.

  Likewise for $a \in A$.  There are some $I'_i$ such that $a \in I'_i$, i.e.
  $ a'_i -t  \leq a \leq b'_i - t$.  Then, adding $t$, $a'_i \leq a + t \leq b'_i$, or
  $ a + t \in \bigcup \limits _ i ^ \infty I_i$.

  Moreover, the values $\ell(I_i) = b - a = b - t - (a - t) = \ell(I'_i)$ are equal for each such
  covering, and
  $$ \sum \limits _ {i=1} ^ \infty \ell (I_i) : I_1, I_2, ... \} = 
  \sum \limits _ {i=1} ^ \infty \ell (I'_i) : I'_1, I'_2, ... \}
  $$
  Since the sets are the same, the $\inf$ is the same as well, and the result is shown.  \qed
  
\end{solution}

\begin{comment}
\begin{exercise}
  problem
\end{exercise}
\begin{solution}
\begin{enumerate}[(a)]
\item
  first answer
\end{enumerate}
\end{solution}
\end{comment}

\end{document}
p
