\documentclass[11pt,oneside]{article}
\usepackage[hmargin=1in,vmargin=1in]{geometry}               % See geometry.pdf to learn the layout options. There are lots.
\geometry{letterpaper}                   % ... or a4paper or a5paper or ...
%\geometry{landscape}                % Activate for for rotated page geometry
\usepackage[parfill]{parskip}    % Activate to begin paragraphs with an empty line rather than an indent
\usepackage{graphicx}
\usepackage{amssymb}
\usepackage{mathrsfs}
\usepackage{epstopdf}
\usepackage{datetime}
\usepackage{url}
%\usepackage{verbatim}
\usepackage{comment}
\specialcomment{solution}{\textbf{Solution. }}{}
%\excludecomment{solution}    %uncomment to remove solutions.

%\usepackage{enumerate}

%Use the enumitem package instead of enumerate
\usepackage[shortlabels]{enumitem}
%\usepackage{enumitem}
%then it will support the same suntax as the enumerate package.
%The enumerate package does not provide any extra configurations other than the label.

%\setlist[enumerate]{topsep=0pt,itemsep=-1ex,partopsep=1ex,parsep=1ex}
\setlist[enumerate]{topsep=0pt,partopsep=0pt}

\DeclareGraphicsRule{.tif}{png}{.png}{`convert #1 `dirname #1`/`basename #1 .tif`.png}
\usepackage{amsmath,amsthm,amscd,amssymb}
\usepackage{latexsym}
\usepackage[colorlinks,citecolor=red,pagebackref,hypertexnames=false]{hyperref}
\numberwithin{equation}{section}

\theoremstyle{definition}
\newtheorem{exercise}{Exercise}
%\newtheorem{solution}{Solution}
\newtheorem*{defn}{Definition}


\def\calA{\mathcal{A}}
\def\calB{\mathcal{B}}
\def\calC{\mathcal{C}}
\def\calT{\mathcal{T}}
\def\OR{\overline{\mathbb{R}}}
\def\RR{\mathbb{R}}
\def\CC{\mathbb{C}}
\def\FF{\mathbb{F}}
\def\QQ{\mathbb{Q}}
\def\ZZ{\mathbb{Z}}
\def\NN{\mathbb{N}}
\def\Nzero{\mathbb{Z}_{\geq 0}}
\def\EE{\mathbb{E}}
\def\PP{\mathbb{P}}
\def\supp{\mathrm{supp}}
\def\diam{\mathrm{diam}}
\def\sp{\mathrm{span}}
\def\ker{\mathrm{ker}}
\def\fancyA{\mathscr{A}}
\def\fancyU{\mathscr{U}}
\def\fancyU{\mathscr{U}}
\def\fancyL{\mathcal{L}}
\def\fancyV{\mathscr{V}}
\def\fancyP{\mathscr{P}}
\def\fancyB{\mathscr{B}}
\def\limn{\lim \limits _n}
\newcommand{\rbr}[1]{\left( {#1} \right)}
\newcommand{\sbr}[1]{\left[ {#1} \right]}
\newcommand{\cbr}[1]{\left\{ {#1} \right\}}
\newcommand{\abr}[1]{\left\langle {#1} \right\rangle}
\newcommand{\abs}[1]{\left| {#1} \right|}
\newcommand{\norm}[1]{\left\|#1\right\|}
\def\one{\mathbf{1}}
\DeclareMathOperator*{\esssup}{ess\,sup}
\newcommand*\wc{{}\cdot{}}
%\newcommand*\wc{ \, \cdot \,}
%wc for wildcard
\renewcommand{\Re}{\operatorname{Re}}
\renewcommand{\Im}{\operatorname{Im}}
\newcommand{\sgn}{\textup{sgn\,}}

\setlength{\parindent}{0pt}
\setlength{\parskip}{11pt
}
\newtheorem{lemma}{Lemma}

\begin{document}

\textbf{HW 8 - MATH 231A - Fall 2023 - Chris Lane}

Date: \hhmmsstime{} \ \today \ \ Git hash: 
\input{/Users/chlane/src/jayalane/2023H2HW/.git/refs/heads/main} 

\begin{exercise}
  Let $(X, \fancyA, \mu)$ be a measure space.
  Let $f_n : X \to \overline \RR (n = 1, 2, 3 ...) $ be a sequence of measurable
  functions.  Let $f: X \to \overline \RR $ be a measurable function.
  Suppose $0 \leq f_n$ a.e.  Suppose $f_n(x)$ is an increaseing
  sequence for a.e. $x in X$.  Suppose $\limn f_n = f$ a.e.  Prove:
  $\limn \int f_n = \int f$.
  \end{exercise}
\begin{solution}

  Let
  \begin{align*}
    A_i &= \{ x \in X : f _ n (x ) < 0 \}, \quad & \mu(A_i) = 0 \\
    B_i &= \{ x \in X : f _ n (x ) > f_{n+1} \}, \quad & \mu(B_i) = 0 \\
    C_i &= \{ x \in X : \limn f _ n (x ) \neq f(x) \}, \quad & \mu(C_i) = 0
  \end{align*}

  Let $D = \cup _i A_i \bigcup \cup _i B_i \bigcup \cup _ i C_i$.

  Then $\mu(D) = 0$, being the countable union of measure zero sets.

  Then let:

  \[ 
  F_n = \one _ D f _n
  \]

  and

  \[
  F = \one _D f
  \].

  Then $0 \leq F_n$, $F_n \leq F_{n+1}$, and $\limn F_n = F$, all for all $x \in X$.

  Then by the base Monotone Convergence Theorem,

  \[
  \int F = \limn \int F_n
  \]

  But $F = f$ a.e., so

  \[
  \int F = \int f
  \]

  By the corollary to the ``almost everywhere'' theorems in class.  
  
  Alternatively and in more detail.

  \[
  \int f = \int \limits {\one_D^c} f + \int \limits _ {\one _D} \leq \infty \int \one _D ^c + \int F = 0 + \int F
  \]

  So $\int f \leq \int F$ but

  \[
  \int F = \int F^+ - \int F^- = \int F^+
  \]

  as $ 0 \leq F $. So 

  \[
  \int F = \int \one_D f^+ \leq \int f^+
  \]

  And

  \[
  \int f^+ = \int f^+ \one_D + \int f^+ \one_D ^ c \leq \infty \times 0 + \int F^+ 
  \]

  So

  \[
  \int f^+ \leq \int F^+
  \]

  So

  \[
  \int f \leq \int F
  \]

  and $\int f = \int F$.  And so $\limn \int f  = \int f$.  \qed
\end{solution}

\begin{exercise}
  Let $(X, \fancyA, \mu)$ be a measure space.  For each $t \in [a,b]$, let $f_t : X \to \overline \RR$
  be a measurable function.  Let $c \in [a, b]$.  Supposed $g: X \to [0, \infty ] $ is a measurable
  integrable function.  Suppose:
  \begin{enumerate}[(a)]
  \item
    $\lim_{t \to c} f_t(x) = f_c(x)$ for all $x \in X$.
  \item
    $|f_t(x)| \leq g(x)$ for all $t \in [a,b]$, $x \in X$.
  \item
    $g$ is finite.
  \end{enumerate}

  Prove:

  \[
  \lim \limits _ { t \to c} \int f_t(x) \mathop{d \mu(x)} = \int f_c(x) \mathop{d \mu(x)}
  \]
\end{exercise}
\begin{solution}
  Since $\lim _ {t \to c} f_t(x) = f_c(x)$ for all $x \in X$, for every sequence $t_n \in [a, b]$
  converging to $c$, we have a nice sequence of functions $f_{t_n}$ converging to our
  $f_c$.  We have $ | f_{t_n}(x) | \leq g(x)$ for all $x \in X$ and $\limn f_{t_n}(x) = f_c(x)$.

  Therefore, by the base Dominated Cnvergence Theorem, $ \limn f_{t_n}(x) = \int f_c(x)$.

  Since this is true for every sequence $t_n \to c$, it is true of the limit in $[a, b]$:

  \[
  \lim _{t \to c} \int f_t(x) \mathop{d \mu(x)} = \int f_c(x) \mathop{d \mu(x)}
  \]

  \qed
  
\end{solution}

\begin{exercise}
  Let $(X, \fancyA, \mu)$ be a measure space.
  Suppose $f : X \to \overline \RR$ is measurable and integrable.  Prove:
  
  \begin{enumerate}[(a)]
  \item
    $ \{ x \in X : f(x) \neq 0 \}$ is $\sigma$-finite.
  \item
    There exists a sequence of sets $A_1, A_2, ... \in \fancyA$ such that $\mu(A_n) < \infty$ for all $n$
    and $f \one _ {A_n} \to f$ pointwise.  (Hint: $A_n = \cup _ { i = 1 } ^ {n} = E_i$.)  
  \item
    For every $\epsilon > 0$, there exists $ A \in \fancyA$ such that $\mu(A) <\infty$ and
    $| \int f - \int f \one _A | \leq \epsilon$.  
  \end{enumerate}
\end{exercise}

\begin{solution}
\begin{enumerate}[(a)]
\item
  \label{ex3a}
  Let $E_n = \{ x \in X : | f(x) |  > \frac{1}{n} \}$.  If we can show that $\mu(E_n) < \infty$ then
  since $B = \cup _ n E_n = \{ x \in X : f(x) \neq 0 \}$ we will have shown that $B$ is $\sigma$-finite.

  To show $\mu(E_n)$ is finite:

  \begin{align*}
    \int f & < \infty \ f \textrm{ integrable} \\
    \int f^+ & < \infty \ f \textrm{ integrable} \\
    \int f^- & < \infty \ f \textrm{ integrable} \\
    \int f^+ + \int f^- & < \infty \\
    \int |f|  & < \infty \\
  \end{align*}

  So on $\one_{E_n}$ we get:
  \[
  \int \limits _{\one_{E_n}} |f | \leq   \int |f | \leq \infty
  \]

  But also:

  \[
  \int \limits _ {\one_{E_n}} |f|  > \frac{1}{n} \mu(E_n) 
  \]

  Combining these,

  \[
  \frac{1}{n} \mu(E_n) < \int \limits _{\one_{E_n}} |f| \leq \int |f| < \infty
  \]

  Or
  
  \[
  \mu(A_n) < \infty \times n = \infty
  \]

  As needed.  \qed

\item
  \label{ex3b}
  Taking the $E_n$ from Exercise 3 \ref{ex3a}, define $A_n = \cup _ {i \leq n} E_i$ and  let $x \in X$ be specified.

  If $f(x) \neq 0$, then $f(x) > \frac{1}{N}$ for some $N \in \NN$.  Then the $A_n$ contain $x$ for $ n \geq N$
  and the sequence clearly converges there to $f(x)$ as its identitically equal to $f(x)$ after $N$.

  So we need to show that the limit
  of $\limn \one _ { A_N} f(x) = f(x)$ when $f(x) = 0$.  But this is obvious because the sequence is identically
  zero which is also the desired convergence limit. \qed

  \item

    Starting with the case that $f = f^+$, and using $A_n$ from Exercise 3 \ref{ex3b}, we have $f_n = f \one_{A_n}$.

    So then we get $ 0 \leq f_n \leq f$.  Exercise 3 \ref{ex3b} showed
    $\limn f_n = f$ and $f_n \leq f_{n+1}$ by construction, so the
    conditions of the monotone convergence theorem apply, and

    \[
    \int f = \limn f_n
    \]

    That is,
    \[
    \limn f \one _{A_n} = f
    \]

    By the definition of limit, given some $\epsilon > 0$, there is an $N \in \NN$ such that
    for all $j \geq N$,

    \[
    | \int f - \int \one_{A_n} f | < \epsilon
    \]

    But this just means $A_n$ is the desired set of finite measure.

    Now, for the case where $f \neq f^+$, we split it into two applications of the above,
    choosing $N_1, N_2$ such that

    \[
    | \int f^+ - \int \one_{A_j} f^+ | < \frac{\epsilon}{3}, \ \ j \geq N_1
    \]

    and
    
    \[
    | \int f^- - \int \one_{A_k} f^- | < \frac{\epsilon}{3}, \ \ j \geq N_2
    \]

    Then for $n = \max(N_1, N_2) + 1$, let $A$ be $A_n$.  
    
    \begin{align*}
      | \int f - \int f \one _ A | &= | \int f^+ - \int f^- - \int f+ \one_A + \int f^- \one_A | \\
       &= | \int f^+ - \int f^+ \one_A - (\int f^- - \int f^- \one_A) | \\
      & \leq | \int f^+ - \int f^+ \one_A|  + | \int f^- - \int f^- \one_A | \\
      & \leq \frac{\epsilon}{3} +  \frac{\epsilon}{3} \\
      & < \epsilon
    \end{align*}

    As needed.  \qed
    
\end{enumerate}
  
  
  
\end{solution}


\begin{comment}
\begin{exercise}
  problem
\end{exercise}
\begin{solution}
\begin{enumerate}[(a)]
\item
  first answer
\end{enumerate}
\end{solution}
\end{comment}


\end{document}
