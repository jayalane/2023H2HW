\documentclass[11pt,oneside]{article}
\usepackage[hmargin=1in,vmargin=1in]{geometry}               % See geometry.pdf to learn the layout options. There are lots.
\geometry{letterpaper}                   % ... or a4paper or a5paper or ...
%\geometry{landscape}                % Activate for for rotated page geometry
%\usepackage[parfill]{parskip}    % Activate to begin paragraphs with an empty line rather than an indent
\usepackage{graphicx}
\usepackage{amssymb}
\usepackage{epstopdf}
\usepackage{url}
%\usepackage{verbatim}
\usepackage{comment}
\specialcomment{solution}{\textbf{Solution. }}{}
%\excludecomment{solution}    %uncomment to remove solutions.

%\usepackage{enumerate}

%Use the enumitem package instead of enumerate
\usepackage[shortlabels]{enumitem}
%\usepackage{enumitem}
%then it will support the same suntax as the enumerate package.
%The enumerate package does not provide any extra configurations other than the label.

%\setlist[enumerate]{topsep=0pt,itemsep=-1ex,partopsep=1ex,parsep=1ex}
\setlist[enumerate]{topsep=0pt,partopsep=0pt}

\DeclareGraphicsRule{.tif}{png}{.png}{`convert #1 `dirname #1`/`basename #1 .tif`.png}
\usepackage{amsmath,amsthm,amscd,amssymb}
\usepackage{latexsym}
\usepackage[colorlinks,citecolor=red,pagebackref,hypertexnames=false]{hyperref}
\numberwithin{equation}{section}

\theoremstyle{definition}
\newtheorem{exercise}{Exercise}
%\newtheorem{solution}{Solution}
\newtheorem*{defn}{Definition}


\def\calA{\mathcal{A}}
\def\calB{\mathcal{B}}
\def\calC{\mathcal{C}}
\def\calT{\mathcal{T}}
\def\OR{\overline{\mathbb{R}}}
\def\RR{\mathbb{R}}
\def\CC{\mathbb{C}}
\def\FF{\mathbb{F}}
\def\QQ{\mathbb{Q}}
\def\ZZ{\mathbb{Z}}
\def\NN{\mathbb{N}}
%\def\NN{\mathbb{Z}_{> 0}}
\def\Nzero{\mathbb{Z}_{\geq 0}}
\def\EE{\mathbb{E}}
\def\PP{\mathbb{P}}
\def\supp{\mathrm{supp}}
\def\diam{\mathrm{diam}}
\def\sp{\mathrm{span}}
\def\ker{\mathrm{ker}}
%\def\sp{\mathrm{span}} %messes up align enviroment
\newcommand{\rbr}[1]{\left( {#1} \right)}
\newcommand{\sbr}[1]{\left[ {#1} \right]}
\newcommand{\cbr}[1]{\left\{ {#1} \right\}}
\newcommand{\abr}[1]{\left\langle {#1} \right\rangle}
\newcommand{\abs}[1]{\left| {#1} \right|}
\newcommand{\norm}[1]{\left\|#1\right\|}
\def\one{\mathbf{1}}
\DeclareMathOperator*{\esssup}{ess\,sup}
\newcommand*\wc{{}\cdot{}}
%\newcommand*\wc{ \, \cdot \,}
%wc for wildcard
\renewcommand{\Re}{\operatorname{Re}}
\renewcommand{\Im}{\operatorname{Im}}
\newcommand{\sgn}{\textup{sgn\,}}


\setlength{\parindent}{0pt}
\setlength{\parskip}{11pt
}
\newtheorem{lemma}{Lemma}

\begin{document}

\textbf{HW 2 - MATH 231A - Fall 2023 - Chris Lane}
\begin{exercise}
  Prove if $f:[a, b] \mapsto \RR $ is bounded and $f$ is continuous
  except at finitely many points of $[a,b]$, then $f$ is Riemann integrable on $[a,b ]$
\end{exercise}
\begin{solution}
  We'll use the Riemann integrability criterion and construct a
  partition such that $L(f, P) \leq U(f, P) + \varepsilon$ showing that
  the integral exists.  (part finished prior to class) So given an $\varepsilon$, we
  choose two types of intervals, ones around the points of
  discontinuity, call them $c_i$ for $i$ from 1 to $k$, and ones
  splitting up the sections of continuity up into very
  small $\delta_{\varepsilon}$ strips.
  $$
  P = B \cup D
  $$
  where $B$ is:
  $$
  B = \{ \max(a, c_i - \rho), \min(c_i + \rho, b), \quad \text{for each} \ \  i, 1 \leq i \leq k \}
  $$
  And $D$ is:
  $$
  D = \{ a + i \frac{(b - a)}{n}, 0 \leq i \leq n, i \in \ZZ \}
  $$
  The two free parameters above $n$ and $\rho$ need to be calculated.  Let $M_f$ be the
  $ \sup \limits_{[a,b]} f$ and $m_f$ be the $ \inf \limits_{[a,b]} f$.  Then $\rho$ should be
    small enough that the $k$ points of continuity of max $M_i - m_i < M_f - m_f$ difference between
    $L$ and $U$ contribute less than say $\frac{\varepsilon}{3(b-a)}$.
    $$ k (2 \rho) (M_f - m_f) < \frac{\varepsilon}{3(b-a)} $$
    So let 
    $$
    \rho < {\frac{\varepsilon}{6k (M_f - m_f)(b-a)} }
    $$
    
    For the wider strips, we have a finite number of open intervals on which $f$ is continuous.  $f$ is also
    bounded, so on each of these intervals, $f$ is uniformly continuous.  That means, for each such interval $I_i$, for
    all $\frac{\varepsilon}{3(b-a)}$, and for all $x_1, x_2 \in I_i$, there is a $\delta_i$ such that
    $$
    | x_1 - x_2 | < \delta_i \implies | f(x_1) - f(x_2) < \frac{\varepsilon}{3(b-a)}
    $$
    Then, let $\delta = \min \limits_i \delta_i$.  Now, set $n$ in the partition so that $\frac{(b-a)}{n} \leq \delta$.

    Now,
    $$ L(f,P) = \sum \limits _ { I_i } m_i (x_i - x_{i-1}) + \sum \limits _k m_k (2 * \rho) $$
    and
    $$ U(f,P) = \sum \limits _ { I_i } M_i (x_i - x_{i-1}) + \sum \limits _k M_k (2 \rho) $$
    So
    $$ U(f,P) - L(f,P) = \sum \limits_{I_i} (M_i - m_i) ( x_i - x_{i-1}) + \sum \limits _k (M_k - m_k) (2 \rho) $$
    But, $ (M_i - m_i) = f(x_1) - f(x_2) $ for some $ x_1, x_2 \in I_i$, and so, by the above uniform continuity,
    $$
    (M_i - m_i) \leq \frac{\varepsilon}{3(b-a)} (b-a) \leq \frac{\varepsilon}{3}
    $$
    And for the sum:
    $$
    \sum \limits_{I_i} (M_i - m_i) (x_i - x_{i-1}) \leq \frac{\varepsilon}{3}
    $$
    For the second sum,
    \begin{align*}
      \sum \limits _k (M_k - m_k) (2 \rho) & \leq \sum \limits _k (M_f - m_f) {\frac{\varepsilon}{3} \over 2k (M_f - m_f)}  & \\
      & \leq \frac{(M_f - m_f) \varepsilon}{3(M_f - m_f)} & \\
      & \leq \frac{\varepsilon}{3}  & \\
    \end{align*}
    $(M_k - m_k) \leq (M_f - m_f)$ because the $\sup$ over a smaller set is smaller, and the $\inf$ over a smaller set is larger,
    so the difference over the smaller set is smaller (or equal) to the difference over the larger set.
    So, combining the two sums,
    $$
    U(f, P) - L(f, P) \leq \frac{\varepsilon}{3} + \frac{\varepsilon}{3} < \varepsilon
    $$
    So we have construction a partition for our $\varepsilon$ and hence fulfilled the conditions of the
    Riemann integrability condition.  \qed
    
    
  
\end{solution}

\begin{exercise}
  Define $f : [a,b ] \mapsto \RR $ as:
  $$
  f(x) = \begin{cases}
    0 & \text{if} \quad x \in \RR \setminus \QQ \\
    \frac{1}{n} & \text{if} \quad x \in \QQ, \text{where}\ \  x = \frac{m}{n}, \ \ 
    \text{with no common factors} \\
  \end{cases}
  $$
  Show that $f$ is Riemann integrable and compute $\int _ {a} ^ {b} f$.  
\end{exercise}


\begin{solution}

  We'll use the Riemann integrability condition theorem and given an $\varepsilon$ will produce
  a partition $P$ such that $U(f, P) - L(f,P) < \varepsilon$, proving the integral exists.  
  
  Additionally, we'll calculate $\underline \int  _{a} ^ {b} f = 0$ showing
  that the integral is in fact $0$ for all $[a,b]$.    


  Let $\varepsilon > 0$.  $L(f,P)$ is identically $0$ across all
  paritions, as every interval contains irrational numbers $x$ where $f(x) = $,
  so the $ \inf \limits_{\text{segment of partition}} = 0$.

  So the condition reduces to $U(f, P) < \varepsilon$.

  Let $n$ be such that $ n > \frac{1}{\varepsilon}$.  Divide
  the $[a,b]$ interval into very small intervals around the numbers
  expressable as $m/n$ with no common factors.  Let $c_1$ be the
  smallest number in $[a,b]$ such that $c_1 = m/n$ where $m$ and $n$
  have no common factors.

  $$
  P = \{ a, \max(a, c_1 - \delta), \min(c_i + \delta, b), \min(c_1 + 1/n - \delta, b), \min(c_1 + 1/n + \delta), ... , \min(c_1 + k/n + \delta, b), b \}
  $$

  
  For $\underline \int _a ^b$,
  the $\inf \limits_{\text{any partition}} L(f, P)$ is $0$;
  for any interval I in the
  partition, $\inf \limits _ {x \in I} f(x) = 0$ as there are
  irrational numbers, and hence places where $f(x) = 0$, between any
  two interval endpoints.

  As above, there are two types of strips, ones that contain rationals
  of the form $k/n$ and those which only contain rationals with smaller denominators.
  So
  \begin{align*}
    U(f, P) & = \sum \limits_{(b-a)/n \text{intervals}} M_i 2 \delta + \sum \limits_{wider intervals}M_i (x_i - x_{i-1}) & \\
    & \leq \sum \limits_{(b-a)/n \text{intervals}} 2 \delta + (b-a) * 1/n & \\
  \end{align*}
  So $ U(f,P) < 2 \delta (b-a)/n + (b-a)/n$.  We want this to be $< \varepsilon / 2$, so this means:
  \begin{align*}
    2 \delta (b -a ) / n + (b-a) /n & < \varepsilon / 2 & \\
    (2 \delta + 1) (b - a) /n & < \varepsilon / 2 & \\
    2 \delta + 1 & < \frac{ \varepsilon n }{2(b-a)} & \\
      2\delta & < \frac{n \varepsilon - 2(b-a)}{2(b-a)} & \\
      \delta & < \frac{n \varepsilon - 2(b-a)}{4(b-a)} & \\
  \end{align*}

  So if we choose $n$ so that $n > 2(b-a)$ and $\delta$ as above, then the U(f, P) for the resultant partition is less than $\varepsilon$, and the conditions are met,
  the integrals exist, and by the calculation on L(f, P),
  $$
  \underline \int _a ^b f = \overline \int _a ^b f = \text{R} \int _a ^b f = 0
  $$
  \qed
  
\end{solution}

\begin{exercise}
  Let $X$ be a set. Let $ \calA = \{A \subset X : A \ \ \text{is 
    countable or} \ \ A^c \ \ \text{is countable} \}$.  Prove $\calA$ is a
    $\sigma$-algebra on $X$. (Note: Finite sets are countable. So
    $\varnothing$ is countable.)
\end{exercise}
\begin{solution}
  \begin{enumerate}[(i)]
    \item
      As noted, $\varnothing $ is countable, and $\varnothing \in \calA$
    \item
      If $A \in \calA$, then by construction, $A^c \in \calA$.
    \item
      Given a sequence of sets
      $$
      A_1, A_2, ... \in \calA
      $$
      is
      $$ \bigcup _ {n =1} ^ \infty A_n
      $$ in $\calA$?

      There are two cases.  Each $A \in \calA$ is either countable or
      a complement of a countable set.  Case one is $A_n$ is countable
      for all $n$.  In this case, the union of a countable set is again countable, and
      $ \bigcup _ {n =1} ^ \infty A_n $ is in $\calA$ by construction.

      In the second case, at least one of $A_n$ is not countable, but
      the complement of a countable set.  Split the union into two,
      $B_1, ... $ where $B_n$ is countable, and $C_1, C_2,...$ where
      each $C_n^c$ is countable.

      Consider the union of the $C_n$
      $$
      \bigcup _ {n=1}^\infty C_n
      $$

      Let $D_n = C_n ^c$.  
      \begin{align*}
      \bigcup _ {n=1}^\infty C_n = & \bigcup _ {n=1}^\infty D_n^c \\ 
      = &  \{ \bigcap _{n=1}^\infty D_n \} ^ c
      \end{align*}
      By DeMorgan's law.  The countable intersection of the $D_n$ is
      also countable, so
      $$
      \bigcup _ {n=1}^\infty C_n \in \calA
      $$
      Call the countable set in its complement $ \{ a_1, a_2, ... \}$
      Also, take the union of all the $B_n$, the union of countable sets
      is countable, so for some enumeration
      $$
      \bigcup _{n=1}^\infty B_n = \{ b_1, b_2, b_3, ... \}
      $$
      So the whole union is now
      $$
      ( X \setminus \{a_1, a_2, a_3, ... \}) \bigcup \{b_1, b_2, ... \}
      $$
      Letting $ A = \{ a_1, a_2, ... \}$ and 
      $ B = \{ b_1, b_2, ... \} $, we have $A \subset X $ and $B \subset X$ and the
      above union becomes:
      $$
      (X \setminus A ) \cup B = X \setminus ( A \setminus B) = (A \setminus B) ^c
      $$
      $A \setminus B$ is a subset of $A$ and so is countable, so the union is the complement of a
      finite set, as needed.  \qed

      
      
  \end{enumerate}

\end{solution}

\begin{comment}
\begin{exercise}
  problem
\end{exercise}
\begin{solution}
\begin{enumerate}[(a)]
\item
  first answer
\end{enumerate}
\end{solution}
\end{comment}

\end{document}
