\documentclass[11pt,oneside]{article}
\usepackage[hmargin=1in,vmargin=1in]{geometry}               % See geometry.pdf to learn the layout options. There are lots.
\geometry{letterpaper}                   % ... or a4paper or a5paper or ...
%\geometry{landscape}                % Activate for for rotated page geometry
%\usepackage[parfill]{parskip}    % Activate to begin paragraphs with an empty line rather than an indent
\usepackage{graphicx}
\usepackage{datetime}
\usepackage{amssymb}
\usepackage{epstopdf}
\usepackage{url}
%\usepackage{verbatim}
\usepackage{comment}
\specialcomment{solution}{\textbf{Solution. }}{}
%\excludecomment{solution}    %uncomment to remove solutions.

%\usepackage{enumerate}

%Use the enumitem package instead of enumerate
\usepackage[shortlabels]{enumitem}
%\usepackage{enumitem}
%then it will support the same suntax as the enumerate package.
%The enumerate package does not provide any extra configurations other than the label.

%\setlist[enumerate]{topsep=0pt,itemsep=-1ex,partopsep=1ex,parsep=1ex}
\setlist[enumerate]{topsep=0pt,partopsep=0pt}

\DeclareGraphicsRule{.tif}{png}{.png}{`convert #1 `dirname #1`/`basename #1 .tif`.png}
\usepackage{amsmath,amsthm,amscd,amssymb}
\usepackage{latexsym}
\usepackage[colorlinks,citecolor=red,pagebackref,hypertexnames=false]{hyperref}
\numberwithin{equation}{section}

\theoremstyle{definition}
\newtheorem{exercise}{Exercise}
%\newtheorem{solution}{Solution}
\newtheorem*{defn}{Definition}
\newtheorem*{claim}{Claim}

\def\pmod{+'}
\def\boldr{\boldsymbol{r}}
\def\boldc{\boldsymbol{c}}

\def\calA{\mathcal{A}}
\def\calB{\mathcal{B}}
\def\calC{\mathcal{C}}
\def\calT{\mathcal{T}}
\def\OR{\overline{\mathbb{R}}}
\def\RR{\mathbb{R}}
\def\CC{\mathbb{C}}
\def\FF{\mathbb{F}}
\def\QQ{\mathbb{Q}}
\def\ZZ{\mathbb{Z}}
\def\NN{\mathbb{N}}
%\def\NN{\mathbb{Z}_{> 0}}
\def\Nzero{\mathbb{Z}_{\geq 0}}
\def\EE{\mathbb{E}}
\def\PP{\mathbb{P}}
\def\supp{\mathrm{supp}}
\def\diam{\mathrm{diam}}
\def\sp{\mathrm{span}}
\def\ker{\mathrm{ker}}
\def\Aut{\operatorname{Aut}}
\def\Inn{\operatorname{Inn}}
\def\Ker{\operatorname{Ker}}
%\def\sp{\mathrm{span}} %messes up align enviroment
\newcommand{\Mod}{\ (\mathrm{mod}\ )}
\newcommand{\rbr}[1]{\left( {#1} \right)}
\newcommand{\sbr}[1]{\left[ {#1} \right]}
\newcommand{\cbr}[1]{\left\{ {#1} \right\}}
\newcommand{\abr}[1]{\left\langle {#1} \right\rangle}
\newcommand{\abs}[1]{\left| {#1} \right|}
\newcommand{\norm}[1]{\left\|#1\right\|}
\def\one{\mathbf{1}}
\DeclareMathOperator*{\esssup}{ess\,sup}
\newcommand*\wc{{}\cdot{}}
%\newcommand*\wc{ \, \cdot \,}
%wc for wildcard
\renewcommand{\Re}{\operatorname{Re}}
\renewcommand{\Im}{\operatorname{Im}}
\newcommand{\sgn}{\textup{sgn\,}}


\setlength{\parindent}{0pt}
\setlength{\parskip}{11pt}

%\title{\parbox{14cm}{\centering{  Interior points of circle and sphere packings}}}
\begin{document}

\textbf{HW 7 - MATH 221A - Fall 2023 - Chris Lane}

Date: \hhmmsstime{} \ \today \ \ Git hash: 
\input{/Users/chlane/src/jayalane/2023H2HW/.git/refs/heads/main}

\begin{exercise}
  Let $G$ act on left cosets of $H$ by multiplication.
  \begin{enumerate}[(a)]
  \item
    Show that the kernel of the action is a subgroup of $H$.  
  \item
    Show that the kernel of the action is $N = \cap _{x \in G} xHx^{-1}$.
  \item
    Show that, if $K$ is a normal subgroup of $G$ contained in $H$,
    then $K \leq N$.  (Thus $N$ is the largest normal subgroup of $G$
    contained in $H$; $N$ is called the core of $H$ in $G$).
  \end{enumerate}
\end{exercise}
\begin{solution} 
  \begin{enumerate}[(a)]
  \item
    Assuming the kernel of the action is $ N = \{ g \in G : (gH)(aH) = (aH) \ \textrm{ for all cosets } aH \}$,
    we need to verify it is non-empty.  $1 \in N$ as $1 (aH) = 1aH = aH$.

    Next we need to verify it is closed under the operation,
    given $n_1, n_2 \in N$, need to check
    \begin{align*}
      (n_1 H) (aH) &= (aH) \\
      (n_2 H) (aH) & = (aH) \\
      (n_1 n_2 H) (aH) & = (n_1H)(n_2 H) (a H) \ \textrm { by defintion } \\
      & = (n_1 H) (a H) \ \textrm{ since } n_2 \in N\\
      & = (a H) \  \textrm{ since } n_1 \in N  \\
    \end{align*}

    And closed under inverses:  $g \in N$.  So $(gH)(aH) = (aH)$.  

    So $(g^{-1} H ) ( gH) = (g^{-1} g H) = (H)$ so $g^{-1} H$ = $1H$ so $g^{-1} \in N$.  
    

    Now we have to show it is contained in $H$.  Let $n \in N$.  Then $n \cdot 1 = 1H$, that is
    $ n 1 h_1 = h_2$, for some $h_1, h_2 \in H$, or $n = h_2 h_1^{-1}$ and $n \in H$.  
    
  \item
    $N \subseteq \cap _{x \in G} xHx^{-1}$ direction:

    $n \in N$ means $n g h_1 = g h_2$, for some $h_1, h_2 \in H$ ($nH$ is identity in $G/H$).
    So

    \[
    n (g h_1) (g h_1)^{-1} = g h_2 g (g h_1)^{-1}
    \]

    and

    \[
    n = g h_2 h_1 ^{-1} g ^ {-1}
    \]

    Or

    \[
    n = g h_3 g ^ {-1}
    \]

    That is $ n \in g H g ^ {-1}$.  That was for any $g \in G$ so $n$ is in the intersection for all $g \in G$:

    \[
    n \in \cap _{x \in G} xHx^{-1}
    \]

    For the $\cap _{x \in G} xHx^{-1} \subseteq N$ direction:

    We have a member of the intersection $g$, for all $x \in G$, $g \in x H x ^ {-1}$.

    So for each $x$, So there is an $h \in H$ such that $ n = x h x ^ {-1}$.  So $nx = xh$ so

    $nxh^{-1}  = x $, but this is a way of saying $n x H = xH$ which is just $nH xH = xH$ i.e.

    $nH = 1$ in $G/H$.

    
  \item

    Let $K$ be normal in $G$ and contained in $H$.

    So for all $g \in G$, $g^{-1} K g \subseteq K$.  Let $aH$ be a left-coset of $G$.  Let $k \in K$.

    $(kH)(aH) = (kaH)$

    Then $ a^{-1}ka = k'$ for some $k'$ by normality, so $ka = ak'$ for some $k \in K \subseteq H$, so
    $(kaH) = aH$, i.e. $k \cdot  a = a$ so $ k $ is in the kernel of the action, and hence in $N$.

    \qed    

  \end{enumerate}
    
\end{solution}

\begin{exercise}
  Suppose $G$ is a group that has no normal subgroups except $\{ 1 \}$ and $G$ itself (a group with this
  property is called \textbf{normal}. Show that if $G$ has a proper subgroup $H$ with index $n$,
  then $G$ is isomorphic to a subgroup of $S_n$. (Hint: use part (A) of the previous.)  Then show that if $G$
  is infinite, then every proper subgroup $H$ of $G$ has infinite index. 
\end{exercise}
\begin{solution}
  The index $[G:H]$ is the number of left-cosets of $H$ in $G$.  Let
  $G$ act on these left cosets by multiplication.  For the purpose of
  this exercise, number the left cosets $(aH)$ as $a_1H$ through $a_nH$;
  identify $a_1$ with the coset $H$.  Now, we construct a homomorphism
  $\phi$ from $g \in G$ via this group action to $S_n$. Given a $g \in G$,
  let $\sigma = \phi(g)$ be defined:

  \[
  \sigma(i) = j, \ \textrm{ where }  a_j = g \cdot a_i
  \]

  We just need to check that $\phi(x y) = \phi(x) \phi(y)$

  \[
  \phi(x y)(i) = (x y) (a_iH) = x ( y a_i H) = x (a_j H) = a_k H 
  \]
  where $y$ acts on $a_iH$ to $a_jH$ and $x$ acts on $a_jH $ to $a_k H$.

  And
  \[
  \phi(x) \cdot \phi(y)(i) = \phi(x) \cdot (y a_i H) = \phi(x) (a_j H) = (a_k H)
  \]

  So $\phi$ preserves the group operation and is a homomorphism.  Now, what is $\ker(\phi)$?

  The kernel will be normal in $G$ so it is either $\{ 1 \}$ or $G$.

  If $\ker(\phi) = G$, then $\phi(g) = 1$, for all $g \in G$.  But that means
  $gH = 1H$, for all $g \in G$.  i.e. $g \in H$ for all $g \in G$ which contradicts
  $H$ being a proper subgroup of $G$.  So $\ker(\phi) = \{ 1 \}$, which means that
  $\phi$ is 1-1, proving it is an isomorphism of $G$ into $S_n$.  \qed


  In the case that $G$ is infinite, if a proper subgroup were to have finite index, the above would
  show that that $G$ would be a subgroup of a finite group, $S_n$, in contradiction; hence all
  proper subgroups have infinite index.

  \qed
\end{solution}

\begin{exercise}
  Let $G(x)$ be the stabilizer of $x$ under a group action (of group $G$ on a set $X$.) Show
  that stabilizers of elements in the orbit of $x$ are conjugate subgroups. Explicitly, for
  every $g \in G$ and $x \in X$ we have $G(g \cdot x) = gG(x)g^{-1}$.
\end{exercise}
\begin{solution}

  \[ 
  G(x) = \{ a \in G : a \cdot x = x \}
  \]

  And
  
  \begin{align*}
    G(g \cdot x) &= \{ a \in G : a \cdot g \cdot x = g \cdot x \} \\
    &= \{ a \in G : ag \cdot x = g \cdot x \} \\
    &= \{ a \in G : g^{-1}ag \cdot x = g^{-1}g \cdot x \} \\
    &= \{ a \in G : g^{-1}ag \cdot x = 1 \cdot x \} \\
    &= \{ a \in G : g^{-1}ag \cdot x = x \} \\
    &= \{ b = ga g^{-1}  \in G : b \cdot x = x \} \\
    &= g G(x) g^{-1}
  \end{align*}

  
\end{solution}

\begin{exercise}
  Let $G = D_8$ be the dihedral group of symmetries of the square.
  \begin{enumerate}[(a)]
    \item 
      What is the stabilizer of a vertex?  of an edge?
    \item
      $G$ acts on the set of diagonals of the square.  What is the stabilizer of the diagonal?
  \end{enumerate}
\end{exercise}
\begin{solution}
  \begin{enumerate}[(a)]
    \item 
      The stabilizer of a vertex is the flips around the axis that
      goes between the vertex and the center.  For example, for the
      vertex $(A)$ (using the conventions from HW2, Exercise 1), the
      stabilizer subgroup of $A$ is $\{E \textrm{ the identity }, F \}$,

      The stabilizer of an edge is the identity and the flip between
      the mid-point of the edge and the center, which is a sequence of
      the base flip and a rotation in our representation, e,g, the
      $AB$ edge is preserved by a flip and then one rotation, moving
      $A$ to $B$ and $B$ to $A$.  There are thus 2 members of the
      stabilizer for a given edge.
    \item
      Each diagonal is fixed by four movements: the identity, flipping
      around that diagonal, flipping around the other diagonal (which
      reverses the vertices of our given diagonal but preserves the
      diagonal) and rotating $180^{\circ}$.  
  \end{enumerate}
  
\end{solution}


\begin{exercise}
  Use the orbit stabilizer theorem to determine the orders of the
  group of rotational symmetries of a cube and a tetrahedron.  
\end{exercise}
\begin{solution}
  For a cube, let the group of rotational movements act on a face.
  There are $6$ faces, so the orbit of a face is order $6$.  Each face is
  left unchanged by $4$ rotations, so the over all order of the group is $24$.

  For a tetrahedron, let the group of rotational movements also act on
  a face.  There are $4$ faces, so the orbit of a face is order $4$.
  Each face is left unchanged merely by the three plane rotations of
  $120 ^{\circ}$, so the order of the rotational groupis $4 \times 3$
  or $12$.
\end{solution}

\begin{exercise}
  Assume that two colorings of the vertices of a square are equivalent
  if one can be mapped into the other by a permutation in the dihedral
  group $G=D_8$.  If $n$ colors are available,
  find the number of distinct colorings of the vertices of a square.
\end{exercise}
\begin{solution}
  So our group here is $D_8$ and the set $S$ is, for each vertex, $a, b, c, d, e, f, g, h$, 
  the $n$ members with $n$ different colors, so $a_1, a_2, ... , a_n, b_1, ... , h_n$.

  Then the number of distinct orbits, which correspond
  to colorings upto the symmetry of $D_8$, is given by Burnside's lemma as:

  \[
  |X / G| = \frac{1}{|G|} \sum \limits _{g \in G} | X^g|
  \]
  where $X/G$ is the set of all orbits, and
  $ X ^g = \{ x \in X : g \cdot x = x \}$.

  Analyzing the various orbits of $D_8$, we have 8 cases to consider

  $ g = 1$, the identity; this fixes all the vertices, so there are $n^4$ choices.  

  $ g = R = (1 2 3 4)$.  All vertices must be the same color, so there
  are $n$ choices.

  $ g = R^2 = (1 3) (2 4)$.  The first cycle can have $n$ choices as can the
  second cycle, so there are $n^2 $ choices.

  $ g = R^3 = (1 4 3 2)$.  Like $R$ all vertices must be the same color, so there
  are $n$ choices.

  $ g = F = (1) (2 4) (3) $.  There are $n^3$ choices with 3 cycles.

  $ g = RF = (1 2) ( 3 4)$.  The are another $n^2$ choices.

  $ g = R^2F = (1 3) (2 ) (4)$, so another $n^3$ choices with the 3 cycles.

  $ g = R^3 F = (1 4) (2 3)$, so $n^2$ choices for the 2 cycles.

  So

  \[
  |X / G| = \frac{1}{8} \left( n^4 + n + n^2 + n + n^3 + n^2 + n^ 3 + n^2 \right)
  \]

  Simplifying a bit:  
  
  \[
  |X / G| = \frac{1}{8} \left( n^4 + 2n^3  + 3n^2 + 2n  \right)
  \]

\end{solution}

\begin{exercise}
  How many distinct colorings of the vertices of a regular hexagon are
  there if we are forced to color exactly three vertices blue and
  three vertices yellow?  Assume that two colorings of the vertices of
  a hexagon are equivalent if one can be mapped into the other by a
  permutation in the dihedral group $G = D_{2 \cdot 6} = D_{12}$.
\end{exercise}
\begin{solution}

  Our group here is $D_{12}$, of order 12, and our set is permutations
  of the vertices that have three blue vertices and three yellow
  vertices.

  Taking the enumeration of $D_{12}$ from the class notes,

  $g = 1 = (1) (2) (3) (4) (5) (6)$, each vertex can be colored
  independently, but there's only ${6 \choose 3}$ or $20$ ways.

  $g = F = (2 6) (3 5) (1) (4)$.  This gives $4$ options (1 2-cylce and 1 1-cycle for blue)

  $g = R = (1 2 3 4 5 6)$.  This one cannot meet the requirements, so contributes $0$.

  $g = R^2 = (1 3 5)(2 4 6)$.  There are just $2$ options here.

  $g = R^3 = (1 4) ( 2 5) ( 3 6)$ also contributes $0$.

  $g = R^4 = (1 5 3) (2 6 4) $ like the conjugate $R^2$ contributes $2$.

  $g = R^5 = (1 6 5 4 3 2) $ cannot meet the requirements, like $R$< and contributes $0$.

  $g = RF = (1 2 ) ( 3 6) (4 5) $ cannot meet the requirements, like $R^3$ and contributes $0$.

  $g = R^2F = (1 3) (2 ) (4 6) (5) $ like $F$ contributes $4$.

  $g = R^3F = (1 4 ) (2 3) (5 6) $ cannot meet the requirements, like $R^3$ and contributes $0$.

  $g = R^4F = (1 5) (2 4) (3) (6) $ like $F$ contributes $4$.
  
  $g = R^5F = (1 2 ) ( 3 6) (4 5) $ cannot meet the requirements, like $R^3$ and contributes $0$.
  

  So the formula works out to:

  \[ | S / G | = \frac{1}{12} ( 20 + 4 + 0 + 2 + 0 + 2 + 0 + 0 + 4 + 0 + 4 + 0 ) = \frac{36}{12} = 3
  \]

  
  
\end{solution}

\begin{exercise}
  The vertices of a regular $p$-gon, where $p$ is an odd prime, are to
  be painted using $n$ colors.  If the $p$-gon can be rotated by not
  flipped, find the number of distinct colorings.
\end{exercise}
\begin{solution}
  $|G| = p$, just generated by the smallest $\frac{2 \pi }{p}$ rotation.

  For the identity element, $1$, there are $n^p$ possible colorings.

  For any other element, which are all structurally similar (identical under automorphisms of $G$ to $G$),
  There are only $n$ choices of coloring, assigning the same color to all vertices.

  So the formula works out to be:

  \[ | S / G | = \frac{1}{p} ( n^p + (p - 1) n)
  \]

  
\end{solution}

\begin{comment}
  \begin{exercise}
    problem
  \end{exercise}
  \begin{solution}
    \Begin{enumerate}[(a)]
    \item
      first answer
    \end{enumerate}
  \end{solution}
\end{comment}

\end{document}
