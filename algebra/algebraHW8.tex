\documentclass[11pt,oneside]{article}
\usepackage[hmargin=1in,vmargin=1in]{geometry}               % See geometry.pdf to learn the layout options. There are lots.
\geometry{letterpaper}                   % ... or a4paper or a5paper or ...
%\geometry{landscape}                % Activate for for rotated page geometry
%\usepackage[parfill]{parskip}    % Activate to begin paragraphs with an empty line rather than an indent
\usepackage{graphicx}
\usepackage{datetime}
\usepackage{amssymb}
\usepackage{epstopdf}
\usepackage{url}
%\usepackage{verbatim}
\usepackage{comment}
\specialcomment{solution}{\textbf{Solution. }}{}
%\excludecomment{solution}    %uncomment to remove solutions.

%\usepackage{enumerate}

%Use the enumitem package instead of enumerate
\usepackage[shortlabels]{enumitem}
%\usepackage{enumitem}
%then it will support the same suntax as the enumerate package.
%The enumerate package does not provide any extra configurations other than the label.

%\setlist[enumerate]{topsep=0pt,itemsep=-1ex,partopsep=1ex,parsep=1ex}
\setlist[enumerate]{topsep=0pt,partopsep=0pt}

\DeclareGraphicsRule{.tif}{png}{.png}{`convert #1 `dirname #1`/`basename #1 .tif`.png}
\usepackage{amsmath,amsthm,amscd,amssymb}
\usepackage{latexsym}
\usepackage[colorlinks,citecolor=red,pagebackref,hypertexnames=false]{hyperref}
\numberwithin{equation}{section}

\theoremstyle{definition}
\newtheorem{exercise}{Exercise}
%\newtheorem{solution}{Solution}
\newtheorem*{defn}{Definition}
\newtheorem*{claim}{Claim}

\def\pmod{+'}
\def\boldr{\boldsymbol{r}}
\def\boldc{\boldsymbol{c}}

\def\calA{\mathcal{A}}
\def\calB{\mathcal{B}}
\def\calC{\mathcal{C}}
\def\calT{\mathcal{T}}
\def\OR{\overline{\mathbb{R}}}
\def\RR{\mathbb{R}}
\def\CC{\mathbb{C}}
\def\FF{\mathbb{F}}
\def\QQ{\mathbb{Q}}
\def\ZZ{\mathbb{Z}}
\def\NN{\mathbb{N}}
%\def\NN{\mathbb{Z}_{> 0}}
\def\Nzero{\mathbb{Z}_{\geq 0}}
\def\EE{\mathbb{E}}
\def\PP{\mathbb{P}}
\def\supp{\mathrm{supp}}
\def\diam{\mathrm{diam}}
\def\sp{\mathrm{span}}
\def\ker{\mathrm{ker}}
\def\Aut{\operatorname{Aut}}
\def\Inn{\operatorname{Inn}}
\def\Ker{\operatorname{Ker}}
%\def\sp{\mathrm{span}} %messes up align enviroment
\newcommand{\Mod}{\ (\mathrm{mod}\ )}
\newcommand{\rbr}[1]{\left( {#1} \right)}
\newcommand{\sbr}[1]{\left[ {#1} \right]}
\newcommand{\cbr}[1]{\left\{ {#1} \right\}}
\newcommand{\abr}[1]{\left\langle {#1} \right\rangle}
\newcommand{\abs}[1]{\left| {#1} \right|}
\newcommand{\norm}[1]{\left\|#1\right\|}
\def\one{\mathbf{1}}
\DeclareMathOperator*{\esssup}{ess\,sup}
\newcommand*\wc{{}\cdot{}}
%\newcommand*\wc{ \, \cdot \,}
%wc for wildcard
\renewcommand{\Re}{\operatorname{Re}}
\renewcommand{\Im}{\operatorname{Im}}
\newcommand{\sgn}{\textup{sgn\,}}


\setlength{\parindent}{0pt}
\setlength{\parskip}{11pt}

%\title{\parbox{14cm}{\centering{  Interior points of circle and sphere packings}}}
\begin{document}

\textbf{HW 7 - MATH 221A - Fall 2023 - Chris Lane}

Date: \hhmmsstime{} \ \today \ \ Git hash: 
\input{/Users/chlane/src/jayalane/2023H2HW/.git/refs/heads/main}

\begin{exercise}
  Let $G$ act on left cosets of $H$ by multiplication.
  \begin{enumerate}[(a)]
  \item
    Show that the kernel of the action is a subgroup of $H$.  
  \item
    Show that the kernel of the action is $N = \cap _{x \in G} xHx^{-1}$.
  \item
    Show that, if $K$ is a normal subgroup of $G$ contained in $H$,
    then $K \leq N$.  (Thus $N$ is the largest normal subgroup of $G$
    contained in $H$; $N$ is called the core of $H$ in $G$).
  \end{enumerate}
\end{exercise}
\begin{solution} 
  \begin{enumerate}[(a)]
  \item
    Assuming the kernel of the action is $ N = \{ g \in G : g(aH) = H \ \textrm{ for all cosets } aH \}$,
    we need to verify it is non-empty.  $1 \in N$ as $1 (aH) = 1aH = aH$.

    Next we need to verify it is closed under the operation,
    given $g_1, g_2 \in N$, $g_1(aH) = H$.  So $g_1 \cdot g_2 (aH) = g_1 (g_2aH) = g_1 (aH) = g_1 a H  = a H$.

    And closed under inverses:  $g \in N$.  So for all $a \in G$, there are $h_1, h_2 \in H$ such
    that $g a h_1 = h_2$.  Now, $1 \in N$, so $ 1 a h_2 = h_2$.  So $g^ {-1} g a h_2 = h_2$, and
    $ g ^ {-1} g a h_1 h_1^{-1} h_2 = h_2$ or $ g ^ {-1}$

    TODO
    
  \item
    TODO
  \item
    TODO
  \end{enumerate}
    
\end{solution}

\begin{exercise}
  Suppose $G$ is a group that has no normal subgroups except $\{ 1 \}$ and $G$ itself (a group with this
  property is called \textbf{normal}. Show that if $G$ has a proper subgroup $H$ with index $n$,
  then $G$ is isomorphic to a subgroup of $S_n$. (Hint: use part (A) of the previous.)  Then show that if $G$
  is infinite, then every proper subgroup $H$ of $G$ has infinite index. 
\end{exercise}
\begin{solution}
  The index $[G:H]$ is the number of left-cosets of $H$ in $G$.  Let
  $G$ act on these left cosets by multiplication.  For the purpose of
  this exercise, number the left cosets $(aH)$ as $a_1H$ through $a_nH$;
  identify $a_1$ with the coset $H$.  Now, we construct a homomorphism
  $\phi$ from $g \in G$ via this group action to $S_n$. Given a $g \in G$,
  let $\sigma = \phi(g)$ be defined:

  \[
  \sigma(i) = j, \ \textrm{ where }  a_j = g \cdot a_i
  \]

  We just need to check that $\phi(x y) = \phi(x) \phi(y)$

  \[
  \phi(x y)(i) = (x y) (a_iH) = x ( y a_i H) = x (a_j H) = a_k H 
  \]
  where $y$ acts on $a_iH$ to $a_jH$ and $x$ acts on $a_jH $ to $a_k H$.

  And
  \[
  \phi(x) \cdot \phi(y)(i) = \phi(x) \cdot (y a_i H) = \phi(x) (a_j H) = (a_k H)
  \]

  So $\phi$ preserves the group operation and is a homomorphism.  Now, what is $\ker(\phi)$?

  The kernel will be normal in $G$ so it is either $\{ 1 \}$ or $G$.

  If $\ker(\phi) = G$, then $\phi(g) = 1$, for all $g \in G$.  But that means
  $gH = 1H$, for all $g \in G$.  i.e. $g \in H$ for all $g \in G$ which contradicts
  $H$ being a proper subgroup of $G$.  So $\ker(\phi) = \{ 1 \}$, which means that
  $\phi$ is 1-1, proving it is an isomorphism of $G$ into $S_n$.  \qed


  In the case that $G$ is infinite, if a proper subgroup were to have finite index, the above would
  show that that $G$ would be a subgroup of a finite group, $S_n$, in contradiction; hence all
  proper subgroups have infinite index.

  \qed
\end{solution}

\begin{exercise}
  Let $G(x)$ be the stabilizer of $x$ under a group action (of group $G$ on a set $X$.) Show
  that stabilizers of elements in the orbit of $x$ are conjugate subgroups. Explicitly, for
  every $g \in G$ and $x \in X$ we have $G(g \cdot x) = gG(x)g^{-1}$.
\end{exercise}
\begin{solution}
  Let $b \in G(x)$.  Then, $ b \cdot x = x$.  So $ b \cdot g^{-1} \cdot x = g^{-1} \cdot x$
  and $g \cdot b \cdot g ^ { -1} \cdot x = g \cdot g ^ { -1 } \cdot x$.

  That is,

  \[
  (gbg^{-1} \cdot x = g g^{-1} \cdot x
  \]

  Or TODO
  
\end{solution}

\begin{exercise}
  Let $G = D_8$ be the dihedral group of symmetries of the square.
  \begin{enumerate}[(a)]
    \item 
      What is the stabilizer of a vertex?  of an edge?
    \item
      $G$ acts on the set of diagonals of the square.  What is the stabilizer of the diagonal?
  \end{enumerate}
\end{exercise}
\begin{solution}
  \begin{enumerate}[(a)]
    \item 
      The stabilizer of a vertex is the flips around the axis that goes between the vertex and the center.  For example, for the vertex $(1)$,
      
    \item
      $G$ acts on the set of diagonals of the square.  What is the stabilizer of the diagonal?
  \end{enumerate}
  
\end{solution}

\begin{comment}
  \begin{exercise}
    problem
  \end{exercise}
  \begin{solution}
    \Begin{enumerate}[(a)]
    \item
      first answer
    \end{enumerate}
  \end{solution}
\end{comment}

\end{document}
