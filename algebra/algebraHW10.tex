\documentclass[11pt,oneside]{article}
\usepackage[hmargin=1in,vmargin=1in]{geometry}               % See geometry.pdf to learn the layout options. There are lots.
\geometry{letterpaper}                   % ... or a4paper or a5paper or ...
%\geometry{landscape}                % Activate for for rotated page geometry
%\usepackage[parfill]{parskip}    % Activate to begin paragraphs with an empty line rather than an indent
\usepackage{graphicx}
\usepackage{datetime}
\usepackage{amssymb}
\usepackage{epstopdf}
\usepackage{url}
%\usepackage{verbatim}
\usepackage{comment}
\specialcomment{solution}{\textbf{Solution. }}{}
%\excludecomment{solution}    %uncomment to remove solutions.

%\usepackage{enumerate}

%Use the enumitem package instead of enumerate
\usepackage[shortlabels]{enumitem}
%\usepackage{enumitem}
%then it will support the same suntax as the enumerate package.
%The enumerate package does not provide any extra configurations other than the label.

%\setlist[enumerate]{topsep=0pt,itemsep=-1ex,partopsep=1ex,parsep=1ex}
\setlist[enumerate]{topsep=0pt,partopsep=0pt}

\DeclareGraphicsRule{.tif}{png}{.png}{`convert #1 `dirname #1`/`basename #1 .tif`.png}
\usepackage{amsmath,amsthm,amscd,amssymb}
\usepackage{latexsym}
\usepackage[colorlinks,citecolor=red,pagebackref,hypertexnames=false]{hyperref}
\numberwithin{equation}{section}

\theoremstyle{definition}
\newtheorem{exercise}{Exercise}
\newtheorem{lemma}{Lemma}
\newtheorem{theorem}{Theorem}
%\newtheorem{solution}{Solution}
\newtheorem*{defn}{Definition}
\newtheorem*{claim}{Claim}

\def\pmod{+'}
\def\boldr{\boldsymbol{r}}
\def\boldc{\boldsymbol{c}}

\def\calA{\mathcal{A}}
\def\calB{\mathcal{B}}
\def\calC{\mathcal{C}}
\def\calT{\mathcal{T}}
\def\OR{\overline{\mathbb{R}}}
\def\RR{\mathbb{R}}
\def\CC{\mathbb{C}}
\def\FF{\mathbb{F}}
\def\QQ{\mathbb{Q}}
\def\ZZ{\mathbb{Z}}
\def\NN{\mathbb{N}}
%\def\NN{\mathbb{Z}_{> 0}}
\def\Nzero{\mathbb{Z}_{\geq 0}}
\def\EE{\mathbb{E}}
\def\PP{\mathbb{P}}
\def\supp{\mathrm{supp}}
\def\diam{\mathrm{diam}}
\def\sp{\mathrm{span}}
\def\ker{\mathrm{ker}}
\def\Aut{\operatorname{Aut}}
\def\Inn{\operatorname{Inn}}
\def\Ker{\operatorname{Ker}}
%\def\sp{\mathrm{span}} %messes up align enviroment
\newcommand{\Mod}{\ (\mathrm{mod}\ )}
\newcommand{\rbr}[1]{\left( {#1} \right)}
\newcommand{\sbr}[1]{\left[ {#1} \right]}
\newcommand{\cbr}[1]{\left\{ {#1} \right\}}
\newcommand{\abr}[1]{\left\langle {#1} \right\rangle}
\newcommand{\abs}[1]{\left| {#1} \right|}
\newcommand{\norm}[1]{\left\|#1\right\|}
\def\one{\mathbf{1}}
\DeclareMathOperator*{\esssup}{ess\,sup}
\newcommand*\wc{{}\cdot{}}
%\newcommand*\wc{ \, \cdot \,}
%wc for wildcard
\renewcommand{\Re}{\operatorname{Re}}
\renewcommand{\Im}{\operatorname{Im}}
\newcommand{\sgn}{\textup{sgn\,}}


\setlength{\parindent}{0pt}
\setlength{\parskip}{11pt}

%\title{\parbox{14cm}{\centering{  Interior points of circle and sphere packings}}}
\begin{document}

\textbf{HW 10 - MATH 221A - Fall 2023 - Chris Lane}

Date: \hhmmsstime{} \ \today \ \ Git hash: 
\input{/Users/chlane/src/jayalane/2023H2HW/.git/refs/heads/main}

\begin{exercise}
  \begin{enumerate}[(a)]
  \item
    Give an example of a group that has no composition series.  Justify.
  \item
    Given an example of two non-isomorphic groups that have composition series with the same
    composition factors (up to rearrangement).  Justify.  
  \end{enumerate}
\end{exercise}
  
\begin{solution}
  \begin{enumerate}[(a)]
  \item
    According to an in class theorem, ``Every finite group $G$ has a composition series'' so
    it must be an infinite group.  Letting
    \[
      \{1\} = G_0 \trianglelefteq G_1 \hdots  \trianglelefteq G_n = \ZZ
    \]

    The only finite subgroup of $\ZZ$ is $\{ 1 \}$, so the last link
    $G_0 \trianglelefteq G_1$ has to be between $\{ 1 \}$ and an
  infinite group, which has to be isomorphic, as all subsets of $\ZZ$
  are, to $\ZZ$ itself; the factor group then is $\ZZ$ which is not
  simple.

  So there is no composition series for $\ZZ$.

  In fact, if you have a finite subnormal series for $\ZZ$ you can
  refine it for ever; each subgroup of $\ZZ$ is just $n\ZZ$ for some
  $n$.  If you have a set $k$ subgroups, take the next prime larger
  than $ q = \prod \limits _ {i=1} ^ k n_i \\$; then $ p q
  \ZZ$ will be a (normal due to Abelianness) subgroup of $G_1$ and
  will refine the sequence.
    
\item
  The smallest two groups I'm aware of that are of the same order
  and not isomorphic are $\ZZ_4$ and the Klein four group, $\ZZ_2 \times \ZZ_2$.  In both cases,
  \[
    \{ 1 \} \trianglelefteq \ZZ_2 \trianglelefteq G
  \]

  is a composition series, with the factor groups being $\ZZ_2$ for
  both segments of the series.
    
  \end{enumerate}
\end{solution}

\begin{exercise}
  Show that every finite $p-$group is solvable.
\end{exercise}

\begin{solution}
  A solvable group is one with a composition series in which all the
  factor groups are Abelian.  First, we'll note that all groups of
  order $p^2$ for some prime $p$ are Abelian. Then we'll proceed on
  induction on the power of $p$ in the group order.  (All $p-$groups
  have order equal to $p^n$ for some $n$).
  
  So, given a group $G$ of order $p^n$, we first note that its
  center $Z(G)$ is non-trivial, of order $p^k$, $0 < k \leq n$.
  
  If $k = n$, then $G$ is abelian and therefore solvable (as
  whatever normal subgroups it has will have Abelian factors; or the
  derived subgroup will be the identity since
  $a b a^{-1} b^{-1} = 1$ for all $a, b$ in a commutitive group).
  
  So, looking at $Z(G)$ where $|Z(G)| = p^k$, $k < n$, $Z(K)$ is
  normal in $G$.  $Z(G)$ is a $p-$group of order lower than $p^n$ so
  is solvable by assumption.  Also, $G / Z(G)$ is a $p-$group of
  order $p ^ { n - k}$, and therefore solvable by assumption. So by
  a theorem in class, since both the normal subgroup $Z(G)$ and the
  factor group $G/Z(G)$ are solvable, $G$ itself is solvable.
  
\end{solution}

\begin{exercise}
  \begin{enumerate}[(a)]
  \item
    Show that the groups $A_n$ and $S_n$ are solvable for $1 \leq n \leq 4$
  \item
    Show that if $G$ is a non-abelian simple group, then $G' = G$ and so $G$
    is not solvable.  
  \item
    In class, we will prove that $A_n$ is simple for $n \geq 5$.  Using this result, show that
    $A_n$ and $S_n$ are not solvable for $n \geq 5$. 
  \end{enumerate}
\end{exercise}
\begin{solution}
  \begin{enumerate}[(a)]
  \item
    a
  \item
    b
  \item
    c
  \end{enumerate}
\end{solution}

\begin{exercise}
  Prove:
  
  \begin{enumerate}[(a)]
  \item
    If $G$ has a composition series, then $G$ is solvable if and only if the composition factors are
    cyclic of prime order.
  \item
    If $G$ has a composition series and $G$ is solvable, then $G$ is finite.  
  \end{enumerate}
\end{exercise}

\begin{solution}
  TODO
\end{solution}

\begin{exercise}
  Let $G = D_6$.  Show that $G$ has a normal subgroup $N$ such that $N$ and $G/N$ are abelian,
  but $G$ is not abelian.  
\end{exercise}
\begin{solution}
  TODO
\end{solution}

\begin{comment}
  \begin{exercise}
    problem
  \end{exercise}
  \begin{solution}
    \begin{enumerate}[(a)]
    \item
      first answer
    \end{enumerate}
  \end{solution}
\end{comment}


\end{document}
