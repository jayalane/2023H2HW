\documentclass[11pt,oneside]{article}
\usepackage[hmargin=1in,vmargin=1in]{geometry}               % See geometry.pdf to learn the layout options. There are lots.
\geometry{letterpaper}                   % ... or a4paper or a5paper or ...
%\geometry{landscape}                % Activate for for rotated page geometry
%\usepackage[parfill]{parskip}    % Activate to begin paragraphs with an empty line rather than an indent
\usepackage{graphicx}
\usepackage{datetime}
\usepackage{amssymb}
\usepackage{epstopdf}
\usepackage{url}
%\usepackage{verbatim}
\usepackage{comment}
\specialcomment{solution}{\textbf{Solution. }}{}
%\excludecomment{solution}    %uncomment to remove solutions.

%\usepackage{enumerate}

%Use the enumitem package instead of enumerate
\usepackage[shortlabels]{enumitem}
%\usepackage{enumitem}
%then it will support the same suntax as the enumerate package.
%The enumerate package does not provide any extra configurations other than the label.

%\setlist[enumerate]{topsep=0pt,itemsep=-1ex,partopsep=1ex,parsep=1ex}
\setlist[enumerate]{topsep=0pt,partopsep=0pt}

\DeclareGraphicsRule{.tif}{png}{.png}{`convert #1 `dirname #1`/`basename #1 .tif`.png}
\usepackage{amsmath,amsthm,amscd,amssymb}
\usepackage{latexsym}
\usepackage[colorlinks,citecolor=red,pagebackref,hypertexnames=false]{hyperref}
\numberwithin{equation}{section}

\theoremstyle{definition}
\newtheorem{exercise}{Exercise}
\newtheorem{lemma}{Lemma}
\newtheorem{theorem}{Theorem}
%\newtheorem{solution}{Solution}
\newtheorem*{defn}{Definition}
\newtheorem*{claim}{Claim}

\def\pmod{+'}
\def\boldr{\boldsymbol{r}}
\def\boldc{\boldsymbol{c}}

\def\calA{\mathcal{A}}
\def\calB{\mathcal{B}}
\def\calC{\mathcal{C}}
\def\calT{\mathcal{T}}
\def\OR{\overline{\mathbb{R}}}
\def\RR{\mathbb{R}}
\def\CC{\mathbb{C}}
\def\FF{\mathbb{F}}
\def\QQ{\mathbb{Q}}
\def\ZZ{\mathbb{Z}}
\def\NN{\mathbb{N}}
%\def\NN{\mathbb{Z}_{> 0}}
\def\Nzero{\mathbb{Z}_{\geq 0}}
\def\EE{\mathbb{E}}
\def\PP{\mathbb{P}}
\def\supp{\mathrm{supp}}
\def\diam{\mathrm{diam}}
\def\sp{\mathrm{span}}
\def\ker{\mathrm{ker}}
\def\Aut{\operatorname{Aut}}
\def\Inn{\operatorname{Inn}}
\def\Ker{\operatorname{Ker}}
%\def\sp{\mathrm{span}} %messes up align enviroment
\newcommand{\Mod}{\ (\mathrm{mod}\ )}
\newcommand{\rbr}[1]{\left( {#1} \right)}
\newcommand{\sbr}[1]{\left[ {#1} \right]}
\newcommand{\cbr}[1]{\left\{ {#1} \right\}}
\newcommand{\abr}[1]{\left\langle {#1} \right\rangle}
\newcommand{\abs}[1]{\left| {#1} \right|}
\newcommand{\norm}[1]{\left\|#1\right\|}
\def\one{\mathbf{1}}
\DeclareMathOperator*{\esssup}{ess\,sup}
\newcommand*\wc{{}\cdot{}}
%\newcommand*\wc{ \, \cdot \,}
%wc for wildcard
\renewcommand{\Re}{\operatorname{Re}}
\renewcommand{\Im}{\operatorname{Im}}
\newcommand{\sgn}{\textup{sgn\,}}


\setlength{\parindent}{0pt}
\setlength{\parskip}{11pt}

%\title{\parbox{14cm}{\centering{  Interior points of circle and sphere packings}}}
\begin{document}

\textbf{HW 10 - MATH 221A - Fall 2023 - Chris Lane}

Date: \hhmmsstime{} \ \today \ \ Git hash: 
\input{/Users/chlane/src/jayalane/2023H2HW/.git/refs/heads/main}

\begin{exercise}
  \begin{enumerate}[(a)]
  \item
    Give an example of a group that has no composition series.  Justify.
  \item
    Given an example of two non-isomorphic groups that have composition series with the same
    composition factors (up to rearrangement).  Justify.  
  \end{enumerate}
\end{exercise}
  
\begin{solution}
  \begin{enumerate}[(a)]
  \item
    According to an in class theorem, ``Every finite group $G$ has a composition series'' so
    it must be an infinite group.  Letting
    \[
      \{1\} = G_0 \trianglelefteq G_1 \hdots  \trianglelefteq G_n = \ZZ
    \]

    The only finite subgroup of $\ZZ$ is $\{ 1 \}$, so the last link
    $G_0 \trianglelefteq G_1$ has to be between $\{ 1 \}$ and an
  infinite group, which has to be isomorphic, as all subgroups of $\ZZ$
  are, to $\ZZ$ itself; the factor group then is $\ZZ$ which is not
  simple.

  So there is no composition series for $\ZZ$.

  In fact, if you have a finite subnormal series for $\ZZ$ you can
  refine it for ever; each subgroup of $\ZZ$ is just $n\ZZ$ for some
  $n$.  If you have a set $k$ subgroups, take the next prime larger
  than $ q = \prod \limits _ {i=1} ^ k n_i $

  Then $ p q \ZZ$ will be a (normal due to Abelianness) subgroup of $G_1$ and
  will refine the sequence.
    
\item
  The smallest two groups I'm aware of that are of the same order
  and not isomorphic are $\ZZ_4$ and the Klein four group, $\ZZ_2 \times \ZZ_2$.  In both cases,
  \[
    \{ 1 \} \trianglelefteq \ZZ_2 \trianglelefteq G
  \]

  is a composition series, with the factor groups being $\ZZ_2$ for
  both segments of the series.
    
  \end{enumerate}
\end{solution}

\begin{exercise}
  Show that every finite $p-$group is solvable.
\end{exercise}

\begin{solution}
  A solvable group is one with a composition series in which all the
  factor groups are Abelian.  First, we'll note that all groups of
  order $p^2$ for some prime $p$ are Abelian. Then we'll proceed on
  induction on the power of $p$ in the group order.  (All $p-$groups
  have order equal to $p^n$ for some $n$).
  
  So, given a group $G$ of order $p^n$, we first note that its
  center $Z(G)$ is non-trivial, of order $p^k$, $0 < k \leq n$.
  
  If $k = n$, then $G$ is abelian and therefore solvable (as
  whatever normal subgroups it has will have Abelian factors; or the
  derived subgroup will be the identity since
  $a b a^{-1} b^{-1} = 1$ for all $a, b$ in a commutitive group).
  
  So, looking at $Z(G)$ where $|Z(G)| = p^k$, $k < n$, $Z(K)$ is
  normal in $G$.  $Z(G)$ is a $p-$group of order lower than $p^n$ so
  is solvable by assumption.  Also, $G / Z(G)$ is a $p-$group of
  order $p ^ { n - k}$, and therefore solvable by assumption. So by
  a theorem in class, since both the normal subgroup $Z(G)$ and the
  factor group $G/Z(G)$ are solvable, $G$ itself is solvable.
  
\end{solution}

\begin{exercise}
  \begin{enumerate}[(a)]
  \item
    Show that the groups $A_n$ and $S_n$ are solvable for $1 \leq n \leq 4$
  \item
    Show that if $G$ is a non-abelian simple group, then $G' = G$ and so $G$
    is not solvable.  
  \item
    In class, we will prove that $A_n$ is simple for $n \geq 5$.  Using this result, show that
    $A_n$ and $S_n$ are not solvable for $n \geq 5$. 
  \end{enumerate}
\end{exercise}
\begin{solution}
  \begin{enumerate}[(a)]
  \item
    $S_1 = A_1= \{1\}$ which is solvable.

    $S_2 = C_2$ which is solvable (one factor group of order two, which is abelian and simple.

    $A_2$ I guess is just $\{1\}$ as there are no even permutations of 2 items.

    $A_3 = C_3$ which is solvable (one factor group of order three, still abelian and simple).

    $S_3$ has $A_3$ as a normal subgroup, and the factor group is $\ZZ_2$.  $A_3$ is solvable and
    the factor group is of order two, simple and abelian.

    $A_4$ is solvable, as the Klein subgroup is normal (from homework
    4); the Klein subgroup is simple and abelian; the factor group is
    of order 3, and hence abelian and simple.

    $S_4$ has $A_4$ as a normal subgroup with factor group of order $2$, and so is solvable.  
  \item
    The derived group $G'$ of $G$ is normal in $G$.  Since $G$ is simple, the only
    normal subgroups are $\{1\}$ or $G$.  If the derived group were $\{1\}$, the group
    would be abelian (there would be no non-identity commutators to generate it with).
    So $G'$ must be $G$, and the sequence of derived groups will not reach $ \{ 1 \}$.  
  \item
    The key thing to prove is that $A_n$ is non-abelian for $n>4$.  This can be proven just using
    permutations on 5 symbols, which leave the other symbols unmoved. If $a = (1 2 3)$ and
    $b = (1 4 5)$, then $ab = (1 4 5 2 3)$ but $ba = (1 2 3 4 5)$ which are not equal.  Since
    $A_n$ is non abelian and simple, the derived group of $A_n$ is $A_n$, meaning the group is not solvable.

    $S_n$ is therefore not solvable as $A_n$ is normal in $S_n$ (as the inverse image of 0 in the
    ``number of cycles'' homomorphism from $S_n$ to $\ZZ_2$. 

  \end{enumerate}
\end{solution}

\begin{exercise}
  Prove:
  
  \begin{enumerate}[(a)]
  \item
    If $G$ has a composition series, then $G$ is solvable if and only if the composition factors are
    cyclic of prime order.
  \item
    If $G$ has a composition series and $G$ is solvable, then $G$ is finite.  
  \end{enumerate}
\end{exercise}

\begin{solution}
  \begin{enumerate}[(a)]
  \item
    If $G$ has a composition series, and the factors are all cyclic of prime order, then the factors are all abelian and $G$ is solvable by definition.

    If $G$ is solvable, then the factor groups are abelian, by a theorem in class.  The series of derived groups yields a
    composition series, in which the factor groups are all simple (or else it could be further refined).  But the only finite
    simple abelian groups are the cyclic groups of prime order.
  \item
    There is a result which I think we proved earlier that no infinite group is simple, (as it contains $\ZZ$ which has
    an infinite number of normal subgroups, and no composition series).  So if $G$ has a composition series, the factor
    groups at each step are finite; the initial group $\{1\}$ is finite, and the total order of $G$ is just the
    order of each factor group multiplied.  If the factor groups are all simple, then they are all finite, and the product, the
    order of $G$, is finite.
  \end{enumerate}
\end{solution}

\begin{exercise}
  Let $G = D_6$.  Show that $G$ has a normal subgroup $N$ such that $N$ and $G/N$ are abelian,
  but $G$ is not abelian.  
\end{exercise}
\begin{solution}
  First, $G$ is not abelian.  We'e shown this on homework before, I think, but easy to see:

  Writing the generators $R$ and $F$ as permutations, $R = (1 2 3 4 5 6)$ and $F = (2 6) ( 3 5) (1) 4)$, we can calculate:
  \[
  RF = (1 2 ) ( 3 6) ( 4 5)
  \]
  But
  \[
  FR = (1 6) ( 2 5) (3 4)
  \]

  Looking at the Sylow 2-subroups, the Sylow theorems give us
  (as $|G| = 12 = 2^2 3$)
  \[
  n_2 \equiv 1 \mod{3}
  n_2 | 3
  \]
  These imply $n_2 = 1$, making the Sylow 2-subgroup, N,  normal.
  
  $|N| = 2^2$ giving us two facts:  $|N|$ is abelian as its order
  is $p^2$ for a prime, and $|G/N| = 3$, so the factor group
  is of order 3, making it $C_3$ the cyclic group of prime order,
  and there for also abelian.
  \qed
\end{solution}

\begin{exercise}
  Classify (up to isomorphism) all groups of order 175. 
\end{exercise}
\begin{solution}
  Let $G$ be a group with $|G| = 175$.  $175 = 5^2 \times 7$.

  Looking at the Sylow formulas (?) for $5$, we have $n_5 \equiv 1 \mod{5}$ and
  $n_5 | 7$. The only solution is $n_5 = 1$.  Call the $2-$subgroup $H$.  

  Likewise, looking at the Sylow formulas (?) for $7$, we have $n_7 \equiv 1 \mod{7}$ and
  $n_7 | 25$. The only solution is $n_7 = 1$; call the $7-$subgroup $K$.  

  This Sylow $p-$subgroups are respectively of order $25$ which is a
  prime squared, and hence Abelian, and of order $7$, which is a
  prime, and hence abelian. Due to the internal product theorem, since
  both $H$ and $K$ are normal, and have trivial intersection (due to
  Lagrange theorem and the relative primeness of primes), $G$ is
  isomorphic to $H \times K$.  The direct product of abelian groups is
  itself abelian.  Now, $|K| = 7$, a prime, so it is $\ZZ_7$.  $|H| =
  25$ so it is either $\ZZ_{25}$ or $\ZZ_5 \times \ZZ_5$.  So the two
  possibilities give us two possible $G$:

  Either $G = \ZZ_7 \times \ZZ_{25}$ or $G = \ZZ_7 \times \ZZ_5 \times \ZZ_5$.

\end{solution}

\begin{exercise}
  Find (up to isomorphism) all finite abelian groups such that
  $\Aut(G)$ has odd order.  
\end{exercise}
\begin{solution}
  $|\Aut(G)|$ is odd means $2 \nmid | \Aut(G)$.  Now, if any element $\phi \in \Aut(G)$ is
  such that $\phi \neq I$ (where $I$ is the identity and $\phi^2 = I$, then,
  because $(I, \phi)$ is a subgroup of $\Aut(G)$ of order two, 2 would divide
  the order of $\Aut(G)$.

  So the groups we are looking for all $\phi \in \Aut(G)$, $\phi \neq I$, $\phi ^2 \neq I$.

  But, consider $\phi$ defined as $\phi : g \to g^{-1}$.  This is an automorphism
  of any group, unless $ g = g ^ {-1}$ for all $g \in G$ (in which case it is
  the identity automorphism).  That condition is equivalent to $g^2 = 1$, for all
  $g \in G$.  Also, importantly, $ \phi ^2 = \phi$.  So a $G$ has an element
  of order two in $\Aut(G)$ unless $g ^2 =1 $, for all $g \in G$.

  So the groups we are looking consist only of elements of order 2 (plus the
  identity).  Since $G$ is abelian by assumption, and all the members are of order
  2, $G = \ZZ_2 ^ n$, for some $n$.  However, if $n >1$, then there are automorphisms
  from permuting the representatives of each copy of $\ZZ_2$.  For example, for $\ZZ_2^2$,
  the Klein group, $\Aut(G)$ is isomorphic to $S_3$ because there are three choices
  for each non-identity to map two.  $|S_n|$ is even for $n \geq 2$, so to avoid
  any even members of $\Aut(G)$, $n = 1$, which is to say $G = Z_2$.  So there's
  just one non-trivial abelian group $G$ where $\Aut(G)$ is odd, $\ZZ_2$.  
  
\end{solution}

\begin{comment}
  \begin{exercise}
    problem
  \end{exercise}
  \begin{solution}
    \begin{enumerate}[(a)]
    \item
      first answer
    \end{enumerate}
  \end{solution}
\end{comment}


\end{document}
