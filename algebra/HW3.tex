\documentclass[11pt,oneside]{article}
\usepackage[hmargin=1in,vmargin=1in]{geometry}               % See geometry.pdf to learn the layout options. There are lots.
\geometry{letterpaper}                   % ... or a4paper or a5paper or ...
%\geometry{landscape}                % Activate for for rotated page geometry
%\usepackage[parfill]{parskip}    % Activate to begin paragraphs with an empty line rather than an indent
\usepackage{graphicx}
\usepackage{amssymb}
\usepackage{epstopdf}
\usepackage{url}
%\usepackage{verbatim}
\usepackage{comment}
\specialcomment{solution}{\textbf{Solution. }}{}
%\excludecomment{solution}    %uncomment to remove solutions.

%\usepackage{enumerate}

%Use the enumitem package instead of enumerate
\usepackage[shortlabels]{enumitem}
%\usepackage{enumitem}
%then it will support the same suntax as the enumerate package.
%The enumerate package does not provide any extra configurations other than the label.

%\setlist[enumerate]{topsep=0pt,itemsep=-1ex,partopsep=1ex,parsep=1ex}
\setlist[enumerate]{topsep=0pt,partopsep=0pt}

\DeclareGraphicsRule{.tif}{png}{.png}{`convert #1 `dirname #1`/`basename #1 .tif`.png}
\usepackage{amsmath,amsthm,amscd,amssymb}
\usepackage{latexsym}
\usepackage[colorlinks,citecolor=red,pagebackref,hypertexnames=false]{hyperref}
\numberwithin{equation}{section}

\theoremstyle{definition}
\newtheorem{exercise}{Exercise}
%\newtheorem{solution}{Solution}
\newtheorem*{defn}{Definition}


\def\calA{\mathcal{A}}
\def\calB{\mathcal{B}}
\def\calC{\mathcal{C}}
\def\calT{\mathcal{T}}
\def\OR{\overline{\mathbb{R}}}
\def\RR{\mathbb{R}}
\def\CC{\mathbb{C}}
\def\FF{\mathbb{F}}
\def\QQ{\mathbb{Q}}
\def\ZZ{\mathbb{Z}}
\def\NN{\mathbb{N}}
%\def\NN{\mathbb{Z}_{> 0}}
\def\Nzero{\mathbb{Z}_{\geq 0}}
\def\EE{\mathbb{E}}
\def\PP{\mathbb{P}}
\def\supp{\mathrm{supp}}
\def\diam{\mathrm{diam}}
\def\sp{\mathrm{span}}
\def\ker{\mathrm{ker}}
%\def\sp{\mathrm{span}} %messes up align enviroment
\newcommand{\rbr}[1]{\left( {#1} \right)}
\newcommand{\sbr}[1]{\left[ {#1} \right]}
\newcommand{\cbr}[1]{\left\{ {#1} \right\}}
\newcommand{\abr}[1]{\left\langle {#1} \right\rangle}
\newcommand{\abs}[1]{\left| {#1} \right|}
\newcommand{\norm}[1]{\left\|#1\right\|}
\def\one{\mathbf{1}}
\DeclareMathOperator*{\esssup}{ess\,sup}
\newcommand*\wc{{}\cdot{}}
%\newcommand*\wc{ \, \cdot \,}
%wc for wildcard
\renewcommand{\Re}{\operatorname{Re}}
\renewcommand{\Im}{\operatorname{Im}}
\newcommand{\sgn}{\textup{sgn\,}}


\setlength{\parindent}{0pt}
\setlength{\parskip}{11pt}


%\title{\parbox{14cm}{\centering{  Interior points of circle and sphere packings}}}
\begin{document}

\textbf{HW 3 - MATH 221A - Fall 2023 - Chris Lane}

% \tiny{
%   \input{/home/jayalane/hw/.git/refs/heads/main}
%}
\begin{exercise}
  Prove that the index is multiplicative. That is, prove that if $K \leq H \leq G$, then $[G:K] = [G:H][H:K]$. 
\end{exercise}
\begin{solution}
  The cosets of $K$ in $G$ lie within cosets of $H$ in $G$.

  $$
  (aK) \subseteq (aH)
  $$

  since $K \subseteq H$, $\{ ak : k \in K\} \subseteq \{ ah : k \in H\}$
  So the $K$ cosets of $K$ in $H$ partition $H$ into $[H:K]$ sets, and
  the $K$ cosets of $K$ in $G$ partition the cosets of $H$ into
  $[H:K]$ sets each; in the finite case, a counting argument shows
  that each $H$ in $G$ coset generates $[H:K]$ sets, making the total
  number of $K$ cosets partitioned into $[G:H]$ sets each of $[H:K]$
  subsets.

  There is said to be a more general argument constructing a bijection
  between the various cosets but I could not find it.
  
\end{solution}

\begin{exercise}
  Prove: If $H \leq G$ with $ [G:H] = 2$, then $H$ is normal. 
\end{exercise}
\begin{solution}
  $$ [G:H] = 2$$ so this means that there are only two cosets, both
  left and right.  $H$ is one of the two cosets (as $H = eH = He$),
  and hence the other coset must be $G - H$ as sets; notably this
  result is the same for both left and right cosets.  Therefore all
  the left cosets are right cosets, and $H$ is normal in $G$.
  Incidentally, this means $H$ is the kernel of
  $$
  f: G \to \ZZ_2
  $$
  where f is defined:
  $$
  f(g) = \begin{cases}
    0 & \text{if} \ g \in H  \\
    1 & \text{if} \ g \notin H \\
  \end{cases}
  $$
\end{solution}

\begin{exercise}
  \begin{enumerate}
  \item
    Let $G$ be a group that contains (at least) two non-identity
    elements $a, b$. Suppose that $a^3 = 1$, $b^2 = 1$, and $ba =
    a^2b$. Show that the elements $1, a, a^2, b, ab, a^2b$ are distinct and that
    they form a group.
  \item
    In $S_3$, let $a = (1, 2,3)$ and $b = (1,2)$. Show that $a^3 = 1$,
    $b^2 = 1$, and $ba = a^2 b$. Conclude that $S_3 = \langle a,b | a^3 = 1, b^2 = 1, ba=a^2 \rangle$.
  \item
    Let $H$ be the subgroup of $S_3$ consisting of $b$ and the
    identity.  Find all left coset and all right cosets of $H$
    in $S_3$.
  \item
    Prove that $H$ is not normal in $S_3$. 
  \end{enumerate}
\end{exercise}
\begin{solution}
  \begin{enumerate}
    \item
    Showing distinctness by enumerating the cases:
    \begin{enumerate}[(i)]
    \item
      $1 \neq a$ by construction.
    \item
      $1 \neq b$ also by construction.
    \item
      Suppose $1 = a^2$.  Then $a = a^3$ and then $a = 1$ which is a contradiction.
    \item
      Suppose $1 = ab$.  Then $b = ab^2 $ and then $ b = a$ which is a contradiction.
    \item
      Suppose $1 = a^2b$.  Then $a = a^3b$ and then $a = b$ which is a contradiction.
    \item
      Suppose $ a = a^2$.  Then $a^2 = a^3$ and then $a^2 = 1$ which is disproven above.
    \item
      $a \neq b$ by construction.
    \item
      Suppose $a = ab$.  Then $a^3 = a^3 b$ and $1 = b$ which is a contradiction.
    \item
      Suppose $a = a^2 b$. Then $a^2 = a^3 b$ and then $a^2 = b$ and then $a^4 = b^2$ or $a = 1$ which is a contradiction.
    \item
      $a \neq b$ by construction.
    \item
      Suppose $ a = ab$.  Then $ a^3 = a^3 b$ and then $1 = b$ which is a contradiction.
    \item
      Suppose $a^2 = b$.  Then $ 1 = ab$ and then $ 1 b = a b^2$ and then $b = a$ which is a contradiction.  
    \item
      TODO Suppose $a^2 = ab$.  vThen $a^3 = a^2 b$ and $1 = a^2b$ which is already disproven above.  
    \item
      Suppose $a^2 = a^2b$.  Then $1 = b$ which is a contradiction. 
    \item
      Suppose $b = ab$.  Then $b^2 = a b ^2$ and $1 = a$ which is a contradiction.
    \item
      Suppose $b=a^2b$ then $1 = a^2$ which is already disproven above.  
    \item
      Suppose $ab = a^2b$ .  Then $ a = a^2$ which is already disproven above.
    \end{enumerate}
    So they are all distinct.  $1$ is the identity by definition.
    
    $a ^{-1} = a^2$ as $a^3 = 1$.
    
    $(a^2)^{-1} = a$ by the same logic.
    
    $b^{-1} = b$ as $b^2 = 1$.

    $(ab) ^ {-1} = b^ {-1} a^{-1} = b a^2 = a^2 b a = a^2 a^2 b = ab$

    $(a^2 b ) -1 = b^{-1} (a^2)^{-1}= b a = a^2 b$

    So they all have inverses.  I think associativity is implicit in the notation, but the full table is like this:

      \begin{center}
    \begin{tabular}{ |c|c|c|c|c|c|c| }
      \hline
          {$\times$} & $1$     & $a$    & $a^2$  & $b$    & $ab$    & $a^2 b $ \\
          \hline
          $1$        & $1$     & $a$    & $a^2$  & $b$    & $ab$    & $a^2 b $ \\
          $a$        & $a$     & $a^2$  & $1$    & $ab$   & $a^2 b$ & $b$      \\
          $a^2$      & $a^2$   & $1$    & $a$    & $a^2b$ & $b$     & $ab$     \\
          $b$        & $ b$    & $a^2b$ & $ab$   & $1$    & $a^2$   & $a$      \\
          $ab$       & $ab$    & $b$    & $a^2b$ & $a$    & $1$     & $a^2$    \\
          $a^2b$     & $a^2b$  & $ab$   & $b$    & $a^2$  & $a$     & $1$      \\
          \hline
    \end{tabular}
  \end{center}

    In calculating that table, the key technique was to push all the $b$ to the right using
    $ ba = a^2b$ then you end up with $a^nb^m$ which is $a^{n \bmod 3}b^{m \bmod 2}$.

  \item
      $a^3 = (1, 2, 3)^3 = (1)(2)(3) = \textrm(identity)$ ($1$ goes to $2$ then $3$ then $1$; $2$ goes to $3$ then $1$ then $2$; $3$ likewise goes to $3$.

      $b^2 = (1, 2)^2 = (1, 2)(1, 2) = (1) (2) = \textrm(identity)$, any transposition repeated is identity.

      $ba = (1, 2)(1, 2, 3) = (1) (2, 3)$

      $a^2b = (1, 2, 3)(1, 2, 3)(1, 2) = (1)(2 3)$ and $b = a^2b$.
      The group from above and $S_3$ there for have the same
      generators that follow the same relations, and have the same
      number of elements (so there's no extra relations in $S_3$) so
      they are isomorphic with the assignment of $a$ and $b$ and
      products as stated.
    \item
      $$
      H = \{ (1), (1, 2) \}
      $$
      Then the left cosets of $H$ are:

      \begin{align*}
        (1)H & = \{ (1), (1, 2) \} & \\
        (1, 3)H & = \{ (1,3), (1,3)(1,2) \} = \{ (1,3), (1, 2, 3) \} & \\
        (2, 3)H & = \{ (2,3), (2, 3)(1, 2) \} = \{ (2, 3), (1, 3, 2) \} & \\
      \end{align*}
      (That's all six members).  The right cosets of $H$ are:
      \begin{align*}
        H(1) & = \{ (1), (1, 2) \} & \\
        H(1, 3) &= \{ (1,3), (1,2)(1,3) \} = \{ (1,3), 1, 3, 2)) \} & \\
        H(2, 3) & = \{ (2,3), (1, 2)(2, 3) \} = \{ (2, 3), (1, 2, 3) \} & \\
      \end{align*}
    \item
      The right coset $H (2, 3) = \{ (2, 3), (1, 2, 3) \} $ is not a left coset, so
      the subset $H$ cannot be normal in $S_3$.

  \end{enumerate}
\end{solution}

\begin{exercise}
  Prove or disprove: For every group $G$ and every subgroup $H$ of
  $G$, if $(aH)(bH) \in G / H$ for all $a, b \in G$, then $H$ is
  normal is $G$.
\end{exercise}
\begin{solution}
  I am interpreting that condition as, for all $a, b \in G$, the set
  $$ 
  (abH) = \{ a h_1 b h_2: \text{for all} \ h_1, h_2 \in H \}
  $$
  is a left coset of $H$ in $G$, that is, there exists a
  $c \in G$ and $h_3 \in H$ such that $a h_1 b h_2 = c h_3$.
  I will attempt to prove it.

  Let $h_1, h_2$ be set to $1$. Then, there exists an $h_3$ such that
  $$
  a 1 b 1 = c h_3
  $$

  So $ a b h_3 ^{-1} = c$ for the left coset $cH$.

  So in general, for all $h_1, h_2 \in H$, there is a $h_4 \in H$ such that
  $$
  a h_1 b h_2 = (a b h_3 ^  { -1} ) h_4
  $$
  or
  $$
  a h_1 b h_2 = ( a b ( h_3 ^ {-1} h_4))
  $$
  But this is a restatement that
  $$
  (aH)(bH) = (abH)
  $$
  Which by the characterization of normality theorem in class, means that $H$ is normal in $G$.  
  \qed
\end{solution}

\begin{comment}
  \begin{exercise}
    problem
  \end{exercise}
  \begin{solution}
    \begin{enumerate}[(a)]
    \item
      first answer
    \end{enumerate}
  \end{solution}
\end{comment}


\end{document}
