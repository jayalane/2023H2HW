\documentclass[11pt,oneside]{article}
\usepackage[hmargin=1in,vmargin=1in]{geometry}               % See geometry.pdf to learn the layout options. There are lots.
\geometry{letterpaper}                   % ... or a4paper or a5paper or ...
%\geometry{landscape}                % Activate for for rotated page geometry
%\usepackage[parfill]{parskip}    % Activate to begin paragraphs with an empty line rather than an indent
\usepackage{graphicx}
\usepackage{amssymb}
\usepackage{epstopdf}
\usepackage{url}
%\usepackage{verbatim}
\usepackage{comment}
\specialcomment{solution}{\textbf{Solution. }}{}
%\excludecomment{solution}    %uncomment to remove solutions.

%\usepackage{enumerate}

%Use the enumitem package instead of enumerate
\usepackage[shortlabels]{enumitem}
%\usepackage{enumitem}
%then it will support the same suntax as the enumerate package.
%The enumerate package does not provide any extra configurations other than the label.

%\setlist[enumerate]{topsep=0pt,itemsep=-1ex,partopsep=1ex,parsep=1ex}
\setlist[enumerate]{topsep=0pt,partopsep=0pt}

\DeclareGraphicsRule{.tif}{png}{.png}{`convert #1 `dirname #1`/`basename #1 .tif`.png}
\usepackage{amsmath,amsthm,amscd,amssymb}
\usepackage{latexsym}
\usepackage[colorlinks,citecolor=red,pagebackref,hypertexnames=false]{hyperref}
\numberwithin{equation}{section}

\theoremstyle{definition}
\newtheorem{exercise}{Exercise}
%\newtheorem{solution}{Solution}
\newtheorem*{defn}{Definition}
\newtheorem*{claim}{Claim}

\def\pmod{+'}
\def\calA{\mathcal{A}}
\def\calB{\mathcal{B}}
\def\calC{\mathcal{C}}
\def\calT{\mathcal{T}}
\def\OR{\overline{\mathbb{R}}}
\def\RR{\mathbb{R}}
\def\CC{\mathbb{C}}
\def\FF{\mathbb{F}}
\def\QQ{\mathbb{Q}}
\def\ZZ{\mathbb{Z}}
\def\NN{\mathbb{N}}
%\def\NN{\mathbb{Z}_{> 0}}
\def\Nzero{\mathbb{Z}_{\geq 0}}
\def\EE{\mathbb{E}}
\def\PP{\mathbb{P}}
\def\supp{\mathrm{supp}}
\def\diam{\mathrm{diam}}
\def\sp{\mathrm{span}}
\def\ker{\mathrm{ker}}
\def\Aut{\operatorname{Aut}}
\def\Inn{\operatorname{Inn}}
\def\Ker{\operatorname{Ker}}
%\def\sp{\mathrm{span}} %messes up align enviroment
\newcommand{\rbr}[1]{\left( {#1} \right)}
\newcommand{\sbr}[1]{\left[ {#1} \right]}
\newcommand{\cbr}[1]{\left\{ {#1} \right\}}
\newcommand{\abr}[1]{\left\langle {#1} \right\rangle}
\newcommand{\abs}[1]{\left| {#1} \right|}
\newcommand{\norm}[1]{\left\|#1\right\|}
\def\one{\mathbf{1}}
\DeclareMathOperator*{\esssup}{ess\,sup}
\newcommand*\wc{{}\cdot{}}
%\newcommand*\wc{ \, \cdot \,}
%wc for wildcard
\renewcommand{\Re}{\operatorname{Re}}
\renewcommand{\Im}{\operatorname{Im}}
\newcommand{\sgn}{\textup{sgn\,}}


\setlength{\parindent}{0pt}
\setlength{\parskip}{11pt}

%\title{\parbox{14cm}{\centering{  Interior points of circle and sphere packings}}}
\begin{document}

\textbf{HW 5 - MATH 221A - Fall 2023 - Chris Lane}
%\today


\begin{exercise}
  problem
\end{exercise}
\begin{solution}
  \begin{enumerate}[(a)]
  \item
    first answer
  \end{enumerate}
\end{solution}

\begin{exercise}
  \label{ex2}
  Let $G$ be a group with normal subgroups of prime orders $p$ and
  $q$, $p \neq q$, say $P$ and $Q$. Prove that $G$ contains an
  elements of order $pq$.
\end{exercise}
\begin{solution}
  First, note that these are cyclic groups, due to the prime order.  Let
  $g_p$ be a generator of $P$ and $g_q$ be a generator of $Q$. 

  Secondly, $P \cap Q = \{1\}$.
  
  To see this, suppose there is some other
  member $ x \neq 1$, $x \in P \cap Q$.  So $x = g_p^n$, for some $n < p$, and
  also $x=g_q^m$,$m < q$.

  Then we would have $ g_q ^m = g_p^n$.  However, we know that $x ^ p
  = 1$, and $x^q = 1$.  So $g_q ^ {m p } = 1$ and $g_p ^ {n q} = 1$.
  So $mp = nq \mod |G|$.

  
  Since $ \mod |G| = k p q$, for some $k \in \NN$, this means $mp = nqs + rkpq$, for some
  $ r < k p q$.  But then that is $mp = q ( ns + rkp)$, which is not possible
  over $\ZZ$ since $m < q$ and $\gcd(p, q) = 1$.

  So, since $P$ and $Q$ are normal, with trival intersection, $PQ$ is isomorphic to $P \times Q$.
  This means that members of $ PQ$ can be written as $(g_p^n, g_q^m)$.

  \begin{claim}
    The element $x = (g_p, g_q)$ has order $pq$.
  \end{claim}
  \begin{proof}
    It is clear that $x ^ {pg} = 1$, as this is just $((g_p ^ p)^q, (g_q^q)^p)$ which is $(1^q, 1^p) = 1$.
    
    Now, suppose there's some $k < pq$ such that $x^k = 1$.  Then
    $(g_p^k, g_q^k) = (1, 1)$ which implies $g_p^k = 1$ for $k < qp$
    which implies $ k = np, n<q$ by the cyclic nature of $P$ and $k<pq$.
    But we also have that $g_q ^ k =1$.  Similarly, switching $p$ and
    $q$ we get $k = mq, m < p$.  But then $mq = np$, with $ p \nmid m$.
    But also $p \nmid q$, so there is a contradiction.  This argument is
    also very close to the proof that $P \bigcap Q = \{0\}$.
  \end{proof}
\end{solution}


\begin{exercise}
  Suppose that $H$ and $K$ are subgroups of a finite group $G$ and at
  least one of $H$ and $K$ is normal in $G$.  Prove: if $|H|$ is
  relatively prime to $[G:K]$, then $H \leq K$.
\end{exercise}
\begin{solution}
  A conservation of prime factors.  
\end{solution}

\begin{exercise}
  Let $C_n$ be a cyclic group of order $n$, e.g.
  $C_n = \{1, a, a^2, .. a^{n-1} \}$ with $a^n = 1$.  Similarly, with $C_m$. Prove:
  \begin{enumerate}[(a)]
  \item
    If $n$ and $m$ are relatively prime, then $C_n \times C_m$ is
    isomorphic to $C_{nm}$ and is therefor cyclic.
  \item
    If $n$ and $m$ are not relatively prime, then $C_n \times C_m$ is
    not cyclic.
  \item
    The product of infinite cyclic groups is not cyclic.
  \end{enumerate}
\end{exercise}
\begin{solution}
  \begin{enumerate}[(a)]
  \item
    This is just Exercise \ref{ex2}, with the slightly less
    restrictive condition $\gcd(n, m) = 1$ rather than $n, m$ are prime.

    Consider $C_{nm} = (c_1, c_2)$, $c_1 \in
    C_n$, $c_2 \in C_m$.

  \item
    If $\gcd(n, m) = a$, $a>1$, let $x = (x_n, x_m)$ be a hypothetical
    generator of $C_n \times C_m$.  But then

    \begin{align*}
      x ^ {\frac{nm}{a}} & = (x_n ^ {\frac{nm}{a}}, x_m ^{\frac{nm}{a}}) & \\
      & = ((x_n ^ n) ^ {\frac{m}{a}}, (x_m ^m)^{\frac{n}{a}}) & \\
      & = (1 ^ {\frac{m}{a}}, 1 ^ {\frac{n}{a}}) & \\
      & = (1, 1) \\
    \end{align*}
    
    But then $x$ could not be a generator, which would require it to have
    order equal to the group order, $nm$.
  \item
    For the product of two infinite cyclic groups $C_\infty \times C_\infty$,
    suppose $ g = (a, b)$ generates the group.

    Now, $g$ can't be $(1,0)$ as no combinations of $(1,0) ^n = (n,
    0)$ is ever going to yield $(0,1)$.  But, how can $g$ generate
    $(1,0)$? This would imply: $g ^ n = (a ^ n, b^n) = (1, 0)$.  But
    in the infinite cyclic group generated by 1, there is no $n$ such
    that $a ^n = 1 $, or else it would be finite.  So that's a
    contradiction.  (as is the fact that $b^n =0$ shows that $b = 0$,
    rendering all members like $(0, i)$ ungenerated.  
  \end{enumerate}
\end{solution}


\begin{comment}
  \begin{exercise}
    problem
  \end{exercise}
  \begin{solution}
    \begin{enumerate}[(a)]
    \item
      first answer
    \end{enumerate}
  \end{solution}
\end{comment}

\end{document}
