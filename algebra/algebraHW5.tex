\documentclass[11pt,oneside]{article}
\usepackage[hmargin=1in,vmargin=1in]{geometry}               % See geometry.pdf to learn the layout options. There are lots.
\geometry{letterpaper}                   % ... or a4paper or a5paper or ...
%\geometry{landscape}                % Activate for for rotated page geometry
%\usepackage[parfill]{parskip}    % Activate to begin paragraphs with an empty line rather than an indent
\usepackage{graphicx}
\usepackage{amssymb}
\usepackage{epstopdf}
\usepackage{url}
%\usepackage{verbatim}
\usepackage{comment}
\specialcomment{solution}{\textbf{Solution. }}{}
%\excludecomment{solution}    %uncomment to remove solutions.

%\usepackage{enumerate}

%Use the enumitem package instead of enumerate
\usepackage[shortlabels]{enumitem}
%\usepackage{enumitem}
%then it will support the same suntax as the enumerate package.
%The enumerate package does not provide any extra configurations other than the label.

%\setlist[enumerate]{topsep=0pt,itemsep=-1ex,partopsep=1ex,parsep=1ex}
\setlist[enumerate]{topsep=0pt,partopsep=0pt}

\DeclareGraphicsRule{.tif}{png}{.png}{`convert #1 `dirname #1`/`basename #1 .tif`.png}
\usepackage{amsmath,amsthm,amscd,amssymb}
\usepackage{latexsym}
\usepackage[colorlinks,citecolor=red,pagebackref,hypertexnames=false]{hyperref}
\numberwithin{equation}{section}

\theoremstyle{definition}
\newtheorem{exercise}{Exercise}
%\newtheorem{solution}{Solution}
\newtheorem*{defn}{Definition}
\newtheorem*{claim}{Claim}

\def\pmod{+'}
\def\calA{\mathcal{A}}
\def\calB{\mathcal{B}}
\def\calC{\mathcal{C}}
\def\calT{\mathcal{T}}
\def\OR{\overline{\mathbb{R}}}
\def\RR{\mathbb{R}}
\def\CC{\mathbb{C}}
\def\FF{\mathbb{F}}
\def\QQ{\mathbb{Q}}
\def\ZZ{\mathbb{Z}}
\def\NN{\mathbb{N}}
%\def\NN{\mathbb{Z}_{> 0}}
\def\Nzero{\mathbb{Z}_{\geq 0}}
\def\EE{\mathbb{E}}
\def\PP{\mathbb{P}}
\def\supp{\mathrm{supp}}
\def\diam{\mathrm{diam}}
\def\sp{\mathrm{span}}
\def\ker{\mathrm{ker}}
\def\Aut{\operatorname{Aut}}
\def\Inn{\operatorname{Inn}}
\def\Ker{\operatorname{Ker}}
%\def\sp{\mathrm{span}} %messes up align enviroment
\newcommand{\Mod}{\ (\mathrm{mod}\ )}
\newcommand{\rbr}[1]{\left( {#1} \right)}
\newcommand{\sbr}[1]{\left[ {#1} \right]}
\newcommand{\cbr}[1]{\left\{ {#1} \right\}}
\newcommand{\abr}[1]{\left\langle {#1} \right\rangle}
\newcommand{\abs}[1]{\left| {#1} \right|}
\newcommand{\norm}[1]{\left\|#1\right\|}
\def\one{\mathbf{1}}
\DeclareMathOperator*{\esssup}{ess\,sup}
\newcommand*\wc{{}\cdot{}}
%\newcommand*\wc{ \, \cdot \,}
%wc for wildcard
\renewcommand{\Re}{\operatorname{Re}}
\renewcommand{\Im}{\operatorname{Im}}
\newcommand{\sgn}{\textup{sgn\,}}


\setlength{\parindent}{0pt}
\setlength{\parskip}{11pt}

%\title{\parbox{14cm}{\centering{  Interior points of circle and sphere packings}}}
\begin{document}

\textbf{HW 5 - MATH 221A - Fall 2023 - Chris Lane}
%\today


\begin{exercise}
  Let $G$ be a group with a normal subgroup $N$ of order 5 such that
  $G/N$ is isomorphic to the symmetric group $S_4$. Prove:
  \begin{enumerate}[(c)]
    \item
    $G$ has exactly 4 subgroups of order 15, none of which is normal in $G$.  
  \end{enumerate}
\end{exercise}
\begin{solution}
  \begin{enumerate}[(c)]
  \item
    First of all, we will show that $N \subseteq H$ where $H$ is any
    subgroup of order 15 in $G$.
    
    Now, since $ HN$ is a subgroup, its order must divide 120, which
    is $5 \times 3 \times 2 ^3$.  Now, suppose $ H \cap N$ = {1}.  We
    have $|H| = 15$, so the second result of the isomorphism theorem
    there is $15 / 1 = |HN| / |N|$.  This means $15 = |HN| / 5$ or
    $|HN| = 75$.  But $75 \nmid 120$f, so by
    contradiction $ H \cap N \neq {1}$.  By the order of the group,
    then, and the cyclicity of $N$, then $N \cap N$ must be $N$ itself.

    So now we have the needed conditions for the correspondence
    theorem to apply, $ N \trianglelefteq G$, and $N \leq H \leq G$.
    So there is a one to one correspondence between subgroups of $G$
    containing $H$ and subgroups of $G/N$, which is $S_4$.  So to
    examine subgroups of order 15 in our order 120 group, we need
    to examine subgeroups of order 3 ( $= 15 / |N|$) in $S_4$.  

    By https://groupprops.subwiki.org/wiki/Symmetric\_group:S4 the
    subgroups of $S_4$ of order $3$.  Similar to the last weeks
    homework, and eaerlier homework regarding $S_4$, these subgroups
    are generated by 3-cycles, with each subgroup keeping one symbol
    constant and rotating the other three; hence there are four of
    them. (Generated by $(1 2 3)$, $(1 2 4)$, $(1 3 4)$ and $(2 3 4)$).

    The four subgroups of $G$ corresponde to these subgroups in $S_4$
    would be normal if the $S_4$ subgroups where normal, which they
    aren't.  To see this, it suffices to show for one of them, as they
    are all the same subgroups under automorphisms of $S_4$ to itself
    (relabelling the symbols of the permutation being an automorphism
    of $S_n$ to itself).

    One of the subgroups is generated by $(1 2 3)$, the inverse being
    $(1 3 2)$.  So multiplying these awround $(2 3 4)$ should result
    in a member of the subgroup generated by $(2 3 4)$.  But:

    $$
    (1 2 3)(2 3 4)(1 3 2) = (1 4 3)(2)
    $$

    This is not a member of the subgroup generated by $(2 3 4)$ hence
    that subgroup is not normal, hence none of the four are normal.
  \end{enumerate}
\end{solution}

\begin{exercise}
  \label{ex2}
  Let $G$ be a group with normal subgroups of prime orders $p$ and
  $q$, $p \neq q$, say $P$ and $Q$. Prove that $G$ contains an
  elements of order $pq$.
\end{exercise}
\begin{solution}
  First, note that these are cyclic groups, due to the prime order.  Let
  $g_p$ be a generator of $P$ and $g_q$ be a generator of $Q$. 

  Secondly, $P \cap Q = \{1\}$.
  
  To see this, suppose there is some other
  member $ x \neq 1$, $x \in P \cap Q$.  So $x = g_p^n$, for some $n < p$, and
  also $x=g_q^m$,$m < q$.

  Then we would have $ g_q ^m = g_p^n$.  However, we know that $x ^ p
  = 1$, and $x^q = 1$.  So $g_q ^ {m p } = 1$ and $g_p ^ {n q} = 1$.
  So $mp = nq \pmod |G|$.

  
  Since $ \pmod |G| = k p q$, for some $k \in \NN$, this
  means $mp = nqs + rkpq$, for some $ r < k p q$.  But
  then that is $mp = q ( ns + rkp)$, which is not possible
  over $\ZZ$ since $m < q$ and $\gcd(p, q) = 1$.

  So, since $P$ and $Q$ are normal, with trival intersection, $PQ$ is
  isomorphic to $P \times Q$.  This means that members of $ PQ$ can be
  written as $(g_p^n, g_q^m)$.

  \begin{claim}
    The element $x = (g_p, g_q)$ has order $pq$.
  \end{claim}
  \begin{proof}
    It is clear that $x ^ {pg} = 1$, as this is just $((g_p ^ p)^q, (g_q^q)^p)$ which is $(1^q, 1^p) = 1$.
    
    Now, suppose there's some $k < pq$ such that $x^k = 1$.  Then
    $(g_p^k, g_q^k) = (1, 1)$ which implies $g_p^k = 1$ for $k < qp$
    which implies $ k = np, n<q$ by the cyclic nature of $P$ and $k<pq$.
    But we also have that $g_q ^ k =1$.  Similarly, switching $p$ and
    $q$ we get $k = mq, m < p$.  But then $mq = np$, with $ p \nmid m$.
    But also $p \nmid q$, so there is a contradiction.
  \end{proof}
\end{solution}


\begin{exercise}
  Suppose that $H$ and $K$ are subgroups of a finite group $G$ and at
  least one of $H$ and $K$ is normal in $G$.  Prove: if $|H|$ is
  relatively prime to $[G:K]$, then $H \leq K$.
\end{exercise}
\begin{solution}
  Since once of the $H, K$ are normal, the product set $HK$ is a
  subgroup of $G$. The second isomorphism theorem gives us that fact
  as well as $H \cap K \leq G$, and an isomorphism between
  $ H / ( H \cap K) \simeq (HN / N)$.

  We have $\gcd(H, [G:K]) = 1$.  The isomorphism gives us $|H / (H \cap K)| = | (HN / N)|$.

  Now, $|H| \vert |G| = |K| [ G:K ]$ there for $|H|$ divides $|K|$.  So it
  can't be that $ K \leq H$.  To see about $ H \leq K$, we'll show
  that $ | H / ( H \cap K )| = 1 $.

  $$
  H / (H \cap K) \simeq HK / K
  $$

  So $|H| / |H \cap K| = |HK| / |K| = k$.  

  Since $KH \leq G$, $ |HK|$ divides $|G|$.  This means that $|KH| / |K|$ divides
  $|G| / |K|$.  But $|G| / |K| = [G: K]$. 

  So $k$ divides $ [G : K]$.

  But also, $|H| / |H \cap K|$ divides $|H|$.  That is, $k$ divides $|H|$.

  But $k$ divides both $|H|$ and $[G : K]$ which, however, are mutually prime, so $k = 1$.

  So $|HK| / |K| = 1$, which implies $ HK = K$, which implies

  $$
  H \leq K
  $$

  as required.
  \qed
  
\end{solution}

\begin{exercise}
  Let $C_n$ be a cyclic group of order $n$, e.g.
  $C_n = \{1, a, a^2, .. a^{n-1} \}$ with $a^n = 1$.  Similarly,
  with $C_m$. Prove:
  \begin{enumerate}[(a)]
  \item
    If $n$ and $m$ are relatively prime, then $C_n \times C_m$ is
    isomorphic to $C_{nm}$ and is therefor cyclic.
  \item
    If $n$ and $m$ are not relatively prime, then $C_n \times C_m$ is
    not cyclic.
  \item
    The product of infinite cyclic groups is not cyclic.
  \end{enumerate}
\end{exercise}

\begin{solution}
  \begin{enumerate}[(a)]
  \item

    If $n$ and $m$ are relatively prime,
    define $\phi : C_n \times C_m \to C_{nm}$ to
    be $\phi(a,b) = a*m +_{\pmod{nm}} b$.

    It is 1-1:
    \begin{align*}
      \phi(a,b) & =\phi(a', b') & \\
      am +_\pmod{nm} b & = a' m +_{\pmod{nm}} b' & \textrm{Taking mod m on both sides} & \\
      b & = b' & \textrm {Substituting back} \\
      am + \pmod{nm} b & = a' m +_{\pmod{nm}} b & \\
      am & = a'm \pmod{nm} & \\
      a & = a' \pmod{nm} &
    \end{align*}

    It is onto.  For some $0 <= c < nm$, $c = a * m + b$, $a < n$,
    $b<m$, by the definition of gcd.  Then $\phi(a, b) = c$.

    It preserves group operations: $\phi((a, b) + (a', b')) =
    (am +_{\pmod} b +_{\pmod} a'm +_{\pmod} b') = (a+a')m +_{\pmod} (b+b') = \phi(a,b) + \phi(a', b')$.

    It preserves group inverses: $\phi(a, b) + \phi(-a, -b) =
    am + b +_{\pmod} (-a)m -b = (a-a)m +_{\pmod} (b - b) = 0 {\pmod{nm}}$.

    The generator is $(1, 1)$ as seen in Exercise 2 very similarly.  
    
  \item
    If $\gcd(n, m) = a$, $a>1$, let $x = (x_n, x_m)$ be a hypothetical
    generator of $C_n \times C_m$.  But then

    \begin{align*}
      x ^ {\frac{nm}{a}} & = (x_n ^ {\frac{nm}{a}}, x_m ^{\frac{nm}{a}}) & \\
      & = ((x_n ^ n) ^ {\frac{m}{a}}, (x_m ^m)^{\frac{n}{a}}) & \\
      & = (1 ^ {\frac{m}{a}}, 1 ^ {\frac{n}{a}}) & \\
      & = (1, 1)
    \end{align*}
    
    But then $x$ could not be a generator, which would require it to have
    order equal to the group order, $nm$.
  \item
    For the product of two infinite cyclic groups $C_\infty \times C_\infty$,
    suppose $ g = (a, b)$ generates the group.

    Now, $g$ can't be $(1,0)$ as no combinations of $(1,0) ^n = (n,
    0)$ is ever going to yield $(0,1)$.  But, how can $g$ generate
    $(1,0)$? This would imply: $g ^ n = (a ^ n, b^n) = (1, 0)$.  But
    in the infinite cyclic group generated by 1, there is no $n$ such
    that $a ^n = 1 $, or else it would be finite.  So that's a
    contradiction.  (as is the fact that $b^n =0$ shows that $b = 0$,
    rendering all members like $(0, i)$ ungenerated.  
  \end{enumerate}
\end{solution}


\begin{comment}
  \begin{exercise}
    problem
  \end{exercise}
  \begin{solution}
    \begin{enumerate}[(a)]
    \item
      first answer
    \end{enumerate}
  \end{solution}
\end{comment}

\end{document}
