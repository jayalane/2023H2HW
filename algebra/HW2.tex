\documentclass[11pt,oneside]{article}
\usepackage[hmargin=1in,vmargin=1in]{geometry}               % See geometry.pdf to learn the layout options. There are lots.
\geometry{letterpaper}                   % ... or a4paper or a5paper or ...
%\geometry{landscape}                % Activate for for rotated page geometry
%\usepackage[parfill]{parskip}    % Activate to begin paragraphs with an empty line rather than an indent
\usepackage{graphicx}
\usepackage{amssymb}
\usepackage{epstopdf}
\usepackage{url}
%\usepackage{verbatim}
\usepackage{comment}
\specialcomment{solution}{\textbf{Solution. }}{}
%\excludecomment{solution}    %uncomment to remove solutions.

%\usepackage{enumerate}

%Use the enumitem package instead of enumerate
\usepackage[shortlabels]{enumitem}
%\usepackage{enumitem}
%then it will support the same suntax as the enumerate package.
%The enumerate package does not provide any extra configurations other than the label.

%\setlist[enumerate]{topsep=0pt,itemsep=-1ex,partopsep=1ex,parsep=1ex}
\setlist[enumerate]{topsep=0pt,partopsep=0pt}

\DeclareGraphicsRule{.tif}{png}{.png}{`convert #1 `dirname #1`/`basename #1 .tif`.png}
\usepackage{amsmath,amsthm,amscd,amssymb}
\usepackage{latexsym}
\usepackage[colorlinks,citecolor=red,pagebackref,hypertexnames=false]{hyperref}
\numberwithin{equation}{section}

\theoremstyle{definition}
\newtheorem{exercise}{Exercise}
%\newtheorem{solution}{Solution}
\newtheorem*{defn}{Definition}


\def\calA{\mathcal{A}}
\def\calB{\mathcal{B}}
\def\calC{\mathcal{C}}
\def\calT{\mathcal{T}}
\def\OR{\overline{\mathbb{R}}}
\def\RR{\mathbb{R}}
\def\CC{\mathbb{C}}
\def\FF{\mathbb{F}}
\def\QQ{\mathbb{Q}}
\def\ZZ{\mathbb{Z}}
\def\NN{\mathbb{N}}
%\def\NN{\mathbb{Z}_{> 0}}
\def\Nzero{\mathbb{Z}_{\geq 0}}
\def\EE{\mathbb{E}}
\def\PP{\mathbb{P}}
\def\supp{\mathrm{supp}}
\def\diam{\mathrm{diam}}
\def\sp{\mathrm{span}}
\def\ker{\mathrm{ker}}
%\def\sp{\mathrm{span}} %messes up align enviroment
\newcommand{\rbr}[1]{\left( {#1} \right)}
\newcommand{\sbr}[1]{\left[ {#1} \right]}
\newcommand{\cbr}[1]{\left\{ {#1} \right\}}
\newcommand{\abr}[1]{\left\langle {#1} \right\rangle}
\newcommand{\abs}[1]{\left| {#1} \right|}
\newcommand{\norm}[1]{\left\|#1\right\|}
\def\one{\mathbf{1}}
\DeclareMathOperator*{\esssup}{ess\,sup}
\newcommand*\wc{{}\cdot{}}
%\newcommand*\wc{ \, \cdot \,}
%wc for wildcard
\renewcommand{\Re}{\operatorname{Re}}
\renewcommand{\Im}{\operatorname{Im}}
\newcommand{\sgn}{\textup{sgn\,}}


\setlength{\parindent}{0pt}
\setlength{\parskip}{11pt}


%\title{\parbox{14cm}{\centering{  Interior points of circle and sphere packings}}}
\begin{document}

\textbf{HW 2 - MATH 221A - Fall 2023 - Chris Lane}

% \tiny{
%   \input{/home/jayalane/hw/.git/refs/heads/main}
%}

\begin{exercise}
  Consier the dihedral group $D_8$ as the group of symmetries of the square labeled so that as we move counterclockwise around the square we encounter the vertices A, B, C, D in turn.
  \begin{enumerate}[(a)]
    \item
      List all the elements of $D_8$. Write each as a product of the
      generators $R$, $F$ and also give its cycle decomposition. For
      example, one element is $RF = (A, B)(C, D)$. (Note: $R$ is the
      counter-clockwise rotation by $2 \pi / n = 2 \pi / 4 = \pi / 2 $
      radians. $F$ is the reflection (flip) about the line joing the
      center to the first vertex (A).)
    \item
      Find the order of each element of $D_8$.
  \end{enumerate}
\end{exercise}

\begin{solution}
  First, the four rotations: 
  \begin{enumerate}[(a)]
  \item
    \begin{align*}
    \begin{pmatrix}
      A & B & C & D \\ 
      A & B & C & D
    \end{pmatrix} & = E \text{(the identity)} & \\
    \begin{pmatrix}
      A & B & C & D \\ 
      B & C & D & A 
    \end{pmatrix} & = (A B C D) = R & \\
    \begin{pmatrix}
      A & B & C & D \\ 
      C & D & A & B
    \end{pmatrix} & = (A C)(B D) = R^2 & \\
    \begin{pmatrix}
      A & B & C & D \\ 
      D & A & B & C
    \end{pmatrix} & = (A D C B))  = R ^ 3 = R ^ {-1} &
    \end{align*}
    
    Then the same four preceded by a flip, F:

    \begin{align*}
    \begin{pmatrix}
      A & B & C & D \\ 
      A & D & C & B
    \end{pmatrix} & = (B D) = F \\ 
    \begin{pmatrix}
      A & B & C & D \\ 
      B & C & D & A 
    \end{pmatrix} & = (A B) (C D) = RF \\ 
    \begin{pmatrix}
      A & B & C & D \\ 
      C & D & A & B
    \end{pmatrix} & = (A C) = R^2 F  \\ 
    \begin{pmatrix}
      A & B & C & D \\ 
      D & A & B & C
    \end{pmatrix} & = (A D)(B C)  = R ^ 3 F  = R ^ {-1} F \\
    \end{align*}
  \item
    The members that aren't a four-cycle are all either one 
    transposition or two disjoint transposition, both of which
    are order two.
    \begin{align*}
      |E| & = 1 & \\
      |R| & = 4 & \\
      |R^2| &= 2 & \\
      |R^3| & = | R ^{-1} | = | R | = 4 & 
    \end{align*}
    The other four members are a flip followed by one of the above rotations, which are
    order 2 by the above cycle decomposition:
    \begin{align*}
      |F| & = 2 & \\
      |R F| & = 2 & \\
      |R^2 F| &= 2 & \\
      |R^3 F| & = 2 & 
    \end{align*}
  \end{enumerate}
\end{solution}

\begin{exercise}
  For $n \geq 3$, show that $S _n $ is not abelian.
\end{exercise}

\begin{solution}
  Let $A$ be 
  
  $$ 
  \begin{pmatrix}
    1 & 2 & 3 \\
    2 & 3 & 1 
  \end{pmatrix}
  $$
  Let $B$ be 
  $$
  \begin{pmatrix}
    1 & 2 \\
    2 & 1
  \end{pmatrix}
  $$

  So then $ A B $ is:
  
  $$
  \begin{pmatrix}
    1 & 2 & 3 \\
    3 & 2 & 1 
  \end{pmatrix}
  $$
  
  But $ B A $ is:
  
  $$
  \begin{pmatrix}
    1 & 2 & 3 \\
    1 & 3 & 2
  \end{pmatrix}
  $$
  So $A B \neq B A$.
  
  But writing these permutations in cycle notation for $S_n, n > 3$ (allowing
  $\sigma(n) = n$, for all $n > 3$, where $\sigma$ is the permutation
  at hand).  $ (1 2 3) ( 1 2 ) = ( 1 3) $ and $ ( 1 2 ) ( 1 2 3) = ( 2
  3) $ is an example of a non-Abelian pair in all $S_n$ for $n \geq
  3$.
\end{solution}
\begin{exercise}
  Let $V \subset S _n$ be generated by all pairs of disjoint
  transpositions. Show that this is homomorphic to the Klein group of
  $2 \times 2$ diagonal matrices with diagonal values taken from
  $+1, -1$.
\end{exercise}
\begin{solution}
  The generators of $V$ are

  \begin{align*}
    a & = (1 2) ( 3 4) & \\
    b & = (1 3) ( 2 4) & \\
    c & = (1 4) ( 2 3)
  \end{align*}

  The identity is:
  $$
    e = (1)
  $$
  
  As they are disjoint transpositions,
  $$
  a^2 = b ^ 2 = c ^ 2 = e
  $$
  
  The other multiplications are:
  \begin{align*}
    a b & = (1 4 ) ( 2 3) = c & \\
    b a & = (1 4 ) ( 2 3)  = c & \\
    a c & = (1 3) (2 4) = b & \\
    c a & = (1 3) (2 4)  = b & \\
    b c & = (1 2) (3 4) = a& \\
    c b & = (1 2) (3 4) = a
  \end{align*}

  The full table (using the labels e, a, b, c) is:

  \begin{center}
    \begin{tabular}{ |c|c|c|c|c| }
      \hline
          {$\times$} & $e$ & $a$ & $b$ & $c$ \\
          \hline
          $e$      & $e$ & $a$ & $b$ & $c$ \\
          $a$      & $a$ & $e$ & $c$ & $b$ \\
          $b$      & $b$ & $c$ & $e$ & $a$ \\
          $c$      & $c$ & $b$ & $a$ & $e$ \\
          \hline
    \end{tabular}
  \end{center}

  For the Klein group, the members are:
  \begin{align*}
    E = \begin{bmatrix} 1 & 0 \\
                        0 & 1 
    \end{bmatrix}
    A = \begin{bmatrix} -1 & 0 \\
                        0 & 1 
    \end{bmatrix}
    B = \begin{bmatrix} 1 & 0 \\
                        0 & -1 
    \end{bmatrix}
    C = \begin{bmatrix} -1 & 0 \\
                        0 & -1 
    \end{bmatrix}
  \end{align*}
  As E is the identity of matrix multiplication, the operations involving E are: 
$ E  A = A = A E$, $ E B = B = B E$, $E C = C = C E$.  

\begin{align*}
  A B = \begin{bmatrix}
    -1 \times 1 + 0 \times 0 & -1 \times 0 + 0 \times -1 \\
    0 \times 1 + 1 \times 0 & 0 \times 0 + 1 \times -1
  \end{bmatrix}
  = \begin{bmatrix}
    -1 & 0 \\
    0 & -1
    \end{bmatrix} = C
\end{align*}
\begin{align*}
  B A = \begin{bmatrix}
    1 \times -1 + 0 \times 0 & 1 \times 0 + 0 \times 1 \\
    0 \times -1 + -1 \times 0 & 0 \times 0 + -1 \times 1
  \end{bmatrix}
  = \begin{bmatrix}
    -1 & 0 \\
    0 & -1
    \end{bmatrix} = C
\end{align*}

\begin{align*}
  B C = \begin{bmatrix}
    1 \times -1 + 0 \times 0 & 1 \times 0 + 0 \times -1 \\
    0 \times -1 + -1 \times 0 & 0 \times 0 + -1 \times -1
  \end{bmatrix}
  = \begin{bmatrix}
    -1 & 0 \\
    0 & 1
    \end{bmatrix} = A
\end{align*}

\begin{align*}
  C B = \begin{bmatrix}
    -1 \times 1 + 0 \times 0 & 1 \times 0 + 0 \times -1 \\
    0 \times -1 + -1 \times 0 & 0 \times 0 + -1 \times -1
  \end{bmatrix}
  = \begin{bmatrix}
    -1 & 0 \\
    0 & 1
    \end{bmatrix} = A
\end{align*}

\begin{align*}
  A C = \begin{bmatrix}
    -1 \times -1 + 0 \times 0 & -1 \times 0 + 0 \times -1 \\
    0 \times -1 + -1 \times 0 & 0 \times 0 + 1 \times -1
  \end{bmatrix}
  = \begin{bmatrix}
    1 & 0 \\
    0 & -1
    \end{bmatrix} = B
\end{align*}
\begin{align*}
  C A = \begin{bmatrix}
    -1 \times -1 + 0 \times 0 & 0 \times -1 + -1 \times 0 \\
    -1 \times 0 + 0 \times -1 & 0 \times 0 + -1 \times 1
  \end{bmatrix}
  = \begin{bmatrix}
    1 & 0 \\
    0 & -1
    \end{bmatrix} = B
\end{align*}

The squares are like this:
\begin{align*}
  A A = \begin{bmatrix}
    -1 \times -1 + 0 \times 0 & 0 \times 1 + -1 \times 0 \\
    -1 \times 0 + 0 \times 1 & 0 \times 0 + 1 \times 1
  \end{bmatrix}
  = \begin{bmatrix}
    1 & 0 \\
    0 & 1
    \end{bmatrix} = E
\end{align*}
\begin{align*}
  B B = \begin{bmatrix}
    1 \times 1 + 0 \times 0 & 0 \times -1 + 1 \times 0 \\
    1 \times 0 + 0 \times -1 & 0 \times 0 + -1 \times -1
  \end{bmatrix}
  = \begin{bmatrix}
    1 & 0 \\
    0 & 1
    \end{bmatrix} = E
\end{align*}
\begin{align*}
  C C = \begin{bmatrix}
    -1 \times -1 + 0 \times 0 & 0 \times -1 + -1 \times 0 \\
    -1 \times 0 + 0 \times -1 & 0 \times 0 + -1 \times -1
  \end{bmatrix}
  = \begin{bmatrix}
    1 & 0 \\
    0 & 1
    \end{bmatrix} = E
\end{align*}

Yielding a multiplication table like this:

  \begin{center}
    \begin{tabular}{ |c|c|c|c|c| }
      \hline
          {$\times$} & $E$ & $A$ & $B$ & $C$ \\
          \hline
          $E$      & $E$ & $A$ & $B$ & $C$ \\
          $A$      & $A$ & $E$ & $C$ & $B$ \\
          $B$      & $B$ & $C$ & $E$ & $A$ \\
          $C$      & $C$ & $B$ & $A$ & $E$ \\
          \hline
    \end{tabular}
  \end{center}

  Which by inspection is the same as the operation matrix for the
  permutation group.
\end{solution}
\begin{exercise}
  Suppose that $H$ is a subgroup of $A _n $ with the property that for
  all $\sigma \in A_n$, $\sigma ^ 2 \in H$.  Prove that all three-cycles
  are in $H$.
\end{exercise}
\begin{solution}
  Let $\tau$ be a three cycle.  Then there exist $x_1, x_2, x_3 $ such
  that $\tau = ( x_1, x_2, x_3)$.  By assumption, $\tau ^2 \in H$.
  But $\tau ^2 = (x_1, x_3, x_2)$. Since $H$ is a subgroup,
  $\tau ^2 \times \tau ^2 \in H$.  But $\tau ^2 \times \tau ^2$
  is $ (x_1, x_2, x_3)$ which is $\tau$ itself, so
  therefore for any three cycle, it is contained in $H$.
  \qed
  

\end{solution}

\begin{exercise}
  Given
  \begin{center}
  $$
  \pi = \begin{bmatrix}
    1 & 2 & 3 & 4 & 5 \\
    2 & 4 & 5 & 1 & 3
  \end{bmatrix}
  $$
  \end{center}
Count the number of inversions of $\pi$. (An inversion is a pair of
integrats in the second row of $\pi$ that are out of their natural
order, i.e., a pair $(\pi(i), \pi(j))$ where $i < j$ and $\pi(i) > \pi(j)$.
What is the parity of the number of inversions of $\pi$? What is the parity of $\pi$?
\end{exercise}
\begin{solution}
  For reference, $\pi = (1 2 4) (3 5)$

  First I will list the pairs involved:
  
  \begin{tabular}{| c | c | c |}
      \hline 
    Pair & $\pi(\text{pair})$ & 1 if swapped, 0 if natural \\
      \hline 
    (1, 2) & (2, 4) & 0 \\
    (1, 3) & (2, 5) & 0 \\
    (1, 4) & (2, 1) & 1 \\
    (1, 5) & (2, 3) & 0 \\
    (2, 3) & (4, 5) & 0 \\
    (2, 4) & (4, 1) & 1 \\
    (2, 5) & (4, 3) & 1 \\
    (3, 4) & (5, 1) & 1 \\
    (3, 5) & (5, 3) & 1 \\
    (4, 5) & (1, 3) & 0 \\
\hline
  \end{tabular}

  The number of inversions is 5, which is odd parity.  From the cycle
  decomposition, (a 3-cycle and a 2-cycle, which has 1 even cycle) the
  permutation is an odd permutation.
  
\end{solution}
\begin{exercise}
For an arbitrary (finite) permutation $\sigma$, the parity of the
permutation is equal to the parity of the number of inversions of
$\sigma$.  
\end{exercise}
\begin{solution}
  By induction on the number of transpositions in a permutation.

  Base case: the identity has zero transpositions, and also is an
  order preserving map of $\NN_n$ to itself, and hence has zero
  inversions.

  Induction case: $\sigma$ has an expression as $n$
  transpositions. Let $(a _ 1, a_2, a_3, ... a_k)$ be a cycle in its
  representation. Replace it with $(a_1) (a_2,. ... a_k)$ (where $k$
  might be 2).  This $ \sigma ' $ has one less transposition than $\sigma$, and so the parity
  of this permutation matches the parity of the inversions.

  The changes in the inversions are due to the changes in:
  \begin{align*}
    \sigma ( a_1 ) = & a_2  \text{but}  \sigma ' (a_1) = a_1 \\
    \sigma ( a_k) = & a_1 \text{but} \sigma ' (a_k) = a_2
  \end{align*}
  In which case, the pair $(a_1, a_2) $ is moved to $(a_2, a_1)$ which
  reverses the sense of the comparison, hence altering the parity by
  one.  It also switched all the pairs $(a_j, a_1)$ and $(a_j, a_2)$.
  In no case does the switch alter the number of inversions. The
  distinct pairs are still less than or greater than as they were.  
\end{solution}

\begin{comment}
  \begin{exercise}
    problem
  \end{exercise}
  \begin{solution}
    \begin{enumerate}[(a)]
    \item
      first answer
    \end{enumerate}
  \end{solution}
\end{comment}




\end{document}
