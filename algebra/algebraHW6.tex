\documentclass[11pt,oneside]{article}
\usepackage[hmargin=1in,vmargin=1in]{geometry}               % See geometry.pdf to learn the layout options. There are lots.
\geometry{letterpaper}                   % ... or a4paper or a5paper or ...
%\geometry{landscape}                % Activate for for rotated page geometry
%\usepackage[parfill]{parskip}    % Activate to begin paragraphs with an empty line rather than an indent
\usepackage{graphicx}
\usepackage{datetime}
\usepackage{amssymb}
\usepackage{epstopdf}
\usepackage{url}
%\usepackage{verbatim}
\usepackage{comment}
\specialcomment{solution}{\textbf{Solution. }}{}
%\excludecomment{solution}    %uncomment to remove solutions.

%\usepackage{enumerate}

%Use the enumitem package instead of enumerate
\usepackage[shortlabels]{enumitem}
%\usepackage{enumitem}
%then it will support the same suntax as the enumerate package.
%The enumerate package does not provide any extra configurations other than the label.

%\setlist[enumerate]{topsep=0pt,itemsep=-1ex,partopsep=1ex,parsep=1ex}
\setlist[enumerate]{topsep=0pt,partopsep=0pt}

\DeclareGraphicsRule{.tif}{png}{.png}{`convert #1 `dirname #1`/`basename #1 .tif`.png}
\usepackage{amsmath,amsthm,amscd,amssymb}
\usepackage{latexsym}
\usepackage[colorlinks,citecolor=red,pagebackref,hypertexnames=false]{hyperref}
\numberwithin{equation}{section}

\theoremstyle{definition}
\newtheorem{exercise}{Exercise}
%\newtheorem{solution}{Solution}
\newtheorem*{defn}{Definition}
\newtheorem*{claim}{Claim}

\def\pmod{+'}
\def\boldr{\boldsymbol{r}}
\def\boldc{\boldsymbol{c}}

\def\calA{\mathcal{A}}
\def\calB{\mathcal{B}}
\def\calC{\mathcal{C}}
\def\calT{\mathcal{T}}
\def\OR{\overline{\mathbb{R}}}
\def\RR{\mathbb{R}}
\def\CC{\mathbb{C}}
\def\FF{\mathbb{F}}
\def\QQ{\mathbb{Q}}
\def\ZZ{\mathbb{Z}}
\def\NN{\mathbb{N}}
%\def\NN{\mathbb{Z}_{> 0}}
\def\Nzero{\mathbb{Z}_{\geq 0}}
\def\EE{\mathbb{E}}
\def\PP{\mathbb{P}}
\def\supp{\mathrm{supp}}
\def\diam{\mathrm{diam}}
\def\sp{\mathrm{span}}
\def\ker{\mathrm{ker}}
\def\Aut{\operatorname{Aut}}
\def\Inn{\operatorname{Inn}}
\def\Ker{\operatorname{Ker}}
%\def\sp{\mathrm{span}} %messes up align enviroment
\newcommand{\Mod}{\ (\mathrm{mod}\ )}
\newcommand{\rbr}[1]{\left( {#1} \right)}
\newcommand{\sbr}[1]{\left[ {#1} \right]}
\newcommand{\cbr}[1]{\left\{ {#1} \right\}}
\newcommand{\abr}[1]{\left\langle {#1} \right\rangle}
\newcommand{\abs}[1]{\left| {#1} \right|}
\newcommand{\norm}[1]{\left\|#1\right\|}
\def\one{\mathbf{1}}
\DeclareMathOperator*{\esssup}{ess\,sup}
\newcommand*\wc{{}\cdot{}}
%\newcommand*\wc{ \, \cdot \,}
%wc for wildcard
\renewcommand{\Re}{\operatorname{Re}}
\renewcommand{\Im}{\operatorname{Im}}
\newcommand{\sgn}{\textup{sgn\,}}


\setlength{\parindent}{0pt}
\setlength{\parskip}{11pt}

%\title{\parbox{14cm}{\centering{  Interior points of circle and sphere packings}}}
\begin{document}

\textbf{HW 6 - MATH 221A - Fall 2023 - Chris Lane}

Date: \hhmmsstime{} \ \today \ \ Git hash: 
\input{/home/jayalane/hw/.git/refs/heads/main}

\begin{exercise}
  Let $R$ be a ring and $I$ be an additive subgroup of $R$.  Prove or
  disprove: $I$ is an ideal if and only if $(r+I)(s+I)$ is a coset of
  $I$ for all $r, s \in R$.
\end{exercise}
\begin{solution}
  I've tried a number of counter examples (with various quotients of
  polynomials rings and various oddly shaped matrices).  

  I've also tried a variety of things to prove it, with no
  success. The most promising thing was letting the condition define a
  multiplication among cosets in the factor ring, and then verify the
  natural ring homomorphism agrees with the ring operation, but I
  can't find it.
  
\end{solution}

\begin{exercise}
  If $R$ is a field, what are the units in $R[X]$?  Justify your answer.  
\end{exercise}
\begin{solution}
  If $R$ is a field, the units of $R[X]$ :

  $$
  \{ a \in R[X] : \textrm{ for some } b \in R[X], a \cdot b = 1 _ {R[X]} \}
  $$

  Now, $1_{R[X]} = 1$, so we want:

  $$
  (a_0 + a_1 x + a_2 x^2 + ... a_n x^n) ( b_0 + b_1 x + ... b_m x^m) = 1
  $$

  Since the $x$ are just symbols here, equality means equality of all the coefficients.

  $$
  a_0 b_0 + (a_0 b_1 + b_0 a_1 ) x + ... + a_n b_m x^ {n+m} = 1
  $$
   
  Or,
  \begin{align*}
    a_0 b_0 & = 1 & \\
    a_0 b_1 + b_0 a_1 & =0 \\
    ...& & \\
    a_n b_m & = 0
  \end{align*}

  \begin{claim}
    For a field, the product of two polynomials, all of the
    coefficients being zero except for the zeroth one implies all the
    coefficients of the factors are zero except for the zeroeth one.  
  \end{claim}
  \begin{proof}
    By induction on $n$, starting with $n = 1$ (the result is
    trivial if one of the polynomials is just a constant).

    Base case:

    $$
    (a_0 + a_1x) (b_0 + a_1x) =  1
    $$

    So

    \begin{equation}
      \label{eq1}
      a_0 b_1 + a_1 b_0 = 0
    \end{equation}

    and
    
    $$
    a_0 b_0 = 1
    $$
    
    So in particular, since R is a field, they are both $\neq 0$. Also: 

    $$
    a_1 b_1 = 0
    $$
    
    So in particular one of them $ = 0$, again because this is a field.
    We can assume $a_0 = 0$. So \ref{eq1} becomes:

    $$
    a_1 b_0 = 0
    $$
    
    But since we know $b_0 \neq 0$, we know $a_1 = 0$.  That proves the base case.

    For the induction step, if the first polynomial is of degree $n$,

    $$
    a_0 + a_1 x + a_2 + x^2 + ... + a_n x^ n
    $$

    Then if:

    $$
    (a_0 + a_1 x + a_2 x^2 + ... + a_n x^ n) (b_0 + b_1 x + b_2 x^2 + ... + b_m x^ m) = 1
    $$

    Then $1 = a_0 b_0$. Looking at the degree $n+m$ coeffients, $0 = a_n b_m$.

    Looking at the degree $n+m -1$ cofficients, $0 = a_{n_1} b_m + a_n b_{m-1}$.

    Multiplying each side there by $a_n$, we get $0 = a_{n_1} a_n b_m + a_n ^2 b_{m-1}$, but $a_n b_m = 0$, so
    $ 0 = a_{n_1} 0 + a_n^2 b_{m-1}$ or $a_n^2 b_{m-1} = 0$.

    Similarly, $ a_n^{i +1} b_{m - i} = 0$.

    Taking this down to $b_0$, we get $a_n^{m+1} b_0 = 0$.  Then we can multiply by $a_0$, getting
    $ a_n^{m+1} a_0 b_0 = 0$, or $a_n^{m+1} = 0$, which makes

    $$
    a_0 + a_1 x + a_2 + x^2 + ... + a_n x^ n = a_0 + a_1 x + a_2 + x^2 + ... + a_{n-1} x^ {n-1}
    $$

    a polynomial of degree $n-1$, proving the induction case.  
  \end{proof}

  So for a unit, all of the higher degree coefficients are 0, leaving us with

  $$
  \{ a \in R: a \neq 0 \}
  $$
  \qed
  
\end{solution}

\begin{exercise}
  Let $M_n(R)$ be the ring of $n$ by $n$ matrices with coefficients
  in the ring $R$. If $C_k$ is the subset of $M_n(R)$ consisting of
  matrices that are $0$ except perhaps in column $k$, show tyhat
  $C_k$ is a left ideal of $M_n(R)$.  SImilarly, if $R_k$ consists
  of matrices that are $0$ except perhaps in row $k$, then $R_k$ is a
  right ideal of $M_n(R)$.  
\end{exercise}
\begin{solution}
  First, $C_k$ is an additive subgroup of $M_n(R)$.
  \begin{enumerate}[(a)]
  \item
    Non-empty

    e.g.
    
    \small
    \[
      \begin{bmatrix}
        0 & 0 & ... & 0 & a_{1k} & 0 & ... & 0 \\
        0 & 0 & ... & 0 & a_{2k} & 0 & ... & 0 \\
        & \ \vdots & & \ \vdots  & & & \ \vdots \\
        0 & 0 & ... & 0 & a_{nk} & 0 & ... & 0 \\
      \end{bmatrix}
    \]
    \normalsize
    
  \item
    Closed under addition

    \small
    \[
      \begin{bmatrix}
        0 & ... & 0 & a_{1k} & 0 & ... & 0 \\
        0 & ... & 0 & a_{2k} & 0 & ... & 0 \\
        \ \vdots & & \ \vdots  & & & \ \vdots \\
        0 & ... & 0 & a_{nk} & 0 & ... & 0 \\
      \end{bmatrix} + \begin{bmatrix}
        0 & ... & 0 & b_{1k} & 0 & ... & 0 \\
        0 & ... & 0 & b_{2k} & 0 & ... & 0 \\
        \ \vdots & & \ \vdots  & & & \ \vdots \\
        0 & ... & 0 & b_{nk} & 0 & ... & 0 \\
      \end{bmatrix} = \begin{bmatrix} 
        0 & ... & 0 & a_{1k} + b_{1k}& 0 & ... & 0 \\
        0 & ... & 0 & a_{2k} + b_{2k}& 0 & ... & 0 \\
        \ \vdots & & \ \vdots  & & & \ \vdots \\
        0 & ... & 0 & a_{nk} + b_{nk}& 0 & ... & 0 \\
      \end{bmatrix} \in C_k
    \]
    \normalsize
    
      \item
    Closed under inverses

    \small
    \[
      \begin{bmatrix}
        0 & ... & 0 & a_{1k} & 0 & ... & 0 \\
        0 & ... & 0 & a_{2k} & 0 & ... & 0 \\
        \ \vdots & & \ \vdots  & & & \ \vdots \\
        0 & ... & 0 & a_{nk} & 0 & ... & 0 \\
      \end{bmatrix} + \begin{bmatrix}
        0 & ... & 0 & -a_{1k} & 0 & ... & 0 \\
        0 & ... & 0 & -a_{2k} & 0 & ... & 0 \\
        \ \vdots & & \ \vdots  & & & \ \vdots \\
        0 & ... & 0 & -a_{nk} & 0 & ... & 0 \\
      \end{bmatrix} = \begin{bmatrix} 
        0 & ... & 0 & a_{1k} - a _{1k}& 0 & ... & 0 \\
        0 & ... & 0 & a_{2k} - a_{2k}& 0 & ... & 0 \\
        \ \vdots & & \ \vdots  & & & \ \vdots \\
        0 & ... & 0 & a_{nk} - a_{nk}& 0 & ... & 0 \\
      \end{bmatrix} = 0
    \]
    \normalsize

\item
  $C_k$ is also a left-ideal.  Let $M \in M(R)$ be a matrix and $C \in C_k$.  We will
  write this multiplication in ``column of row vectors $\times$ row of column vectors'' notation.

  \[
  M = \begin{bmatrix}
    \boldr_1 & \\
    \boldr_2 & \\
    \boldr_3 & \\
    \ \vdots & \\
    \boldr_n & \\
  \end{bmatrix}, C_k = \begin{bmatrix}
    0 & 0 & ... & 0 & \boldc & 0 & ... & 0
  \end{bmatrix}
 \]

  Then, $M \times C$ is:
  \[
  M \times C = \begin{bmatrix}
    \boldr_1 \cdot 0 & \boldr_1 \cdot 0 & ... & \boldr_1 \cdot 0 & \boldr_1 \cdot c & \boldr_1 \cdot 0 & ... & \boldr_1 \cdot 0 \\
    \boldr_2 \cdot 0 & \boldr_2 \cdot 0 & ... & \boldr_2 \cdot 0 & \boldr_2 \cdot c & \boldr_2 \cdot 0 & ... & \boldr_2 \cdot 0 \\
    \ \vdots &        &  & & \ \vdots & & & \ \vdots \\
    \boldr_n \cdot 0 & \boldr_n \cdot 0 & ... & \boldr_n \cdot 0 & \boldr_n \cdot c & \boldr_n \cdot 0 & ... & \boldr_n \cdot 0 \\
  \end{bmatrix} \in C_k
 \]
  

  \end{enumerate}
\end{solution}
\begin{exercise}
  Prove: If $R$ is a commutative ring whose only principal ideals are
  ${0} = \langle 0 \rangle$ and $ R = \langle 1 \rangle$, then $R$ is
  a field.
\end{exercise}
\begin{solution}
  Given an arbitrary element $c \in R, c \neq 0$.
  Consider $\langle c \rangle$.  It cannot be ${0}$ as $c \neq 0$.
  So $\langle c \rangle = \langle 1 \rangle $.
  That means, $ 1 \in \langle c \rangle$.

  By the linear formation of $\langle c \rangle$, this means there is
  a $d \in R$ such that $1 = d c$ (where the sum is just over $1$
  element, $c$.  But then $d$ is a multiplicative inverse of $c$.  If
  an arbitrary non-zero element has a multiplicative inverse, then the
  ring is a field. 
  
  
\end{solution}
\begin{exercise}
  Prove: If $R$ is a ring and $(R, +)$ is cyclic, then every ideal in $R$ is principal.  
\end{exercise}
\begin{solution}
  Let $I$ be an ideal.

  Each $i \in I, i \neq 0$, is $i=a^{k_i}$, for some $k, k>0$, because $R$ is cyclic.

  Then there is an $i' \in I$ where $k'$ is the smallest of the $k_i$
  (as a subset of $\NN$ there is a smallest).

  Claim:  $ \langle i' \rangle = I$.

  Let $j \in I$.  Since $j \in R$, there is $j \in \ZZ$ such that $j = a^m$.

  Now, by the Euclidean algorithm, there are $n, r \in \ZZ$ such that

  $$
  m = k' n + r, r<k'
  $$

  The claim is equivalent to showing that $r=0$.  Supppose not: then,

  $$
  m = k' n + r, r<k'
  $$

  implies

  $$
  a^m = a^{(k'n + r)}
  $$

  implies (note the slightly confusing use of exponentiation to
  represented repeateed $+$ operations in a ring; the multiplicative
  notation would have been more confusing tho as this is repeated
  addition, not ring multiplication).  

  $$
  j = a^{(k'n)} + a^r
  $$

  implies

  $$
  j + (-(a^{k'}n) = a^r
  $$

  Now, $j + (-(a^{k'}n)) \in I$ as it is closed under + and - (as + inverse operation).

  That means $a^r \in I$.  But $ r < k'$ which is a contradiction.  So
  $r=0$, and the claim holds, and we have shown every ideal is generated by one element, that is
  every ideal is principal.
  \qed
  
  
\end{solution}
\begin{exercise}
  Let $I_1, ... I_n$ be ideals in a ring $R$.
  Suppose $R/\cap_{i=1}^n I_i$ is isomorphic to $R/I_1 \times ... \times R/I_n$ via the map
  $a+\cap_{i=1}I_i \mapsto (a + I_1, ... , a+ I_n)$.  Show that $I_1,
  ... , I_n$ are pairwise relatively prime.
\end{exercise}
\begin{solution}

  Choose $ i \neq j$.  Assume without loss of generality that $i=1$,
  $j=2$.  Further, let $I = I_1$ and $J = I_2$.

  There is an element of the product:

  \[
  (0 + I, 1 + J, 0 + I_3, ... , 0 + I_4)
  \]

  Since there is an isomorphism from  $R/\cap_{i=1}^n I_i$,
  there is a member of that ring, $a + \cap_{i=1}^n I_i$ such that

  \[
  \phi(a) =   (0 + I, 1 + J, 0 + I_3, ... , 0 + I_4)
  \]
  
  So, there exists an $x \in \bigcap \limits _{i=1} ^ n$, $x \in I$, also $x \in J$, such that:

  \[
  a + x = 0 + i, \textrm{ some } i \in I
  \]

  or
  \[
  a = i - x
  \]
  
  which means $a \in I$.  For $J$,:

  \[
  a + x = 1 + j, \textrm{ some } j \in J
  \]
  
  So

  \[
  i - x + x - j = 1
  \]

  Therefore, we have produced:

  \[
  i - j = 1, i \in I, j \in J
  \]

  Or, $I + J = R$, $I_1$ $I_2$ are relatively prime
  \qed

\end{solution}

\begin{comment}
  \begin{exercise}
    problem
  \end{exercise}
  \begin{solution}
    \begin{enumerate}[(a)]
    \item
      first answer
    \end{enumerate}
  \end{solution}
\end{comment}

\end{document}
