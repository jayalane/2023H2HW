\documentclass[11pt,oneside]{article}
\usepackage[hmargin=1in,vmargin=1in]{geometry}               % See geometry.pdf to learn the layout options. There are lots.
\geometry{letterpaper}                   % ... or a4paper or a5paper or ...
%\geometry{landscape}                % Activate for for rotated page geometry
%\usepackage[parfill]{parskip}    % Activate to begin paragraphs with an empty line rather than an indent
\usepackage{graphicx}
\usepackage{amssymb}
\usepackage{epstopdf}
\usepackage{url}
%\usepackage{verbatim}
\usepackage{comment}
\specialcomment{solution}{\textbf{Solution. }}{}
%\excludecomment{solution}    %uncomment to remove solutions.

%\usepackage{enumerate}

%Use the enumitem package instead of enumerate
\usepackage[shortlabels]{enumitem}
%\usepackage{enumitem}
%then it will support the same suntax as the enumerate package.
%The enumerate package does not provide any extra configurations other than the label.

%\setlist[enumerate]{topsep=0pt,itemsep=-1ex,partopsep=1ex,parsep=1ex}
\setlist[enumerate]{topsep=0pt,partopsep=0pt}

\DeclareGraphicsRule{.tif}{png}{.png}{`convert #1 `dirname #1`/`basename #1 .tif`.png}
\usepackage{amsmath,amsthm,amscd,amssymb}
\usepackage{latexsym}
\usepackage[colorlinks,citecolor=red,pagebackref,hypertexnames=false]{hyperref}
\numberwithin{equation}{section}

\theoremstyle{definition}
\newtheorem{exercise}{Exercise}
%\newtheorem{solution}{Solution}
\newtheorem*{defn}{Definition}
\newtheorem*{claim}{Claim}

\def\pmod{+'}
\def\calA{\mathcal{A}}
\def\calB{\mathcal{B}}
\def\calC{\mathcal{C}}
\def\calT{\mathcal{T}}
\def\OR{\overline{\mathbb{R}}}
\def\RR{\mathbb{R}}
\def\CC{\mathbb{C}}
\def\FF{\mathbb{F}}
\def\QQ{\mathbb{Q}}
\def\ZZ{\mathbb{Z}}
\def\NN{\mathbb{N}}
%\def\NN{\mathbb{Z}_{> 0}}
\def\Nzero{\mathbb{Z}_{\geq 0}}
\def\EE{\mathbb{E}}
\def\PP{\mathbb{P}}
\def\supp{\mathrm{supp}}
\def\diam{\mathrm{diam}}
\def\sp{\mathrm{span}}
\def\ker{\mathrm{ker}}
\def\Aut{\operatorname{Aut}}
\def\Inn{\operatorname{Inn}}
\def\Ker{\operatorname{Ker}}
%\def\sp{\mathrm{span}} %messes up align enviroment
\newcommand{\rbr}[1]{\left( {#1} \right)}
\newcommand{\sbr}[1]{\left[ {#1} \right]}
\newcommand{\cbr}[1]{\left\{ {#1} \right\}}
\newcommand{\abr}[1]{\left\langle {#1} \right\rangle}
\newcommand{\abs}[1]{\left| {#1} \right|}
\newcommand{\norm}[1]{\left\|#1\right\|}
\def\one{\mathbf{1}}
\DeclareMathOperator*{\esssup}{ess\,sup}
\newcommand*\wc{{}\cdot{}}
%\newcommand*\wc{ \, \cdot \,}
%wc for wildcard
\renewcommand{\Re}{\operatorname{Re}}
\renewcommand{\Im}{\operatorname{Im}}
\newcommand{\sgn}{\textup{sgn\,}}


\setlength{\parindent}{0pt}
\setlength{\parskip}{11pt}

%\title{\parbox{14cm}{\centering{  Interior points of circle and sphere packings}}}
\begin{document}

\textbf{HW 4 - MATH 221A - Fall 2023 - Chris Lane}

\begin{exercise}
  \begin{enumerate}[(a)]
  \item
    Let $\ZZ$ be the integers and $n\ZZ$ be the set of integer
    multiples of $n$.  Show that $\ZZ / n\ZZ$ is isomorphic to $\ZZ_n$,
    the additive group of integers module $n$.  (This is not quite a
    tautology if we view $\ZZ_n$ concretely as the set $\{0, 1, 2, ..., n-1\}$,
    with the sums and differences reduced module $n$.
  \item
     Prove: if $m$ divides $n$, then $\ZZ_m \leq \ZZ_n$, in the sense
     that $\ZZ_n$ is isomorphic to a subgroup of $\ZZ_m$.
   \item
     Prove $\ZZ_n/\ZZ_m \simeq \ZZ_{m/n}$. 
  \end{enumerate}
\end{exercise}

\begin{solution}
  \begin{enumerate}[(a)]
  \item
    First, the group $\ZZ$ is abelian, so $n\ZZ$ (and all subgroups)
    are normal, so $\ZZ / n\ZZ$ is a well defined group.  
    
    The idea here is really that the Euclidean algorithm, for all $a,
    n \in \ZZ$, there are $m \in \ZZ, r \in ZZ, 0 \leq r < n$, such
    that $ a = m n + r$, establishes a mapping from coset representatives of $ \ZZ / n\ZZ$,
    $a + n\ZZ$ and members of the concrete set representation $ r$ where $ 0 \leq r < n$.  
    
    The function $\phi$ takes the coset represented by $a$ to $r$ where $a = mn +r$ as above. 

    It is a group homomorphism.  Let $a + \ZZ, b + \ZZ \in \ZZ / n\ZZ$.

    Then, let $ a = mn + r$, $b = qn + s$, $ 0 \leq r, s < n$.  Let $\pmod$ be the operation in $\ZZ_n$. 
    $$
    \phi (\langle a \rangle) + \phi(\langle b \rangle) = (r \pmod s). 
    $$
    But
    $$ \phi(\langle a \rangle + \langle b \rangle ) = \phi (mn + r + qn + s) = r \pmod s
    $$
    So the group operation is preserved. 

    $$ \phi (\langle -a \rangle) + \phi (\langle a \rangle ) = \phi (\langle -mn - r + mn + r \rangle) = \phi(\langle 0n + 0 \rangle ) = 0
    $$
    That is, $\phi(-a)$ is the inverse of $\phi(a)$.

    Now, $\phi$ is one to one. Suppose $\phi(\langle a \rangle ) =
    \phi(\langle b \rangle )$.  That is, $a = mn +r$ and $b = qn + r$,
    and, so $\phi(a) = r = \phi(b)$. But then $r \in a + n\ZZ $ and $
    r \in b + n \ZZ$ so the cosets are also the same.

    This leaves only surjectivity to show.  Let $r \in \ZZ_n$.  $\phi(\langle r \rangle) = r$.  (For example,
    $0 n + r \in \langle r \rangle$.

    So $\phi$ is an isomorphism.  
    
  \item
    Let $n = d m$.  Let $G \leq \ZZ_n$ be generated by $d$ (when considered as 
    $1 + 1 + ... + 1$ $d$ times, for $1 \in \ZZ_n$).  Then $G \simeq \ZZ_m$.

    To show this, let $\phi \ZZ_m \to G$ be defined by $\phi(a) = da, a \in \ZZ_m$. $\phi$ is an
    isomorphism from $G$ to $\ZZ_m$.  

    \begin{itemize}
    \item
      Preserves group operations

      $\phi(a) + \phi(b) = da + db = d(a + b) = \phi(a + b)$ ($da + db = d(a +b) $ can
      be proven inductively on d if we aren't comfortable with just saying this is how
      numbers modulo $n$ work).

    \item
      Preserves group inverses

      $\phi(a) + \phi(-a) = \phi(da + d(-a)) = \phi(0) $ so $\phi(-a)$ is the inverse of $\phi(a)$.
      $da + d(-a) = 0$ can be seen by virtue that $\ZZ_n$ is Abelian; we just move each of the $d$ $a$'s
      next to each of the $d$ $-a$'s and they all sum to $0$.

    \item
      Onto

      For $da \in G$, $a < m$ because $da < n$.  $da < n$, so $\phi(a) = da$ as needed.

    \item
      1-1

      Let $a, b \in \ZZ_m$.  Assume $\phi(a) = \phi(b)$.  So $da = db$, but this implies $a=b$ (again by
      arithmetic or by induction on d).  
    \end{itemize}
  \item
    
  \end{enumerate}
\end{solution}

\begin{exercise}
  \begin{enumerate}[(a)]
  \item
    Let $G$ be a group.  An \textbf{automorphism} of $G$ is an
    isomorphism from $G$ to $G$. Let $\Aut(G)$ be the set of all
    isomorphisms from $G$ to $G$. In one (possibly long) sentence,
    explain why $\Aut(G)$ is a subgroup of $S_G$ (the group of all permutations of $G$).
  \item
    Let $G$ be a group.  Let $a \in G$.  Let $f_a : G \to G$ be
    conjugation by $a$, i.e. $f_a(x) = a x a^ {-1}$ for each $x \in G$.
    Prove that $f_a$ is an automorphism of $G$.
  \item
    Let $G$ be a group.  An \textbf{inner automorphism} of $G$ is an
    automorphism of the form $f_a$ (as above). Let $\Inn(G) $ be the
    set of all inner automorphisms of $G$. Prove $\Inn(G)$ is a subgroup of
    $ \Aut(G)$.
  \item
    Let $G$ be a group. Let $Z(G) = \{ x \in G : xy = yx \ \ \text{for
      all} \ y \in G \}$.  We call $Z(G)$ the \textbf(center) of $G$.
    Prove that $Z(G)$ is a normal subgroup of $G$ and that $\Inn(G) \simeq G / Z(G)$.
  \end{enumerate}
\end{exercise}
\begin{solution}
  \begin{enumerate}[(a)]
  \item
    An isomorphism on $G$ takes each member of $G$ to another member $G$ which corresponds to defining a permutation among the members of $G$ treated as arbitrary symbols
  \item
    \begin{itemize}
    \item
      $f_a$ preserves the group operation:
      
      $$
      f_a(x_1) f_a(x_2) = a x_1 a^{-1} a x_2 a ^{-1} = a x_1 x_2 a^ {-1} = f_a ( x_1 x_2 )
      $$
    \item
      and inverse:
      
      $$
      f_a(x) f(x ^ {-1}) = a x a^ { _1} a x ^ {-1} a = a x a ^ {-1} a x ^ {-1} a ^ {-1} = a x x ^ {-1} a ^ {-1} = a 1 a ^{-1} = f_a(1)
      $$
      
    \item
      $f_a$ is a 1-1 map from $G$ to $G$:
      
      Suppose $f_a(x_1) = f_a(x_2)$
      
      $$
      a x_1 a^ {-1} = a x_2 a ^ {-1}
      $$
      multiplying on the left by $a^{-1}$:
      $$
      x_1 a^ {-1} = x_2 a ^ {-1}
      $$
      and on the right by $a$:
      $$
      x_1 = x_2
      $$
    \item
      And onto; given $x \in G$, $ y= a^ {-1} x a \in G$ also, as $G$ is a group.
      But then
      $$
      f_a(y) = a y a ^ {-1} = a a ^ {-1} x a a ^ {-1} = x
      $$
    \end{itemize}
    \qed
  \item
    Prove that $\Inn(G)$ is a subgroup of $\Aut(G)$
    \begin{itemize}
    \item
      $\Inn(G)$ is non-empty: the identity map $f_1 : x \to x$ is in $\Inn(G)$.
    \item
      $\Inn(G)$ is closed under composition.  Let $x \in G$.  
      
      $$
      f_a \circ f_b (x) = f_a(bxb^{-1}) = a b x b^{-1} a^{-1} = (ab) x (ab)^ {-1} = f_{ab} (x)
      $$
      
      So $f_a \circ f_b = f _ {ab} \in \Inn(G)$.
      
    \item
      $\Inn(G)$ is closed under inverses; let $x \in G$.
      
      $$
      f_a \circ f_{a^{-1}} (x)  = f_a(a ^ {-1} x a) = a a^ {-1} x a a^{-1} = x = 1 x 1 = f_{1}(x)
      $$
      
      So $f_a ^ {-1} = f_{a^{-1}} \in \Inn(G)$.
      
      \qed
      
    \item
      Prove that the center is a normal subgroup.
      \begin{itemize}
      \item
        Non empty: $1 \in Z(G)$. ( $1y = y1$, for $y \in G$).
      \item
        Closed under operation:
        
        Let $x, y \in Z(G)$.  Given $z \in G$,
        
        $$
        z (xy) = (zx) y = (xz) y = x (zy) = x (yz) = xy(z)
        $$
        
        So $xy \in Z(G)$.
      \item
        Closed under inverses:
        
        Let $x \in Z(G)$.  Given $z \in G$,
        \begin{align*}
          xz & = zx & \\
          x^{-1} x z & = x^{-1} z x & \\
          z & = x^{-1} z x & \\
          z x ^ {-1} & = x^{-1} z x x ^ { -1} & \\
          z x ^ {-1} = x^ { -1} z
        \end{align*}
        
        So $x^{-1} \in Z(G)$.
        
      \item
        Normal.
        
        Given $ g \in G$, $c \in Z(G)$:
        
        $$
        g c g ^ {-1} = c g g^{-1} = c \in Z(G)
        $$
        
        Since $g,c $ were arbitrary, so $Z(G) \trianglelefteq G$.
        
        \qed
        
        Prove that $ \Inn(G) \simeq G / Z(G)$
        
        Let $\phi : G \to \Inn(G)$ be defined as $\phi (g) = f_g$ where
        $f_g$ is as above.  We will show that $\Ker(\phi) = Z(G)$ establishing
        the isomorphism (and the normality as well, making the prior argument redundant).
        
        $ \Ker(\phi) = \{ f_g \in \Inn(G) : \phi(f_g) = f_1 \}  $
        
        If $\phi(f_g) = f_1$, that means for all $x \in G$, $f_g(x) = f_1(x)$ or
        
        $$
        gxg^{-1} = 1x1^{-1}
        $$
        
        Simplifying:
        
        $$
        gx = xg
        $$
        
        Which since $x$ was arbitrary $x \in G$ is the criterion for $ g \in Z(G)$.
        
        That is, $\phi ^ {-1}(1) \subseteq  Z(G)$.
        
        To see the other direction, let $z \in Z(G)$.  $\phi(z) = f_z$.
        
        Applying that to arbitrary $x \in G$, $f_z(x) = zxz^{-1} = x$, by definition of Z(G).
        
        So $f_z = f_1$ and $\phi^{-1} (1) = Z(G)$.
        
        \qed
        
      \end{itemize}
    \end{itemize}
  \end{enumerate}
\end{solution}
  
\begin{exercise}
  Let $G$ be a group with a normal subgropup $N$ of order 5 such that $G/N$ is isomorphic to the
  symmetric group $S_4$.  Prove:
  \begin{enumerate}[(a)]
  \item
    $|G| = 120$
  \item
    G has a normal subgroup of order $20$.  
  \end{enumerate}
\end{exercise}
\begin{solution}
  \begin{enumerate}[(a)]
  \item
    If $G/N \simeq S_4$, the $|G| = |N| [G:N]$ by Lagrange's
    Theorem.  The index of N in G is the size of the image of $G$
    under the natural homomorphism $\phi: G \to G/N$.  We are told
    $G/N$ is isomorphic to the symmetric group $S_4$, so the index is the
    order of $S_4$ which is $4!$ or $24$.  So $|G| = 5 \times 24 = 120$.
  \item
    $A_4$ has a normal subgroup of order $4$, the group of double
    transpositions, which we saw in homework 2 (although did not
    prove normality at the time,
    https://groupprops.subwiki.org/wiki/Symmetric\_group:S4 seems to
    show it.)  By the Correspondence Theorem from class, the normality of $N$
    in $G$ gives us a bijection from subgroups of $G$ containing $N$
    and subgroups of $G/N$.  $G/N$ is $S_4$ and contains this normal
    subgroup of order $4$, $K$, so we can conclude there is a normal subgroup H
    of order $ |N| \times  |K| = 5 \times 4 = 20$.  
  \end{enumerate}
\end{solution}


\begin{comment}
  \begin{exercise}
    problem
  \end{exercise}
  \begin{solution}
    \begin{enumerate}[(a)]
    \item
      first answer
    \end{enumerate}
  \end{solution}
\end{comment}

\end{document}
