\documentclass[11pt,oneside]{article}
\usepackage[hmargin=1in,vmargin=1in]{geometry}               % See geometry.pdf to learn the layout options. There are lots.
\geometry{letterpaper}                   % ... or a4paper or a5paper or ...
%\geometry{landscape}                % Activate for for rotated page geometry
%\usepackage[parfill]{parskip}    % Activate to begin paragraphs with an empty line rather than an indent
\usepackage{graphicx}
\usepackage{amssymb}
\usepackage{epstopdf}
\usepackage{url}
%\usepackage{verbatim}
\usepackage{comment}
\specialcomment{solution}{\textbf{Solution. }}{}
%\excludecomment{solution}    %uncomment to remove solutions.

%\usepackage{enumerate}

%Use the enumitem package instead of enumerate
\usepackage[shortlabels]{enumitem}
%\usepackage{enumitem}
%then it will support the same suntax as the enumerate package.
%The enumerate package does not provide any extra configurations other than the label.

%\setlist[enumerate]{topsep=0pt,itemsep=-1ex,partopsep=1ex,parsep=1ex}
\setlist[enumerate]{topsep=0pt,partopsep=0pt}

\DeclareGraphicsRule{.tif}{png}{.png}{`convert #1 `dirname #1`/`basename #1 .tif`.png}
\usepackage{amsmath,amsthm,amscd,amssymb}
\usepackage{latexsym}
\usepackage[colorlinks,citecolor=red,pagebackref,hypertexnames=false]{hyperref}
\numberwithin{equation}{section}

\theoremstyle{definition}
\newtheorem{exercise}{Exercise}
%\newtheorem{solution}{Solution}
\newtheorem*{defn}{Definition}
\newtheorem*{claim}{Claim}

\def\calA{\mathcal{A}}
\def\calB{\mathcal{B}}
\def\calC{\mathcal{C}}
\def\calT{\mathcal{T}}
\def\OR{\overline{\mathbb{R}}}
\def\RR{\mathbb{R}}
\def\CC{\mathbb{C}}
\def\FF{\mathbb{F}}
\def\QQ{\mathbb{Q}}
\def\ZZ{\mathbb{Z}}
\def\NN{\mathbb{N}}
%\def\NN{\mathbb{Z}_{> 0}}
\def\Nzero{\mathbb{Z}_{\geq 0}}
\def\EE{\mathbb{E}}
\def\PP{\mathbb{P}}
\def\supp{\mathrm{supp}}
\def\diam{\mathrm{diam}}
\def\sp{\mathrm{span}}
\def\ker{\mathrm{ker}}
%\def\sp{\mathrm{span}} %messes up align enviroment
\newcommand{\rbr}[1]{\left( {#1} \right)}
\newcommand{\sbr}[1]{\left[ {#1} \right]}
\newcommand{\cbr}[1]{\left\{ {#1} \right\}}
\newcommand{\abr}[1]{\left\langle {#1} \right\rangle}
\newcommand{\abs}[1]{\left| {#1} \right|}
\newcommand{\norm}[1]{\left\|#1\right\|}
\def\one{\mathbf{1}}
\DeclareMathOperator*{\esssup}{ess\,sup}
\newcommand*\wc{{}\cdot{}}
%\newcommand*\wc{ \, \cdot \,}
%wc for wildcard
\renewcommand{\Re}{\operatorname{Re}}
\renewcommand{\Im}{\operatorname{Im}}
\newcommand{\sgn}{\textup{sgn\,}}


\setlength{\parindent}{0pt}
\setlength{\parskip}{11pt}


%\title{\parbox{14cm}{\centering{  Interior points of circle and sphere packings}}}
\begin{document}

\textbf{HW 4 - MATH 221A - Fall 2023 - Chris Lane}

\begin{exercise}
  \begin{enumerate}[(a)]
  \item
    Let $\ZZ$ be the integers and $n\ZZ$ be the set of integer
    multiples of $n$.  Show that $\ZZ / n\ZZ$ is isomorphic to $\ZZ_n$,
    the additive group of integers module $n$.  (This is not quite a
    tautology if we view $\ZZ_n$ concretely as the set $\{0, 1, 2, ..., n-1\}$,
    with the sums and differences reduced module $n$.
  \item
     Prove: if $m$ divides $n$, then $\ZZ_m \leq \ZZ_n$, in the sense
     that $\ZZ_n$ is isomorphic to a subgroup of $\ZZ_m$.  
  \end{enumerate}
\end{exercise}

\begin{solution}
  \begin{enumerate}[(a)]
  \item
    First, the group $\ZZ$ is abelian, so $n\ZZ$ (and all subgroups)
    are normal, so $\ZZ / n\ZZ$ is a well defined group.  
    
    The idea here is really that the Euclidean algorithm, for all $a,
    n \in \ZZ$, there are $m \in \ZZ, r \in ZZ, 0 \leq r < n$, such
    that $ a = m n + r$, establishes a mapping from coset representatives of $ \ZZ / n\ZZ$,
    $a + n\ZZ$ and members of the concrete set representation $ r$ where $ 0 \leq r < n$.  
    
    The function $\phi$ takes the coset represented by $a$ to $r$ where $a = mn +r$ as above. 

    It is a group homomorphism.  Let $a + \ZZ, b + \ZZ \in \ZZ / n\ZZ$.

\def\pmod{+'}
    Then, let $ a = mn + r$, $b = qn + s$, $ 0 \leq r, s < n$.  Let $\pmod$ be the operation in $\ZZ_n$. 
    $$
    \phi (\langle a \rangle) + \phi(\langle b \rangle) = (r \pmod s). 
    $$
    But
    $$ \phi(\langle a \rangle + \langle b \rangle ) = \phi (mn + r + qn + s) = r \pmod s
    $$
    So the group operation is preserved. 

    $$ \phi (\langle -a \rangle) + \phi (\langle a \rangle ) = \phi (\langle -mn - r + mn + r \rangle) = \phi(\langle 0n + 0 \rangle ) = 0
    $$
    That is, $\phi(-a)$ is the inverse of $\phi(a)$.

    Now, $\phi$ is one to one. Suppose $\phi(\langle a \rangle ) =
    \phi(\langle b \rangle )$.  That is, $a = mn +r$ and $b = qn + r$,
    and, so $\phi(a) = r = \phi(b)$. But then $r \in a + n\ZZ $ and $
    r \in b + n \ZZ$ so the cosets are also the same.

    This leaves only surjectivity to show.  Let $r \in \ZZ_n$.  $\phi(\langle r \rangle) = r$.  (For example,
    $0 n + r \in \langle r \rangle$.

    So $\phi$ is an isomorphism.  
    
  \item
    Let $n = d m$.  Let $G \leq \ZZ_n$ be generated by $d$ (when considered as
    $1 + 1 + ... + 1$ $d$ times).
    \begin{claim}
      
    \end{claim}
    \begin{proof}
    \end{proof}

  \end{enumerate}
\end{solution}

\begin{comment}
  \begin{exercise}
    problem
  \end{exercise}
  \begin{solution}
    \begin{enumerate}[(a)]
    \item
      first answer
    \end{enumerate}
  \end{solution}
\end{comment}

\end{document}
