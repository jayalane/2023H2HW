\documentclass[11pt,oneside]{article}
\usepackage[hmargin=1in,vmargin=1in]{geometry}               % See geometry.pdf to learn the layout options. There are lots.
\geometry{letterpaper}                   % ... or a4paper or a5paper or ...
%\geometry{landscape}                % Activate for for rotated page geometry
%\usepackage[parfill]{parskip}    % Activate to begin paragraphs with an empty line rather than an indent
\usepackage{graphicx}
\usepackage{datetime}
\usepackage{amssymb}
\usepackage{epstopdf}
\usepackage{url}
%\usepackage{verbatim}
\usepackage{comment}
\specialcomment{solution}{\textbf{Solution. }}{}
%\excludecomment{solution}    %uncomment to remove solutions.

%\usepackage{enumerate}

%Use the enumitem package instead of enumerate
\usepackage[shortlabels]{enumitem}
%\usepackage{enumitem}
%then it will support the same suntax as the enumerate package.
%The enumerate package does not provide any extra configurations other than the label.

%\setlist[enumerate]{topsep=0pt,itemsep=-1ex,partopsep=1ex,parsep=1ex}
\setlist[enumerate]{topsep=0pt,partopsep=0pt}

\DeclareGraphicsRule{.tif}{png}{.png}{`convert #1 `dirname #1`/`basename #1 .tif`.png}
\usepackage{amsmath,amsthm,amscd,amssymb}
\usepackage{latexsym}
\usepackage[colorlinks,citecolor=red,pagebackref,hypertexnames=false]{hyperref}
\numberwithin{equation}{section}

\theoremstyle{definition}
\newtheorem{exercise}{Exercise}
%\newtheorem{solution}{Solution}
\newtheorem*{defn}{Definition}
\newtheorem*{claim}{Claim}

\def\pmod{+'}
\def\boldr{\boldsymbol{r}}
\def\boldc{\boldsymbol{c}}

\def\calA{\mathcal{A}}
\def\calB{\mathcal{B}}
\def\calC{\mathcal{C}}
\def\calT{\mathcal{T}}
\def\OR{\overline{\mathbb{R}}}
\def\RR{\mathbb{R}}
\def\CC{\mathbb{C}}
\def\FF{\mathbb{F}}
\def\QQ{\mathbb{Q}}
\def\ZZ{\mathbb{Z}}
\def\NN{\mathbb{N}}
%\def\NN{\mathbb{Z}_{> 0}}
\def\Nzero{\mathbb{Z}_{\geq 0}}
\def\EE{\mathbb{E}}
\def\PP{\mathbb{P}}
\def\supp{\mathrm{supp}}
\def\diam{\mathrm{diam}}
\def\sp{\mathrm{span}}
\def\ker{\mathrm{ker}}
\def\Aut{\operatorname{Aut}}
\def\Inn{\operatorname{Inn}}
\def\Ker{\operatorname{Ker}}
%\def\sp{\mathrm{span}} %messes up align enviroment
\newcommand{\Mod}{\ (\mathrm{mod}\ )}
\newcommand{\rbr}[1]{\left( {#1} \right)}
\newcommand{\sbr}[1]{\left[ {#1} \right]}
\newcommand{\cbr}[1]{\left\{ {#1} \right\}}
\newcommand{\abr}[1]{\left\langle {#1} \right\rangle}
\newcommand{\abs}[1]{\left| {#1} \right|}
\newcommand{\norm}[1]{\left\|#1\right\|}
\def\one{\mathbf{1}}
\DeclareMathOperator*{\esssup}{ess\,sup}
\newcommand*\wc{{}\cdot{}}
%\newcommand*\wc{ \, \cdot \,}
%wc for wildcard
\renewcommand{\Re}{\operatorname{Re}}
\renewcommand{\Im}{\operatorname{Im}}
\newcommand{\sgn}{\textup{sgn\,}}


\setlength{\parindent}{0pt}
\setlength{\parskip}{11pt}

%\title{\parbox{14cm}{\centering{  Interior points of circle and sphere packings}}}
\begin{document}

\textbf{HW 7 - MATH 221A - Fall 2023 - Chris Lane}

Date: \hhmmsstime{} \ \today \ \ Git hash: 
\input{/home/jayalane/hw/.git/refs/heads/main}

\begin{exercise}
  Prove that a cyclic group of order $n$ has the presentation:
  \[
  \langle a | a^n = 1 \rangle
  \]
\end{exercise}
\begin{solution}

  We'll use the Van Dyck Theorem.  We start with showing that the
  cyclic group $C_n$ is generated by the alphabet $a$.  Then we will
  show the cyclic group, $C_n$, satisfies the relation.

  The definition of a cyclic group is that it is generated by one element.  
  
  Moreover, if $a$ generates a finite group of order $n$ then $ a ^ n = 1$
  by one of the first results of class, and is basically a restatement of
  ``a group generated by a single element''.

  To obtain the secondary consequences of the theorem, we need to show
  that $ | \langle a | a^n = 1 \rangle | \leq |C_n|$; we'll prove that
  the sentences in the group $G / \overline K $ are of the form (where
  $G$ is the free group over ${a}$, and $\overline K$ is the smallest
  normal subgroup containing the relation $K = \{ a^n\}$):
  \[
  a ^ k, \textrm{ for some } 0 \leq k < n
  \]

  First of all, since $a ^ n = 1$, $a ^ {n-1} \cdot a = 1$, so any $a^{-1}$ in a sentence
  can be replaced with $n-1 $ copies of $a$.  So all the sentences end up being
  a sentence of zero of more $a$.  Next, each run of $n$ $a$ in row can be
  deleted from the sentence, so the number of $a$ in the distinct sentences ranges are
  from $0$ to $n-1$.  So $|G / \overline K| = n = |C_k|$, so the
  additional results of the Van Dyck theorem apply, and the two groups are isomorphic.  
\end{solution}


\begin{exercise}
  Let $Q$ be the quaternion group, (i.e. $Q = \{1, i, j, k, -1 , -i,
  -j, -k\}$, with multiplication defined by $(-1)^2 = 1$,
  $i^2 = j^2 = k^2 = -1$, $ij = k$, $jk =i$, $ki =j$,
  $ji = -k$, $kj =-i$, $ik =-j$).  

  Prove that Q has a presentation $H = \langle a,b |  a^4 =1 , a^2 = b^2, ab= ba^{-1}\rangle$.
  
\end{exercise}
\begin{solution}

  Define the mapping $f : H \to Q$ by

  \begin{align*}
    f(a) &= i \\
    f(b) &= j \\
    f(1) &= 1
  \end{align*}

  To show that the group $Q$ is generated by $f(a)$ and $f(b)$, we have to show all $q \in Q$ is obtained from sentences in $a$ (or $i$) and
  $b$ (or $j$):

  \begin{align*}
      a^0 b^0 & = 1\\
    a^1 b^0 & =  i\\
    a^2 b^0 & = i ^ 2 = -1\\
    a^3 b^0 & = i^3 = -i \\
    a^0 b^1 & = j\\
    a^1 b^1 & = i j = k \\
    a^2 b^1 & = -1 j =  -j\\
    a^3 b^1 & = -i j = -k\\
  \end{align*}
  
  Now, the given presentation is equivalent to $H = \langle a,b |  a^4 =1 , a^2 (b^ {-1}) ^2 = 1,  abab^{-1} = 1 \rangle$.

  These relations are satisfied:

  \begin{align*}
    f(a^4) & = i ^ 4 = (i \times i) \times (i \times i) = -1 \times -1 = 1 \\
    f(a^2 \times (b ^ 2) ^ {-1}) & = (i \times i ) \times (j \times j) ^ {-1} = -1 \times (-1 ) ^ {-1} = 1 \\
    f(a \times b \times a ^ {-1} \times b) &= i \times j \times -i \times j = k \times -k = 1 
  \end{align*}

  Now, we have to show that $ |G/{\overline K}| \leq |Q|$ and we are done.

  All words in the presentation have $a$, $a^{-1}$, $b$, $b^{-1}$, in some order.
  We can replace all $a^{-1}$ with ${a^3}$ because $a^4 =1$.  Likewise, since $b^2 = -1$,
  $ b^4 =1$, and we can replace all the $b^{-1}$ with $b^3$.

  So all the words are equivalent to words over $\{a, b\}$.

  Similarly, any consecutive runs of 4 $a$ or 4 $b$ can be deleted, so we end up with equivalent
  sentences like $a^{n_1}b^{n_2}a^{n_3}b^{n_4} ... b^{n_N}$ for coefficients $n_1, ... , n_N$, with
  $n_i \in \{ 0, 1, 2 , 3\} $  and $N \in \NN$.

  Now, we can repeatedly swap all the $ba$ pairs into $a^3b$ since:

  \begin{align*}
    ab & = b a ^ {-1} \\
    aba & = b \\
    ba & = a ^ { -1 } b\\
    ba & = a ^ 3 b
  \end{align*}

  Possibly re-applying the ``delete consecutive runs of 4 of the same letter'', we end up with sentences of the form:

  \[
  a ^ i b ^ j, \ \ (\textrm{ where } 0 \leq i, j, \leq 3)
  \]

  However, since $a^2 = b^2$, we can actually guarantee $j \leq 1$.  If $j > 1$, we swap 2 $b$ for $a$.  Giving us
  the following eight sentences:

  \begin{align*}
    a^0 b^0 & \ \ ( \mapsto 1)\\
    a^1 b^0 & \ \ ( \mapsto i)\\
    a^2 b^0 & \ \ ( \mapsto -1)\\
    a^3 b^0 & \ \ ( \mapsto -i)\\
    a^0 b^1 & \ \ ( \mapsto j)\\
    a^1 b^1 & \ \ ( \mapsto k)\\
    a^2 b^1 & \ \ ( \mapsto -j)\\
    a^3 b^1 & \ \ ( \mapsto -k)\\
  \end{align*}

  These are all mapped to different members of $Q$ under $f$ so they are different in the source group, and
  so $|G / \overline K| = 8 = |H|$ so the groups are in fact isomorphic. \qed
  
\end{solution}

\begin{exercise}
Prove that $S_3$ has a representation $H = \langle a, b | a^3 =1, b^2 = 1, ba = a^{-1}b \rangle $.  
\end{exercise}
\begin{solution}

  First, we must show there are $a, b \in S_3$ such that they generate $S_3$.  Let $a = (1 2 3)$ and $b = (1 2)$.

  Then

  \begin{align*}
    (1) & = 1 \\
    (1 2 ) & = b \\
    (1 3) & = a^2 b \\
    (2 3) & = ab \\
    (1 2 3 ) & = a \\
    (1 3 2) & = a^2
  \end{align*}

  So all six members are thus generated.

  In addition, we can verify that the relations hold in $S_3$:

  \begin{itemize}
  \item
    $a^3 = (1 2 3)^3 = (1) = 1$, as $(1 2 3)$ is a 3-cycle
  \item
    $b^2 = (1 2)^2 = (1) = 1$ as $(1 2 )$ is a 2-cycle
  \item
    $ba = (1 2 3) (1 2) = (1 3) (2) = a^2 b = a^{-1} b$
  \end{itemize}

  So the conditions of the Van Dyke theorem are fulfilled; we just need to show that
  $|G / \overline K | = |S_3|$.

  To do this we can show that all sentences reduce down to six distinct sentences, of the following form:

  \[
  a^i b^j, \textrm{ where } i \in \{ 0, 1, 2 \}, j \in \{ 0, 1 \}
  \]


  All words in the presentation have $a$, $a^{-1}$, $b$, $b^{-1}$, in some order.
  We can replace all $a^{-1}$ with ${a^2}$ because $a^3 =1$.  Likewise, since $b^2 = 1$,
  *and we can replace all the $b^{-1}$ with $b$.

  So all the words are equivalent to words over $\{a, b\}$.

  
  So for any word, we can remove repeating runs of 3 $a$ or 2 $b$ as in Excercise 1 and 2.  Using $ba = a^3 b$ form of the
  third relation, we can shuffle all the $a$ to the beginning of the word; then removing all the repeated runs again,
  we end up with the desired form.

  The six values, $\{ 1, a, a^2, b, ab, a^2b \}$ all map onto different members of $S_3$ (using the same computation
  showing that the mapping is surjective), so the   $|G / \overline K | = |S_3|$, and the map is an isomorphism.  

\end{solution}

\begin{comment}
  \begin{exercise}
    problem
  \end{exercise}
  \begin{solution}
    \Begin{enumerate}[(a)]
    \item
      first answer
    \end{enumerate}
  \end{solution}
\end{comment}

\end{document}
