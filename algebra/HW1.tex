\documentclass[11pt,oneside]{article}
\usepackage[hmargin=1in,vmargin=1in]{geometry}               % See geometry.pdf to learn the layout options. There are lots.
\geometry{letterpaper}                   % ... or a4paper or a5paper or ...
%\geometry{landscape}                % Activate for for rotated page geometry
%\usepackage[parfill]{parskip}    % Activate to begin paragraphs with an empty line rather than an indent
\usepackage{graphicx}
\usepackage{amssymb}
\usepackage{epstopdf}
\usepackage{url}
%\usepackage{verbatim}
\usepackage{comment}
\specialcomment{solution}{\textbf{Solution. }}{}
%\excludecomment{solution}    %uncomment to remove solutions.

%\usepackage{enumerate}

%Use the enumitem package instead of enumerate
\usepackage[shortlabels]{enumitem}
%\usepackage{enumitem}
%then it will support the same suntax as the enumerate package.
%The enumerate package does not provide any extra configurations other than the label.

%\setlist[enumerate]{topsep=0pt,itemsep=-1ex,partopsep=1ex,parsep=1ex}
\setlist[enumerate]{topsep=0pt,partopsep=0pt}

\DeclareGraphicsRule{.tif}{png}{.png}{`convert #1 `dirname #1`/`basename #1 .tif`.png}
\usepackage{amsmath,amsthm,amscd,amssymb}
\usepackage{latexsym}
\usepackage[colorlinks,citecolor=red,pagebackref,hypertexnames=false]{hyperref}
\numberwithin{equation}{section}

\theoremstyle{definition}
\newtheorem{exercise}{Exercise}
%\newtheorem{solution}{Solution}
\newtheorem*{defn}{Definition}


\def\calA{\mathcal{A}}
\def\calB{\mathcal{B}}
\def\calC{\mathcal{C}}
\def\calT{\mathcal{T}}
\def\OR{\overline{\mathbb{R}}}
\def\RR{\mathbb{R}}
\def\CC{\mathbb{C}}
\def\FF{\mathbb{F}}
\def\QQ{\mathbb{Q}}
\def\ZZ{\mathbb{Z}}
\def\NN{\mathbb{N}}
%\def\NN{\mathbb{Z}_{> 0}}
\def\Nzero{\mathbb{Z}_{\geq 0}}
\def\EE{\mathbb{E}}
\def\PP{\mathbb{P}}
\def\supp{\mathrm{supp}}
\def\diam{\mathrm{diam}}
\def\sp{\mathrm{span}}
\def\ker{\mathrm{ker}}
%\def\sp{\mathrm{span}} %messes up align enviroment
\newcommand{\rbr}[1]{\left( {#1} \right)}
\newcommand{\sbr}[1]{\left[ {#1} \right]}
\newcommand{\cbr}[1]{\left\{ {#1} \right\}}
\newcommand{\abr}[1]{\left\langle {#1} \right\rangle}
\newcommand{\abs}[1]{\left| {#1} \right|}
\newcommand{\norm}[1]{\left\|#1\right\|}
\def\one{\mathbf{1}}
\DeclareMathOperator*{\esssup}{ess\,sup}
\newcommand*\wc{{}\cdot{}}
%\newcommand*\wc{ \, \cdot \,}
%wc for wildcard
\renewcommand{\Re}{\operatorname{Re}}
\renewcommand{\Im}{\operatorname{Im}}
\newcommand{\sgn}{\textup{sgn\,}}


\setlength{\parindent}{0pt}
\setlength{\parskip}{11pt}


%\title{\parbox{14cm}{\centering{  Interior points of circle and sphere packings}}}
\begin{document}

\textbf{HW 1 - MATH 221A - Fall 2023 - Chris Lane}

%\begin{comment}
\begin{exercise}
Find the order of each element of $ \mathbb{Z} _ 6 $
\end{exercise}
%\end{comment}

\begin{solution}
\begin{enumerate}[(a)]
\item 
    $ \langle \overline 0 \rangle  = \{ \overline 0 \} $ so $ \lvert \langle \overline 0 \rangle \rvert = 1 $ It seems there is some ambiguity on the order of the identity, but setting it to $ 1 $ means each divisor including $1$ and $n$ of $n$ get their own subgroup.  
  \item
    $ \langle \overline 1 \rangle  = \{ \overline 0, \overline 1, \overline 2, \overline 3, \overline 4, \overline 5 \} $ so $ \lvert \langle \overline 1 \rangle \rvert = 6 $
  \item
    $ \langle \overline 2 \rangle  = \{ \overline 0, \overline  2, \overline  4 \} $ so $ \lvert \langle \overline 2 \rangle \rvert = 3 $
  \item
    $ \langle \overline 3 \rangle  = \{ \overline 0, \overline  3 \} $ so $ \lvert \langle \overline 3 \rangle \rvert = 2 $
  \item
    $ \langle \overline 4 \rangle  = \{ \overline 0, \overline  4, \overline  2 \} $ so $ \lvert \langle \overline 4 \rangle \rvert = 3 $
  \item
    $ \langle \overline 5 \rangle  = \{ \overline 0 , \overline 5 , \overline 4 , \overline 3 , \overline 2 , \overline 1 \} $ so $ \lvert \langle \overline 5 \rangle \rvert = 6 $
\end{enumerate}
\end{solution}








\end{document}
