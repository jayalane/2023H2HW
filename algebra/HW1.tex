\documentclass[11pt,oneside]{article}
\usepackage[hmargin=1in,vmargin=1in]{geometry}               % See geometry.pdf to learn the layout options. There are lots.
\geometry{letterpaper}                   % ... or a4paper or a5paper or ...
%\geometry{landscape}                % Activate for for rotated page geometry
%\usepackage[parfill]{parskip}    % Activate to begin paragraphs with an empty line rather than an indent
\usepackage{graphicx}
\usepackage{amssymb}
\usepackage{epstopdf}
\usepackage{url}
%\usepackage{verbatim}
\usepackage{comment}
\specialcomment{solution}{\textbf{Solution. }}{}
%\excludecomment{solution}    %uncomment to remove solutions.

%\usepackage{enumerate}

%Use the enumitem package instead of enumerate
\usepackage[shortlabels]{enumitem}
%\usepackage{enumitem}
%then it will support the same suntax as the enumerate package.
%The enumerate package does not provide any extra configurations other than the label.

%\setlist[enumerate]{topsep=0pt,itemsep=-1ex,partopsep=1ex,parsep=1ex}
\setlist[enumerate]{topsep=0pt,partopsep=0pt}

\DeclareGraphicsRule{.tif}{png}{.png}{`convert #1 `dirname #1`/`basename #1 .tif`.png}
\usepackage{amsmath,amsthm,amscd,amssymb}
\usepackage{latexsym}
\usepackage[colorlinks,citecolor=red,pagebackref,hypertexnames=false]{hyperref}
\numberwithin{equation}{section}

\theoremstyle{definition}
\newtheorem{exercise}{Exercise}
%\newtheorem{solution}{Solution}
\newtheorem*{defn}{Definition}


\def\calA{\mathcal{A}}
\def\calB{\mathcal{B}}
\def\calC{\mathcal{C}}
\def\calT{\mathcal{T}}
\def\OR{\overline{\mathbb{R}}}
\def\RR{\mathbb{R}}
\def\CC{\mathbb{C}}
\def\FF{\mathbb{F}}
\def\QQ{\mathbb{Q}}
\def\ZZ{\mathbb{Z}}
\def\NN{\mathbb{N}}
%\def\NN{\mathbb{Z}_{> 0}}
\def\Nzero{\mathbb{Z}_{\geq 0}}
\def\EE{\mathbb{E}}
\def\PP{\mathbb{P}}
\def\supp{\mathrm{supp}}
\def\diam{\mathrm{diam}}
\def\sp{\mathrm{span}}
\def\ker{\mathrm{ker}}
%\def\sp{\mathrm{span}} %messes up align enviroment
\newcommand{\rbr}[1]{\left( {#1} \right)}
\newcommand{\sbr}[1]{\left[ {#1} \right]}
\newcommand{\cbr}[1]{\left\{ {#1} \right\}}
\newcommand{\abr}[1]{\left\langle {#1} \right\rangle}
\newcommand{\abs}[1]{\left| {#1} \right|}
\newcommand{\norm}[1]{\left\|#1\right\|}
\def\one{\mathbf{1}}
\DeclareMathOperator*{\esssup}{ess\,sup}
\newcommand*\wc{{}\cdot{}}
%\newcommand*\wc{ \, \cdot \,}
%wc for wildcard
\renewcommand{\Re}{\operatorname{Re}}
\renewcommand{\Im}{\operatorname{Im}}
\newcommand{\sgn}{\textup{sgn\,}}


\setlength{\parindent}{0pt}
\setlength{\parskip}{11pt}


%\title{\parbox{14cm}{\centering{  Interior points of circle and sphere packings}}}
\begin{document}

\textbf{HW 1 - MATH 221A - Fall 2023 - Chris Lane}

%\begin{comment}
\begin{exercise}
\begin{enumerate}[(1)]
\item 
Find the order of each element of $ \ZZ _ 6 $
\item 
  List all subgroups of $ \ZZ _ 6 $
\end{enumerate}
\end{exercise}
%\end{comment}

\begin{solution}
\begin{enumerate}[(1)]
\item 
\begin{enumerate}[(a)]
\item 
    $ \langle \overline 0 \rangle  = \{ \overline 0 \} $ and $ \lvert \langle \overline 0 \rangle \rvert = 1 $ It seems there is some ambiguity on the order of the identity, but setting it to $ 1 $ means each divisor including $1$ and $n$ of $n$ get their own subgroup.  
  \item
    $ \langle \overline 1 \rangle  = \{ \overline 0, \overline 1, \overline 2, \overline 3, \overline 4, \overline 5 \} $ so $ \lvert \langle \overline 1 \rangle \rvert = 6 $ (the set is just 0, 1, 1 + 1, 1 + 1 + 1, etc. till it repeats).
  \item
    $ \langle \overline 2 \rangle  = \{ \overline 0, \overline  2, \overline  4 \} $ so $ \lvert \langle \overline 2 \rangle \rvert = 3 $
  \item
    $ \langle \overline 3 \rangle  = \{ \overline 0, \overline  3 \} $ so $ \lvert \langle \overline 3 \rangle \rvert = 2 $
  \item
    $ \langle \overline 4 \rangle  = \{ \overline 0, \overline  4, \overline  2 \} $ so $ \lvert \langle \overline 4 \rangle \rvert = 3 $
  \item
    $ \langle \overline 5 \rangle  = \{ \overline 0 , \overline 5 , \overline 4 , \overline 3 , \overline 2 , \overline 1 \} $ so $ \lvert \langle \overline 5 \rangle \rvert = 6 $
\end{enumerate}
\item 
  One could look at the above subgroups generated by the elements, but the theorem from class relating the divisors to the sbugroups allows us to do less work.  
\begin{enumerate}[(a)]
\item 
    The divisor $ 6 $ gives us the trival subgroup: $ \langle \overline 0 \rangle =  \{ \overline 0 \} $
\item
    The divisor $ 1 $ gives us the full group as an improper subgroup: $ \langle \overline 1 \rangle  = \{ \overline 0, \overline 1, \overline 2, \overline 3, \overline 4, \overline 5 \} $
  \item
    The divisor $ 2 $ gives us a subgroup of ordrer $ 3 = 6 \div 2 $: $\langle \overline 2 \rangle  = \{ \overline 0, \overline  2, \overline  4 \} $
  \item
    The divisor $3 $ gives us a subgroup of order $ 2 = 6 \div 3 $: $ \langle \overline 3 \rangle  = \{ \overline 0, \overline  3 \} $ 

\end{enumerate}

\end{enumerate}
\end{solution}

\begin{exercise}
\begin{enumerate}[(1)]
\item 
  Prove: If $H$ and $K$ are proper subgroups of a group $G$, and
  neither contains the other, then $H \bigcup K $ is not a subgroup of
  $G$.  (Proper means not equal to the whole group $G$.)  Moreover,
  if $h \notin K$ and $k \notin H$, then $ hk \notin H \bigcup K$. 
\item 
  Prove: If $H$ and $K$ are proper subgroups of a group $G$, then $H
  \bigcup K \neq G $.  (Proper means not equal to
  the whole group $G$.)
\end{enumerate}  
\end{exercise}

\begin{solution}
  \begin{enumerate}[(1)]
    \item
  The property that fails is completion under group multiplication.
  Let $k \in K$ and $ h \in H$. Since they are not subsets of each
  other, we can pick them such that $ k \notin H$ and $h \notin K$.
  Consider $h k$; is it in $H$?  But then $ h k \in H $ would mean $ h
  ^ {-1} ( h k ) \in H$ which would mean that $ ( h ^ { -1 } h ) k \in
  H $ which would mean $ e k \in H $ which would imply $ k \in H $
  which is not true, so $ h k \notin H$.

  Likewise, can $ h k $ be in $K$?  But right multiplication by $ k ^
  { -1 } $ similarly shows that this implies $ h \in K $ which is not
  true; so neither case is possible and the element $ h k \notin H
  \bigcup K $ and $ H \bigcup K $ is not closed under group
  multiplication and so not a group.
\item
  There are two cases to consider, one when one of $H$ or $K$ is a subset of
  the other.  Assume, without loss of generality, that $ K \subset H $.
  Then, $ K \bigcup H = H$.  But since $H$ is a proper subgroup, there is a
  $ g \in G$ where $g \notin H$, so therefore $H \bigcup K \neq G$.

  The other case is that neither is a subset of the other, in which
  case the above exercise applies, and the element $ hk, h \notin K, k
  \notin H$ is not in $ H \bigcup K$, and therefore $H \bigcup K \neq G$.
  \end{enumerate}

\end{solution}


\begin{exercise}
  Give an example of an infinite group that has a non-trivial finite subgroup.  
\end{exercise}

\begin{solution}
  Let $G$ be the group of rotations of the unit circle, with
  composition as the group operation.  Then $G$ is a group of
  uncountable order, as it has a unique member for each $ \theta \in [
    0, 2 \pi ) $ However, for each $ n \in \NN$, it has a
    subgroup of that order, represented by a rotation of $ \theta = \dfrac{2 \pi}{n}$.
    That rotation has the property that repeating
    it $n$ times will return the circle to its initial position, an
    operation that is the identity for this group.
\end{solution}

\begin{exercise}
  Prove that every infinite group has a non-trivial proper subgroup.
\end{exercise}

\begin{solution}
  Let $G$ be an infinite group.  It is infinite, so there is at least
  one element $g$ not equal to $e$.  Take $ \langle g \rangle$.  There
  are two cases.

  If $ \langle g \rangle = G $, then $ \langle g ^ 2 \rangle $
  is a proper non-trivial subgroup.  It is non-trivial
  because $g ^2 \neq e$ since otherwise $g$ would not generate the
  entire group $G$.  It is proper because $ g \notin \langle g ^ 2 \rangle $,
  or else $ \langle g \rangle $ would be finite (as there would be
  a $d$ such that $ ( g g) ^ d = g$ meaning $ g ^ {2 d} = g$ meaning $
  g ^ { 2d - 1 } = e $ but that contradicts $g$ generating $G$.

  On the other hand, if $ \langle g \rangle \neq G$, then $ \langle g \rangle $ is the
  proper non-trivial subgroup right there.  
  

\end{solution}

\begin{comment}
\begin{exercise}
  problem
\end{exercise}
\begin{solution}
\begin{enumerate}[(a)]
\item
  first answer
\end{enumerate}
\end{solution}
\end{comment}




\end{document}
