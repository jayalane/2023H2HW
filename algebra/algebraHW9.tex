\documentclass[11pt,oneside]{article}
\usepackage[hmargin=1in,vmargin=1in]{geometry}               % See geometry.pdf to learn the layout options. There are lots.
\geometry{letterpaper}                   % ... or a4paper or a5paper or ...
%\geometry{landscape}                % Activate for for rotated page geometry
%\usepackage[parfill]{parskip}    % Activate to begin paragraphs with an empty line rather than an indent
\usepackage{graphicx}
\usepackage{datetime}
\usepackage{amssymb}
\usepackage{epstopdf}
\usepackage{url}
%\usepackage{verbatim}
\usepackage{comment}
\specialcomment{solution}{\textbf{Solution. }}{}
%\excludecomment{solution}    %uncomment to remove solutions.

%\usepackage{enumerate}

%Use the enumitem package instead of enumerate
\usepackage[shortlabels]{enumitem}
%\usepackage{enumitem}
%then it will support the same suntax as the enumerate package.
%The enumerate package does not provide any extra configurations other than the label.

%\setlist[enumerate]{topsep=0pt,itemsep=-1ex,partopsep=1ex,parsep=1ex}
\setlist[enumerate]{topsep=0pt,partopsep=0pt}

\DeclareGraphicsRule{.tif}{png}{.png}{`convert #1 `dirname #1`/`basename #1 .tif`.png}
\usepackage{amsmath,amsthm,amscd,amssymb}
\usepackage{latexsym}
\usepackage[colorlinks,citecolor=red,pagebackref,hypertexnames=false]{hyperref}
\numberwithin{equation}{section}

\theoremstyle{definition}
\newtheorem{exercise}{Exercise}
\newtheorem{lemma}{Lemma}
\newtheorem{theorem}{Theorem}
%\newtheorem{solution}{Solution}
\newtheorem*{defn}{Definition}
\newtheorem*{claim}{Claim}

\def\pmod{+'}
\def\boldr{\boldsymbol{r}}
\def\boldc{\boldsymbol{c}}

\def\calA{\mathcal{A}}
\def\calB{\mathcal{B}}
\def\calC{\mathcal{C}}
\def\calT{\mathcal{T}}
\def\OR{\overline{\mathbb{R}}}
\def\RR{\mathbb{R}}
\def\CC{\mathbb{C}}
\def\FF{\mathbb{F}}
\def\QQ{\mathbb{Q}}
\def\ZZ{\mathbb{Z}}
\def\NN{\mathbb{N}}
%\def\NN{\mathbb{Z}_{> 0}}
\def\Nzero{\mathbb{Z}_{\geq 0}}
\def\EE{\mathbb{E}}
\def\PP{\mathbb{P}}
\def\supp{\mathrm{supp}}
\def\diam{\mathrm{diam}}
\def\sp{\mathrm{span}}
\def\ker{\mathrm{ker}}
\def\Aut{\operatorname{Aut}}
\def\Inn{\operatorname{Inn}}
\def\Ker{\operatorname{Ker}}
%\def\sp{\mathrm{span}} %messes up align enviroment
\newcommand{\Mod}{\ (\mathrm{mod}\ )}
\newcommand{\rbr}[1]{\left( {#1} \right)}
\newcommand{\sbr}[1]{\left[ {#1} \right]}
\newcommand{\cbr}[1]{\left\{ {#1} \right\}}
\newcommand{\abr}[1]{\left\langle {#1} \right\rangle}
\newcommand{\abs}[1]{\left| {#1} \right|}
\newcommand{\norm}[1]{\left\|#1\right\|}
\def\one{\mathbf{1}}
\DeclareMathOperator*{\esssup}{ess\,sup}
\newcommand*\wc{{}\cdot{}}
%\newcommand*\wc{ \, \cdot \,}
%wc for wildcard
\renewcommand{\Re}{\operatorname{Re}}
\renewcommand{\Im}{\operatorname{Im}}
\newcommand{\sgn}{\textup{sgn\,}}


\setlength{\parindent}{0pt}
\setlength{\parskip}{11pt}

%\title{\parbox{14cm}{\centering{  Interior points of circle and sphere packings}}}
\begin{document}

\textbf{HW 9 - MATH 221A - Fall 2023 - Chris Lane}

Date: \hhmmsstime{} \ \today \ \ Git hash: 
\input{/Users/chlane/src/jayalane/2023H2HW/.git/refs/heads/main}

  \begin{lemma}
    For all integers $k, \ell \geq 0$ and $m \geq 1$,

    \[
    p ^ \ell \textrm{ divides }  { p^k m  \choose p^k } \ \ \textrm { if and only if } \ \ p ^ \ell \textrm{ divides } m
    \]
  \end{lemma}
\begin{exercise}
  Use the lemma above to prove the following generalization of Sylow's First Theorem.

  \begin{theorem}
    Let $G$ be a finite group such that $|G| = p^r m$, where $p \nmid m$ and $p$ is a prime.  For each
    $ 0 \leq k \leq r$, $G$ has a subgroup of order $p^k$.
  \end{theorem}
\end{exercise}
\begin{solution}
  So let $\ell = k$ and $n = m p ^ {r - \ell}$.  

  Then let $X$ be the set of subsets of $G$ of order $p ^ \ell$. Let $G$ act on $X$ by left multiplication.

  \[
  X = \{ S \subseteq G: |S| =  p^\ell \}
  \]

  \[
  g \cdot S \to gS
  \]

  Then, $ p ^ \ell \mid |X| \ \textrm{ iff } p ^ \ell | n$ but by construction, $ p^\ell \nmid n$, so
  $p^\ell \nmid |X|$.  Now
  $|X| = \sum _ { x \in G} |O_{G_x} |$.  So there is an orbit, say $S \in X$ whose order is not divisible by $p^\ell$

  (or else the total would be divisible by $p^\ell)$.   Let $P \subseteq G$ be the stabilizer of $S$ ,that is
  $ P = \{ x \in X : gS = S\}$.  So for any given $s \in S$, we have $P s \subseteq S$ so that $|P| = |Ps|$ (as all
  cosets are equal in size) and $|Ps| \leq |S| = p ^ \ell$ (orbit of $S$ under conjugation is not
  divisible by $p ^ \ell$ but $S$ is a subset of size $p ^ \ell$.

  Now, the orbit stabilizer theorem also gives us a condition, the orbit of $S$, $|O_S| = |G| / |P| = p^\ell n / |P|$.  The size of the  orbit of $S$ is not divisible by $p^\ell$ so $|P|$ must be and then is larger than $p^\ell$, hence $|P| = p^\ell$, and $P$ is the desired subgroup.  

  
\end{solution}

\begin{exercise}
  
  \begin{enumerate}[(a)]
  \item
    If $G/Z(G)$ is cyclic, show that $G = G(Z)$, and therefore $G$ is abelian.
  \item
    Show that every group of order $p^2$ is abelian.  
  \end{enumerate}
\end{exercise}
\begin{solution}
  
  \begin{enumerate}[(a)]
  \item
    Let $Z(G)$, the center of $G$, be written as $Z$.  Assume $G/Z$ is cyclic.  That means all $x \in G/Z$ can be written
    as $ xZ = g^n Z$ for some fixed $g \in G$.  Now, examine $f \in G$.  We want to show that $fh = hf$ for all $h \in G$. 
    We know that $f Z = g^n Z$ for some $ n \in \ZZ$.  That is, $f 1 = g^n z_1$ or $f = g^n z_1$ for some $z_1 \in Z$.
    
    Now pick some arbitrary $h \in G$.  Similar to $f$, $h = g^m z_2$ for some $m \in \ZZ$ and $z_2 \in Z$.
    
    So
    
    \begin{align*}
      f \cdot h &= g^n \cdot z_1 \cdot g^m \cdot z_2 \\
      &=  g^n \cdot g^m \cdot z_1 \cdot z_2 \textrm {, due to } z_1 \in Z  \\
      &=  g^m \cdot g^n \cdot z_1 \cdot z_2 \textrm {, due to cyclicity } \\ 
      &=  g^m \cdot g^n \cdot z_2 \cdot z_1 \textrm {, due to } z_1, z_2 \in Z \\
      &=  g^m \cdot z_2 \cdot g^n \cdot z_1 \textrm {, due to } z_2 \in Z \\
      &= h \cdot f
    \end{align*}
    
    So $f$ commutes with all $h \in G$ and therefore $f\in Z$.  $f$ was arbitrary
    so $G = Z$ and $G$ is abelian.  
  \item
    $|G| = p^2$, $p$ prime.  So $|Z(G)| \in \{ 1, p, p^2\}$.  There is an in class lemma to the
    effect that for a p-subgroup the center is non-trivial; I'm not entirely certain that's meant to
    apply when the p-subgroup isn't a subgroup but the group, but the logic is from the class equation;
    if $|Z(G)| = 1$ and all the other orbits are multiples of $p$ due to the lack of factors of the
    group order, the orbit sizes  won't add up to the group order.
    
    So we can eliminate $1$ as a contender for $|Z(G)|$.  If $|Z(G)| = p^2$, then it's abelian, as that
    means all elements commute with one another.
    
    In the case $|Z(G)| = p$, then $|G / Z(G)| = |G| / |Z(G)| = p^2 / p = p$.  But then
    the quotient is of prime order and hence must be cyclic, so by (a) the group is abelian.
    
  
    \qed
    
  \end{enumerate}
\end{solution}
\begin{exercise}
  Let $G$ be a group with $|G| = pqr$, where $p$, $q$, and $r$ are distinct primes and
  (without loss of generality), $p > q > r$.  Prove :
  \begin{enumerate}[(a)]
  \item
    $|G| \geq 1 + n_p (p-1) + n_q (q-1) + n_r (r - 1)$.
  \item
    If $n_p, n_q, n_r > 1$, then $np = qr$, $n_q \geq p$ and $n_r \geq q$.
  \item
    $|G|$ is not simple.
  \end{enumerate}
\end{exercise}
\begin{solution}
  \begin{enumerate}[(a)]
  \item
    Counting members of $G$ of various orders, we get:
    one member at least of order one, $1$.

    Members of Sylow p-subgroups, other than 1, have order $p$. We
    have $n_p$ such subgroups; they don't intersect non-trivially
    because the degree of the prime is one); they contribute $n_p (p-1)$ members at least.

    Similarly, for $q$, they contribute $n_q (q-1)$ members at least.

    And likewise for $r$, whose Sylow r-subgroups contribute $n_r (r - 1)$ members at least.

    Adding them all together, we get as desired:

    \[
    | G | \geq  1 + n_p (p-1) + n_q (q-1) + n_r (r - 1)
    \]

  \item
    The third Sylow theorem here gives us $n_p \equiv 1 (\mod p)$ as well as $n_p | qr$.  Since
    $p >q$ and $p > r$, this forces $n_p$ to be $qr$.  The only choices given the prime factorization
    here are $1$, $p$, $q$, $r$, $pq$, $qr$, $pr$ or $pqr$, but $n_p \equiv 1 (\mod p$ rules out
    $p$, $pq$, $pr$ or $pqr$.  $p > q > r$ rules out $q$ or $r$, $n_p > 1$ rules out $1$ leaving just $n_p = qr$.

    Similarly for $n_q$, we have $n_q \equiv 1 ( \mod q ) $ and $n_q | pr$.  We assume $  n_q > 1$
    and $ r$ is too small to be $\equiv 1 (\mod q)$ so that leaves $n_q \geq p$ (it could be $p$ or $rp$).

    The same logic applies to $n_r$, leaving it forced to be a multiple of $q$.

    \qed
  \item
    Combining (a) and (b), and assuming simplicity (and the consequent forcing of the $n_i$s $>1$,
    \begin{align*}
    |G| \geq 1 + n_p (p -1) + n_q(q - 1) + n_r(r-1) \\
    |G| \geq 1 + qr (p -1) + p(q - 1) + q(r-1) \\
    |G| \geq 1 + qrp -qr + pq - p + qr-q \\
    |G| \geq 1 + pqr + pq - p -q
    \end{align*}

    For two disctinct primes, $pq -p -q \geq 1$ as the smallest it could be is $2 \times 3 - 2 - 3 = 1$

    So
    \[
    |G| \geq 1 + pqr + 1
    \]

    But $|G|$ is $pqr$, so there is a contradiction and the assumption of simplicity is proven false.
    \qed
  \end{enumerate}
\end{solution}
\begin{exercise}
  Show that if $G$ is a grup of order 80, then $G$ is not simple.  
\end{exercise}

\begin{solution}
  $|G| = 80 = 2^4 * 5$.

  Looking at Sylow 2-subgroups, we get $n_2 \equiv 1 ( \mod 2)$ and $ n_2 | 5$, so
  $n_2 \in \{ 1, 5\}$.  If $n_2 = 1$, we are done as that forces the Sylow 2-subgroup
  to be normal.

  So assume $n_1 = 5$.  Now looking at the Sylow 5-subgroups, we have $n_5  \equiv 1 (\mod 5)$
  and $n_5 | 2^4$.  So $n_5 \in \{1, 16\} $.  Again, if $n_5 = 1$, we are done, so assume
  $n_5 = 16$.  There's only 1 power of 5 in $|G|$ so they don't intersect non-trivially, so
  that gives us $4 \times 16$  elements of order 5, leaving $80 - 64 = 16$ which is only
  enough room for one Sylow p-subgroup, which is a contradiction, so either $n_2 = 1$ or $n_5 = 1$.
  \qed
  
\end{solution}

\begin{exercise}
  Show that if $G$ is a grup of order $2^4 5^6$, then $G$ is not simple.  
\end{exercise}

\begin{exercise}
  Prove: If $G$ is a group of order $p^r m$ where $ p \nmid m$, then $G$ has a subgroup
  of index $n_p$
\end{exercise}

\begin{solution}
  Let $X$ be the collection of Sylow p-subgroups of $G$.  So $|X| = n_p$.  Note that $G$ acts on $X$
  by conjugation.  Let $P$ be any Sylow p-subgroup of $G$.  The stabilizer of $P$ is:

  \[
  N_G(P) = \{ g \in G : gPg^{-1} = P \}
  \]

  This is a subgroup: $ ePe = P$ so $e \in N_G(P)$.  And it's closed under the group operation:
  \begin{align*}
    f \cdot g \cdot P &= f \cdot gPg^{-1} \\
    &= fgPg^{-1}f^{-1}  \\
    &= fgP(fg)^{-1}  \\
    &= fg \cdot P
  \end{align*}
\end{solution}

  The index of $N_G(P)$ is $n_p$ as the Sylow p-subgroups are all conjugate to one another, .
  \qed
  
\begin{solution}
  $|G| = 80 = 2^4 * 5^6$.

  Looking at Sylow 2-subgroups, we get $n_2 \equiv 1 ( \mod 2)$ and $ n_2 | 5$, so
  $n_2 \in \{ 1, 5\}$.  If $n_2 = 1$, we are done as that forces the Sylow 2-subgroup
  to be normal.

  So assume $n_1 = 5$.  Now looking at the Sylow 5-subgroups, we have
  $n_5 \equiv 1 (\mod 5)$ and $n_5 | 2^4$.  So $n_5 \in \{1, 16\} $.
  Again, if $n_5 = 1$, we are done, so assume $n_5 = 16$.  Then by the
  above problem, there is a subgroup of $G$ of order 16, which,
  according to Homework 8, Problem 2, means that $G$ is isomorphic to
  $S_{16}$.  However, this also is a contradiction, as $|S_{16}| = 16!$ but
  $|G| = 2^4 * 5^6$.  $16!$ has only 5, 10, and 15 as multiplicands that carry a factor of
  $5$, so the highest power of $5$ that divides $16$ is $5^3$, while $|G|$ carries $5^6$.

  So $|G|$ cannot be simple.  
  \qed
  
\end{solution}

\begin{comment}
  \begin{exercise}
    problem
  \end{exercise}
  \begin{solution}
    \begin{enumerate}[(a)]
    \item
      first answer
    \end{enumerate}
  \end{solution}
\end{comment}


\end{document}
